\documentclass[]{elsarticle} %review=doublespace preprint=single 5p=2 column
%%% Begin My package additions %%%%%%%%%%%%%%%%%%%

\usepackage[hyphens]{url}

  \journal{European Economic Review} % Sets Journal name

\usepackage{lineno} % add

\usepackage{graphicx}
%%%%%%%%%%%%%%%% end my additions to header

\usepackage[T1]{fontenc}
\usepackage{lmodern}
\usepackage{amssymb,amsmath}
\usepackage{ifxetex,ifluatex}
\usepackage{fixltx2e} % provides \textsubscript
% use upquote if available, for straight quotes in verbatim environments
\IfFileExists{upquote.sty}{\usepackage{upquote}}{}
\ifnum 0\ifxetex 1\fi\ifluatex 1\fi=0 % if pdftex
  \usepackage[utf8]{inputenc}
\else % if luatex or xelatex
  \usepackage{fontspec}
  \ifxetex
    \usepackage{xltxtra,xunicode}
  \fi
  \defaultfontfeatures{Mapping=tex-text,Scale=MatchLowercase}
  \newcommand{\euro}{€}
\fi
% use microtype if available
\IfFileExists{microtype.sty}{\usepackage{microtype}}{}
\usepackage[]{natbib}
\bibliographystyle{plainnat}

\ifxetex
  \usepackage[setpagesize=false, % page size defined by xetex
              unicode=false, % unicode breaks when used with xetex
              xetex]{hyperref}
\else
  \usepackage[unicode=true]{hyperref}
\fi
\hypersetup{breaklinks=true,
            bookmarks=true,
            pdfauthor={},
            pdftitle={An Independent European Macro? A History of European Macroeconomics through the Lens of the European Economic Review},
            colorlinks=false,
            urlcolor=blue,
            linkcolor=magenta,
            pdfborder={0 0 0}}

\setcounter{secnumdepth}{5}
% Pandoc toggle for numbering sections (defaults to be off)


% tightlist command for lists without linebreak
\providecommand{\tightlist}{%
  \setlength{\itemsep}{0pt}\setlength{\parskip}{0pt}}



\usepackage{float}
\usepackage{booktabs}
\usepackage{longtable}
\usepackage{array}
\usepackage{multirow}
\usepackage{wrapfig}
\usepackage{colortbl}
\usepackage{pdflscape}
\usepackage{tabu}
\usepackage{threeparttable}
\usepackage{threeparttablex}
\usepackage[normalem]{ulem}
\usepackage{makecell}
\usepackage{xcolor}



\begin{document}


\begin{frontmatter}

  \title{An Independent European Macro? A History of European
Macroeconomics through the Lens of the European Economic Review}
    \author[]{}
  
      \cortext[cor1]{Corresponding author}
  
  \begin{abstract}
  
  \end{abstract}
  
 \end{frontmatter}

\newpage

\hypertarget{appendices}{%
\section*{Appendices}\label{appendices}}
\addcontentsline{toc}{section}{Appendices}

\hypertarget{a---summary-tables}{%
\subsection*{A - Summary Tables}\label{a---summary-tables}}
\addcontentsline{toc}{subsection}{A - Summary Tables}

Here are the tables listing the different clusters and topics, with
their synthetic indicator of how much they are ``European''.

\begin{table}[!h]

\caption{\label{tab:summary-communities}Summary of Bibliographic Communities}
\centering
\fontsize{9}{11}\selectfont
\begin{tabular}[t]{lr}
\toprule
Communities & Differences\\
\midrule
\cellcolor{gray!6}{Modeling Consumption \& Production} & \cellcolor{gray!6}{2.0986873}\\
Disequilibrium \& Keynesian Macro & 2.0014930\\
\cellcolor{gray!6}{International Macroeconomics \& Target Zone} & \cellcolor{gray!6}{1.3359739}\\
Optimal Taxation 1 & 1.3235057\\
\cellcolor{gray!6}{Political Economics of Central Banks} & \cellcolor{gray!6}{1.0265264}\\
\addlinespace
Exchange Rate Dynamics & 0.7201769\\
\cellcolor{gray!6}{Taxation, Tobin's Q \& Monetarism} & \cellcolor{gray!6}{0.6388438}\\
Theory of Unemployment \& Job Dynamics & 0.5181369\\
\cellcolor{gray!6}{Capital \& Income Taxation} & \cellcolor{gray!6}{0.4078902}\\
Target Zone \& Currency Crises & 0.4058932\\
\addlinespace
\cellcolor{gray!6}{Coordination \& Sunspots 2} & \cellcolor{gray!6}{0.3465849}\\
Optimal Taxation 2 & 0.2787993\\
\cellcolor{gray!6}{Monetary Policy, Financial Transmission \& Cycles 2} & \cellcolor{gray!6}{0.2314883}\\
Business Cycles, Cointegration \& Trends & 0.2154401\\
\cellcolor{gray!6}{Taxation, Debt \& Growth} & \cellcolor{gray!6}{-0.0067936}\\
\addlinespace
Terms of Trade \& Devaluation & -0.0817679\\
\cellcolor{gray!6}{Endogenous Growth} & \cellcolor{gray!6}{-0.1105033}\\
RBC & -0.2094801\\
\cellcolor{gray!6}{Monetary Policy, Financial Transmission \& Cycles 1} & \cellcolor{gray!6}{-0.2379855}\\
Coordination \& Sunspots 1 & -0.3313256\\
\addlinespace
\cellcolor{gray!6}{Exchange Rate Dynamics \& Expectations} & \cellcolor{gray!6}{-0.4273431}\\
Monetary Policy, Target \& Output Gap & -0.6211658\\
\cellcolor{gray!6}{REH, Monetary Policy \& Business Cycles} & \cellcolor{gray!6}{-0.6758668}\\
Inflation, Interest Rates \& Expectations & -0.7403062\\
\cellcolor{gray!6}{Monetary Approach of Balance of Payments} & \cellcolor{gray!6}{-0.7658536}\\
\addlinespace
Credit Rationing, Rational Expectations \& Imperfect Information & -0.8270409\\
\cellcolor{gray!6}{Inflation \& Rigidities} & \cellcolor{gray!6}{-0.8396888}\\
Demand for Money & -0.8564033\\
\cellcolor{gray!6}{New Theory of Money: Search, Bargaining...} & \cellcolor{gray!6}{-1.2139877}\\
Permanent Income Hypothesis \& Life-Cycle & -1.4920219\\
\addlinespace
\cellcolor{gray!6}{Monetary Economics \& Demand for Money} & \cellcolor{gray!6}{-1.6768415}\\
Intergenerational Model, Savings and Consumption & -3.2194818\\
\cellcolor{gray!6}{Marginal Taxation} & \cellcolor{gray!6}{-3.3179236}\\
\bottomrule
\end{tabular}
\end{table}

\newpage

\begingroup\fontsize{9}{11}\selectfont

\begin{ThreePartTable}
\begin{TableNotes}
\item \textit{Note: } 
\item Differences values are the sum of (i) the difference in the gamma mean between EER and Top 5; (ii) the same difference but between European-based and US-based authors
\end{TableNotes}
\begin{longtable}[t]{>{}l>{}r>{\raggedright\arraybackslash}m{25em}}
\caption{\label{tab:summary-topics}Summary of Topics}\\
\toprule
Topics & Differences & Terms with the highest frex value\\
\midrule
\endfirsthead
\caption[]{Summary of Topics \textit{(continued)}}\\
\toprule
Topics & Differences & Terms with the highest frex value\\
\midrule
\endhead

\endfoot
\bottomrule
\insertTableNotes
\endlastfoot
Topic 46 & 0.030 & german;
money
demand;
germany;
unite
kingdom;
unite;
\cellcolor{gray!6}{kingdom}\\
Topic 22 & 0.026 & political;
oecd
country;
oecd;
world;
country;
index\\
Topic 6 & 0.024 & real
exchange;
real
exchange
rate;
exchange
rate;
flexible
exchange;
flexible
exchange
rate;
target
\cellcolor{gray!6}{zone}\\
Topic 25 & 0.023 & real
wage;
contract;
employment;
capacity;
wage;
stickiness\\
Topic 37 & 0.021 & unemployment;
job;
unemployment
rate;
creation;
flow;
phillips
\cellcolor{gray!6}{curve}\\
\addlinespace
Topic 3 & 0.014 & system;
monetary
expansion;
monetary
system;
union;
expansion;
stability\\
Topic 28 & 0.013 & exchange
market;
foreign
exchange
market;
intervention;
foreign
exchange;
transaction;
transaction
\cellcolor{gray!6}{cost}\\
Topic 39 & 0.013 & trade
balance;
trade;
wealth;
relative
price;
balance;
external\\
Topic 8 & 0.012 & credibility;
strategic;
policy;
maker;
economic
policy;
policy
\cellcolor{gray!6}{rule}\\
Topic 20 & 0.011 & inflation
target;
target;
stabilize;
central
bank;
central;
length\\
\addlinespace
Topic 45 & 0.009 & power
parity;
purchase
power
parity;
purchase
power;
power;
purchase;
\cellcolor{gray!6}{parity}\\
Topic 44 & 0.008 & level;
national
income;
inflationary;
equation;
price
level;
money
balance\\
Topic 40 & 0.006 & economic
growth;
growth
rate;
productivity
growth;
growth;
fast;
\cellcolor{gray!6}{region}\\
Topic 43 & 0.005 & federal;
signal;
monetary
policy;
federal
reserve;
revision;
feed\\
Topic 17 & 0.004 & indexation;
distortion;
labor
market;
labor;
product;
\cellcolor{gray!6}{corporate}\\
\addlinespace
Topic 11 & 0.003 & walrasian;
competitive;
temporary;
existence;
search;
equilibrium\\
Topic 30 & 0.003 & business
cycle;
business;
cycle;
real
business
cycle;
real
business;
\cellcolor{gray!6}{volatility}\\
Topic 31 & 0.002 & welfare
cost;
cost;
welfare;
bear;
survey;
household\\
Topic 13 & 0.001 & unanticipated;
activity;
economic
activity;
national;
economy;
\cellcolor{gray!6}{gap}\\
Topic 26 & 0.001 & inflation;
inflation
rate;
relative;
evidence;
dispersion;
nominal
price\\
\addlinespace
Topic 29 & 0.001 & capital
market;
mobility;
capital
mobility;
capital;
imperfect;
\cellcolor{gray!6}{intensity}\\
Topic 38 & 0.001 & asset
price;
financial
market;
stock
market;
return;
asset
market;
stock\\
Topic 4 & -0.001 & macroeconomics;
rich;
history;
robert;
divide;
\cellcolor{gray!6}{lead}\\
Topic 34 & -0.001 & plan;
stage;
multiple
equilibrium;
option;
crisis;
currency\\
Topic 15 & -0.002 & production;
class;
factor;
identical;
preference;
\cellcolor{gray!6}{input}\\
\addlinespace
Topic 23 & -0.002 & short
run;
run;
short;
burden;
indirect;
externality\\
Topic 24 & -0.002 & term;
spread;
short
term;
term
structure;
premium;
\cellcolor{gray!6}{structure}\\
Topic 36 & -0.002 & spend;
government
spend;
deficit;
fiscal;
government;
government
debt\\
Topic 47 & -0.002 & investment;
monopolistic;
dynamic;
competition;
macroeconomic;
\cellcolor{gray!6}{replace}\\
Topic 50 & -0.002 & budget
constraint;
constraint;
project;
budget;
bad;
loan\\
\addlinespace
Topic 35 & -0.004 & likelihood;
variable;
estimation;
autoregressive;
variance;
endogenous
\cellcolor{gray!6}{variable}\\
Topic 18 & -0.005 & generation;
overlap;
overlap
generation;
social
security;
live;
generation
model\\
Topic 48 & -0.005 & liquidity;
credit;
debt;
insurance;
access;
\cellcolor{gray!6}{investor}\\
Topic 32 & -0.006 & standard;
gold;
dollar;
reserve;
price
level;
size\\
Topic 1 & -0.007 & inventory;
hold;
association;
century;
create;
\cellcolor{gray!6}{rationally}\\
\addlinespace
Topic 5 & -0.007 & price
adjustment;
oil;
price;
commodity
price;
sticky;
import\\
Topic 21 & -0.007 & process;
procedure;
property;
incentive;
build;
\cellcolor{gray!6}{endogenous}\\
Topic 9 & -0.008 & skill;
asymmetric
information;
program;
change;
research;
complementarity\\
Topic 27 & -0.008 & perfect
foresight;
foresight;
time
vary;
time;
perfect;
continuous
\cellcolor{gray!6}{time}\\
Topic 41 & -0.008 & optimal;
optimal
tax;
growth
model;
function;
optimal
policy;
optimal
\cellcolor{gray!6}{taxation}\\
\addlinespace
Topic 49 & -0.008 & report;
composition;
regime;
critique;
puzzle;
\cellcolor{gray!6}{profit}\\
Topic 7 & -0.010 & control;
stochastic;
game;
equivalence;
equivalent;
solution\\
Topic 42 & -0.010 & stationary;
rational
expectation
equilibrium;
expectation
equilibrium;
expectation;
unique;
rational
\cellcolor{gray!6}{expectation}\\
Topic 33 & -0.012 & depression;
theory;
subsequent;
classical;
pure;
principle\\
Topic 12 & -0.015 & lm;
risk
aversion;
utility
function;
aversion;
intertemporal;
\cellcolor{gray!6}{risk}\\
\addlinespace
Topic 2 & -0.017 & money
supply;
money
stock;
money;
fix
exchange;
supply;
fix
exchange
rate\\
Topic 10 & -0.018 & tax;
capital
income;
income
tax;
tax
system;
redistribution;
income
taxation\\
Topic 16 & -0.018 & public;
strategy;
finance;
local;
provision;
desirable\\
Topic 19 & -0.021 & expect
rate;
cash;
sargent;
nominal;
expect
inflation;
\cellcolor{gray!6}{expect}\\
Topic 14 & -0.022 & income
distribution;
labor
income;
permanent;
sensitivity;
permanent
income;
income\\*
\end{longtable}
\end{ThreePartTable}
\endgroup{}

\newpage

\hypertarget{appendix}{%
\subsection*{B - Information on the Methods}\label{appendix}}
\addcontentsline{toc}{subsection}{B - Information on the Methods}

\hypertarget{corpus}{%
\subsubsection*{B.1. Corpus Creation}\label{corpus}}
\addcontentsline{toc}{subsubsection}{B.1. Corpus Creation}

For the present study we used two different corpora. The first corpus is
composed of all EER articles and allows us to track how publications,
citations, references and authors affiliations evolved since the
creation of the journal in 1969 up to the 2002. The second corpus is
composed of all macroeconomic articles published in the top five
economics journals and the EER. Macroeconomic articles are identified
thanks to the former and new classification of the JEL codes
\citep{jel1991}.\footnote{See \ref{eer-top5-macro} for the list of JEL
  codes used.} This is used as the basis for topic modeling and
bibliographic coupling analysis to contrast the top macroeconomics
publications authored by European-based and US-based authors, and/or
published in top 5 journals and in the EER.

\hypertarget{eer-publications}{%
\paragraph*{EER Publications}\label{eer-publications}}
\addcontentsline{toc}{paragraph}{EER Publications}

For the creation of the first corpus composed of all EER articles, we
used a mix of \emph{Web of Science} (WoS) and \emph{Scopus}. While WoS
has all articles of the EER between 1969-1970 and 1974-2002, it is
missing most articles published between 1971 and 1973. To make up for
the missing data, we use Scopus to complete the dataset. This operation
required normalization of the Scopus dataset, and manual cleaning of
variables that were missing from Scopus compared to WoS. This mostly
includes cleaning the references to match \emph{Scopus} references with
WoS ones, and identification of author's affiliation.

Moreover, given that the size of our corpus is modest, we made an
extensive semi-automatic cleaning of references to improve references
identification by adding the most commonly cited books, book chapter,
and articles that are not otherwise identified in WoS when possible.

\hypertarget{eer-top5-macro}{%
\paragraph*{EER and Top 5 Macroeconomics
Articles}\label{eer-top5-macro}}
\addcontentsline{toc}{paragraph}{EER and Top 5 Macroeconomics Articles}

The construction of this corpus is made in multiple steps:

\begin{enumerate}
\def\labelenumi{\arabic{enumi}.}
\item
  Identifying macroeconomics articles

  \begin{itemize}
  \item
    We identified all articles published in macroeconomics using JEL
    codes related to macroeconomics (we get JEL codes of Top 5 and EER
    articles thanks to the Econlit database). We consider that an
    article is a macroeconomics article if it has one of the following
    codes:

    \begin{itemize}
    \tightlist
    \item
      For old JEL codes (pre-1991): 023, 131, 132, 133, 134, 223, 311,
      313, 321, 431, 813, 824.
    \item
      For new JEL codes (1991 onward): all E, F3 and F4.\footnote{The
        new classification has a clear categorisation of Macroeconomics
        (the letter `E'), but we had F3 and F4 as they deal with
        international macroeconomics. For the older JEL codes, we use
        the table of correspondence produce by the \emph{Journal of
        Economic Literature} itself \citep{jel1991}.}.
    \end{itemize}
  \end{itemize}
\item
  Using these JEL codes, we match econlit articles with WoS articles
  when they shared the same:

  \begin{itemize}
  \tightlist
  \item
    Journal, Volume, First Page
  \item
    Year, Journal, First Page, Last Page
  \item
    Year, Volume, First Page, Last Page
  \item
    First Author, Year, Volume, First Page
  \item
    First Author, Title, Year
  \item
    Title, Year, First Page
  \end{itemize}
\end{enumerate}

Out of the 3592 articles we get in econlit, we matched 3428 of them in
WoS.\footnote{Most of the unmatched articles are not `articles' properly
  speaking: they often are reply and comments on other published
  articles. (Investigate this deeper)}

\begin{enumerate}
\def\labelenumi{\arabic{enumi}.}
\setcounter{enumi}{2}
\item
  Using this list of articles in WoS, we took all articles in
  macroeconomics published in the EER (Corpus 1 improved with Scopus)
  and in the top five journals (\emph{American Economic Review},
  Econometrica, \emph{Review of Economic Studies}, \emph{Journal of
  Political Economy}, \emph{Quarterly Journal of Economics}).
\item
  Finally, we were able to collect abstracts:

  \begin{itemize}
  \tightlist
  \item
    using \emph{Scopus} for the EER. All abstracts have been matched
    with the EER corpus.
  \item
    using \emph{Microsoft Academics} to collect the highest number of
    available abstracts for the Top 5 as too many abstracts were missing
    in WoS or \emph{Scopus}. The abstracts extracted from this database
    are matched with our WoS Top 5 corpus using
  \end{itemize}
\end{enumerate}

\hypertarget{b.2.-variable-creation}{%
\subsubsection*{B.2. Variable creation}\label{b.2.-variable-creation}}
\addcontentsline{toc}{subsubsection}{B.2. Variable creation}

\hypertarget{author-affiliation}{%
\paragraph*{Authors' affiliation}\label{author-affiliation}}
\addcontentsline{toc}{paragraph}{Authors' affiliation}

Authors' affiliations information were extracted from WoS. However, the
affiliations are not per author, but instead per institutional
departments per paper. For example, in the case of an article with two
authors from the same department, the department (and institution or
country associated with it) is only counted once. Similarly, a
single-authored article where the author has three affiliations can
result in one article having three affiliations. While in some cases we
can inferred the institutional affiliation for each author (e.g., one
institution, multiple authors), in others we cannot (e.g., two
institutions, three authors). For example, in an article with two
authors from Princeton and one author from Stanford, we only know that
the article was written by at least one author from Princeton and at
least one from Stanford, but not that the individual ratio was two
third.

We restructure the information in two ways.

First, for each article, we only kept one occurrence of each unique
institutions (university, research institutes\ldots) to avoid the
multiplication of observations resulting from the variety of departments
observed in some institutions. In other words, for each article, authors
are group by their institutional affiliation not by their department or
research team.

Second, and more importantly, for the purpose of our analysis, we mostly
looked at the share of papers authored by European-based and US-based
economists. While we do not have individual affiliation, we know with
certainty when a paper has only European authors, only American authors,
or a mix of the two. For this reason, while the share of institutions
within the corpus is only an estimation based on the occurrences of
affiliation, the information generated to identify US authored papers
and European authored paper is certain.

\hypertarget{network}{%
\subsubsection*{B.3. Bibliographic Coupling and Cluster
Detection}\label{network}}
\addcontentsline{toc}{subsubsection}{B.3. Bibliographic Coupling and
Cluster Detection}

A first way to identify potential differences between European and
American macroeconomics is to find articles written by Europeans and
published in European journals, resembling each others but dissimilar to
American articles. To do that, we used bibliographic coupling
techniques. In a bibliographic coupling network, a link is created
between two articles when they have one or more references in common.
The more references that two articles have in common, the stronger the
link. Bibliographic coupling is one way to measure how similar two
articles are in a corpus. To normalize and weight the link between two
articles, we used the refined bibliographic coupling strength of
\citet{shen2019}. This method normalized and weight the strength between
articles by taking into account two important elements

\begin{enumerate}
\def\labelenumi{\arabic{enumi}.}
\tightlist
\item
  The size of the bibliography of the two linked articles. It means that
  common references between two articles with long bibliography are
  weighted as less significant since the likeliness of potential common
  references is higher. Conversely, common references between two
  articles with a short bibliography is weighted as more significant.
\item
  The number of occurrences of each reference in the overall corpus.
  When a reference is shared between two articles, it is weighted as
  less significant if it is very common reference across the entire
  corpus and very significant if it is scarcely cited. The assumption is
  that a very rare common reference points to a higher content
  similarity between two articles than a highly cited reference.
\end{enumerate}

For all macroeconomics articles published in the EER and in the Top 5,
we build the networks with \texttt{time\_window}-year overlapping
windows. This results in 23.

We used Leiden detection algorithm \citep{traag2019} that optimize the
modularity on each network to identify groups of articles that are
similar to each others and dissimilar to the rest of the network. We use
a resolution of 1 with 1000 iterations. This results in 466 across all
networks. Because networks have a lot of overlaps, many clusters between
two periods are composed of the same articles. To identify these
clusters that are very similar between two time windows, we considered
that \emph{(i)} if at least 55\% of the articles in a community of the
first time window where in the same cluster in the second time window,
and that \emph{(ii)} if the cluster was also composed by at least 55\%
of articles of the first time window, \emph{then} it is the same cluster

Simply put, if two clusters share a high number of articles, and are
both mostly composed by these shared articles, they are considered the
same cluster.

This gives us 154, with 33 that are at least 0.04\% of a network at any
given point and are stable enough to exists for at least 2 time windows.

For each clusters, we identified the American or European oriented
nature of its publications and authors. A first measure we used is the
over/under representation of European/American authors in the cluster.
For each cluster, we subtracted the relative share of European authors
to American authors in the cluster, to the relative share of European
authors to American on the same time window of the cluster leading us to
the first index:\newline

\(\text{Author EU/US Orientation}=\log(\frac{\text{Share Of European Authored Articles In The Cluster}}{\text{Share Of American Authored Articles In The Cluster}})-\log(\frac{\text{Share Of European Authored Articles In The Time Window}}{\text{Share Of American Authored Articles In The Time Window}})\)\newline

We then use a second similar index for the publication venue of the
articles in the cluster. For each cluster, we subtracted the relative
share of EER publications to Top 5 publications in the cluster, to the
relative share of EER publications to Top 5 publications on the same
time window of the cluster:\newline

\(\text{Journal EU/US Orientation}=\log(\frac{\text{Share Of EER Articles In The Cluster}}{(1-\text{Share Of EER Articles In The Cluster})})-\log(\frac{\text{Share Of EER Articles In The Time Window}}{(1-\text{Share Of EER Articles In The Time Window})})\)\newline

To get an overall index score of the European/US orientation of
clusters, we simply sum the two previous index:\newline

\(\text{Overall EU/US Orientation}=\text{Author EU/US Orientation} + \text{Journal EU/US Orientation}\)\newline

Finally, clusters are placed on a scatterplot with the Y-axis for the
EER vs Top 5 score, and the X-Axis for the American vs European authors
score. The size of the points captures the size of the cluster with the
number of articles that are in it, and the color of the cluster is
simply the sum of the two Y and X scores.

\hypertarget{topic}{%
\subsubsection*{B.4. Topic Modelling}\label{topic}}
\addcontentsline{toc}{subsubsection}{B.4. Topic Modelling}

\hypertarget{preprocessing}{%
\paragraph*{Preprocessing}\label{preprocessing}}
\addcontentsline{toc}{paragraph}{Preprocessing}

We have several steps to clean our texts before running our topic
models:

\begin{enumerate}
\def\labelenumi{\arabic{enumi}.}
\tightlist
\item
  Once we have our corpus, we merge titles and abstracts together for
  all EER and Top 5 articles.
\item
  We use the \emph{tidytext} and \emph{tokenizers} R packages to
  `tokenise' the resulting texts (when there is no abstract, only the
  title if thus tokenise)?\footnote{See Silge J, Robinson D (2016).
    ``tidytext: Text Mining and Analysis Using Tidy Data Principles in
    R.'' \emph{JOSS}, \emph{1}(3) and Lincoln A. Mullen et al., ``Fast,
    Consistent Tokenization of Natural Language Text,'' Journal of Open
    Source Software 3, no.23 (2018): 655.} Tokenisation is the process
  of transforming human-readable text into machine readable objects.
  Here, the text is split in unique words (unigrams), bigrams (pair of
  words) and trigrams. In other words, to each article is now associated
  a list of unigrams, bigrams and trigrams, some appearing several times
  in the same title + abstract.
\item
  Stop words are removed using the \emph{Snowball}
  dictionary.\footnote{See
    http://snowball.tartarus.org/algorithms/english/stop.txt.} We add to
  this dictionary some current verbs in abstract like ``demonstrate'',
  ``show'', ``explain''. Such verbs are likely to be randomly
  distributed in abstracts, but we want to limit the noise as much as
  possible.
\item
  We lemmatise the words using the \emph{textstem} package.\footnote{Rinker,
    T. W. (2018). textstem: Tools for stemming and lemmatizing text
    version 0.1.4. Buffalo, New York.} The lemmatisation is the process
  of grouping words together according to their ``lemma'' which depends
  on the context. For instance, different form of a verb are reduced to
  its infinitive form. The plural of nouns are reduced to the singular.
\end{enumerate}

\hypertarget{choosing-the-number-of-topics}{%
\paragraph*{Choosing the number of
topics}\label{choosing-the-number-of-topics}}
\addcontentsline{toc}{paragraph}{Choosing the number of topics}

We use the Correlated Topic Model \citep{blei2007} method implemented in
the \emph{STM} R package.\footnote{Roberts ME, Stewart BM, Tingley D
  (2019). ``stm: An R Package for Structural Topic Models.''
  \emph{Journal of Statistical Software}, \emph{91}(2), 1-40.}

From the list of words we have tokenised, cleaned and lemmatised, we
test different thresholds and choices by running different models:

\begin{itemize}
\tightlist
\item
  by exluding trigrams or not;
\item
  by removing the terms that are present in less than 0.5\% of the
  Corpus (), 1\% () and 2\% ();
\item
  by removing articles with less than 6 words or with less than 12
  words.\footnote{Here, only articles with no abstract are impacted.}
\end{itemize}

Crossing all these criteria, we thus have 12 different possible
combinations. For each of these 12 different combinations, we have run
topic models for different number of topics from 20 to 110 with a gap of
5. The chosen model integrates trigram, removes only terms that appear
in less than 0,5\% of the documents and keep all articles if they have
more than 6 words in their title + abstract. We choose to keep the model
with 55 topics.

We have chosen the criteria and the number of topics by comparing the
performance of the different models in terms of the FREX value
\citep{bischof2012}.

\bibliography{bibliography.bib}


\end{document}
