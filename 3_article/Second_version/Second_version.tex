% Options for packages loaded elsewhere
\PassOptionsToPackage{unicode}{hyperref}
\PassOptionsToPackage{hyphens}{url}
%
\documentclass[
  12pt,
  onecolumn]{article}
\usepackage{amsmath,amssymb}
\usepackage{lmodern}
\usepackage{setspace}
\usepackage{iftex}
\ifPDFTeX
  \usepackage[T1]{fontenc}
  \usepackage[utf8]{inputenc}
  \usepackage{textcomp} % provide euro and other symbols
\else % if luatex or xetex
  \usepackage{unicode-math}
  \defaultfontfeatures{Scale=MatchLowercase}
  \defaultfontfeatures[\rmfamily]{Ligatures=TeX,Scale=1}
\fi
% Use upquote if available, for straight quotes in verbatim environments
\IfFileExists{upquote.sty}{\usepackage{upquote}}{}
\IfFileExists{microtype.sty}{% use microtype if available
  \usepackage[]{microtype}
  \UseMicrotypeSet[protrusion]{basicmath} % disable protrusion for tt fonts
}{}
\makeatletter
\@ifundefined{KOMAClassName}{% if non-KOMA class
  \IfFileExists{parskip.sty}{%
    \usepackage{parskip}
  }{% else
    \setlength{\parindent}{0pt}
    \setlength{\parskip}{6pt plus 2pt minus 1pt}}
}{% if KOMA class
  \KOMAoptions{parskip=half}}
\makeatother
\usepackage{xcolor}
\usepackage[a4paper,bindingoffset=0mm,inner=30mm,outer=30mm,top=30mm,bottom=30mm]{geometry}
\usepackage{graphicx}
\makeatletter
\def\maxwidth{\ifdim\Gin@nat@width>\linewidth\linewidth\else\Gin@nat@width\fi}
\def\maxheight{\ifdim\Gin@nat@height>\textheight\textheight\else\Gin@nat@height\fi}
\makeatother
% Scale images if necessary, so that they will not overflow the page
% margins by default, and it is still possible to overwrite the defaults
% using explicit options in \includegraphics[width, height, ...]{}
\setkeys{Gin}{width=\maxwidth,height=\maxheight,keepaspectratio}
% Set default figure placement to htbp
\makeatletter
\def\fps@figure{htbp}
\makeatother
\setlength{\emergencystretch}{3em} % prevent overfull lines
\providecommand{\tightlist}{%
  \setlength{\itemsep}{0pt}\setlength{\parskip}{0pt}}
\setcounter{secnumdepth}{5}
\newlength{\cslhangindent}
\setlength{\cslhangindent}{1.5em}
\newlength{\csllabelwidth}
\setlength{\csllabelwidth}{3em}
\newlength{\cslentryspacingunit} % times entry-spacing
\setlength{\cslentryspacingunit}{\parskip}
\newenvironment{CSLReferences}[2] % #1 hanging-ident, #2 entry spacing
 {% don't indent paragraphs
  \setlength{\parindent}{0pt}
  % turn on hanging indent if param 1 is 1
  \ifodd #1
  \let\oldpar\par
  \def\par{\hangindent=\cslhangindent\oldpar}
  \fi
  % set entry spacing
  \setlength{\parskip}{#2\cslentryspacingunit}
 }%
 {}
\usepackage{calc}
\newcommand{\CSLBlock}[1]{#1\hfill\break}
\newcommand{\CSLLeftMargin}[1]{\parbox[t]{\csllabelwidth}{#1}}
\newcommand{\CSLRightInline}[1]{\parbox[t]{\linewidth - \csllabelwidth}{#1}\break}
\newcommand{\CSLIndent}[1]{\hspace{\cslhangindent}#1}
\ifLuaTeX
\usepackage[bidi=basic]{babel}
\else
\usepackage[bidi=default]{babel}
\fi
\babelprovide[main,import]{british}
% get rid of language-specific shorthands (see #6817):
\let\LanguageShortHands\languageshorthands
\def\languageshorthands#1{}
\usepackage{booktabs}
\usepackage{longtable}
\usepackage{array}
\usepackage{multirow}
\usepackage{wrapfig}
\usepackage{float}
\usepackage{colortbl}
\usepackage{pdflscape}
\usepackage{tabu}
\usepackage{threeparttable}
\usepackage{threeparttablex}
\usepackage[normalem]{ulem}
\usepackage{makecell}
\usepackage{xcolor}
\ifLuaTeX
  \usepackage{selnolig}  % disable illegal ligatures
\fi
\IfFileExists{bookmark.sty}{\usepackage{bookmark}}{\usepackage{hyperref}}
\IfFileExists{xurl.sty}{\usepackage{xurl}}{} % add URL line breaks if available
\urlstyle{same} % disable monospaced font for URLs
\hypersetup{
  pdftitle={An Independent European Macroeconomics? A History of European Macroeconomics through the Lens of the European Economic Review},
  pdfauthor={Aurélien Goutsmedt; Alexandre Truc },
  pdflang={en-UK},
  hidelinks,
  pdfcreator={LaTeX via pandoc}}

\title{An Independent European Macroeconomics? A History of European
Macroeconomics through the Lens of the European Economic Review}
\author{Aurélien Goutsmedt\footnote{UCLouvain, ISPOLE; F.R.S.-FNRS.} \and Alexandre
Truc \footnote{Université Côte d'Azur, CNRS GREDEG France.}}
\date{2023-04-21}

\begin{document}
\maketitle
\begin{abstract}
Economics in Europe has encountered a process of internationalisation
since the 1970s. To a certain extent, this internationalisation is also
an `Americanisation' and many European departments and economics have
adopted the standards of US economics, notably mathematical modelling,
the use of econometrics, and the neoclassical theory as a modelling
benchmark. Regarding this process, we can wonder if European economics
has just been mimicking US economics since the 1970s, or if some
European specialities have survived or emerged.

In this article, we use topic modelling and bibliometric coupling to
identify what have been some European specialities between 1969 and
2002. We focus on macroeconomics, and we use the articles published in
the European Economic Review and compare their bibliographic references
and textual content (\emph{via} titles and abstracts) to what has been
published in the top 5 journals.

In the late 1970s and early 1980s, disequilibrium theory constituted a
significant part of the research undertaken by European macroeconomists,
and did not limit to the GET, but also represented a unifying framework
to deal with different macroeconomic issues. After it lose its influence
in the second part of the 1980s, political economy occupied this role.
It constituted a resource for tackling the issues raised by the European
integration and the building of a European monetary system, and
constituted a common language for many European macroeconomists.
\end{abstract}

\setstretch{1.5}
\hypertarget{introduction}{%
\section{Introduction}\label{introduction}}

In 1987 in the \emph{European Economic Review}, the director of the
Centre for Economic Policy Research, Richard Portes, attempted to assess
the ``state and status of economics in Europe''. He regarded ``the
standard of comparison {[}as{]} obvious: the United States, by far the
dominant producer'' (\protect\hyperlink{ref-portes1987}{Portes, 1987, p.
1329}). He then asked ``whether there is now any economics outside and
independent of the United States.'' (p.~1330) He listed many clues
testifying of the US domination, ending it by the observation that ``the
leaders of the economics profession in Europe were trained as
postgraduates in the United States. Many take from the US their
professional standards, their views of what are the interesting
problems, and their approaches to them'' (\emph{ibid.}).

Indeed, in the early 1970s in many Western European countries, economics
had entered a process of internationalisation
(\protect\hyperlink{ref-fourcade2009}{Fourcade, 2009, chap. 3} and 4;
\protect\hyperlink{ref-fourcade2006}{Fourcade, 2006}).\footnote{Of
  course, the circulation of economic ideas has always been relatively
  internationalised and examples of knowledge and economists circulation
  before the 1970s abound (e.g.
  \protect\hyperlink{ref-hagemann2011a}{Hagemann, 2011};
  \protect\hyperlink{ref-hesse2012}{Hesse, 2012}). However, one observe
  a significant acceleration of this process after the 1970s
  (\protect\hyperlink{ref-coats1996}{Coats, 1996}), even though national
  rhythms may differ (e.g.
  \protect\hyperlink{ref-backhouse1997a}{Backhouse, 1997}).} To some
extent, such process was also a form of ``Americanisation''
(\protect\hyperlink{ref-coats1996}{Coats, 1996};
\protect\hyperlink{ref-goutsmedt2021}{Goutsmedt et al., 2021}): US
professional and intellectual standards were progressively adopted in
European countries, mimicking the functioning of the US academic field.
English gradually spread as the dominant language in economics
(\protect\hyperlink{ref-sandelin1997}{Sandelin and Ranki, 1997}) and
publications in peer-review journals became the norm for assessing
research productivity. The organisation of international events were
encouraged to boost research centres visibility
(\protect\hyperlink{ref-cherrier2021}{Cherrier and Saïdi, 2021};
\protect\hyperlink{ref-goutsmedt2021}{Goutsmedt et al., 2021}). In terms
of content, the Americanisation of the discipline in Europe favoured the
intellectual standards that had become widespread in the US in the
postwar era (\protect\hyperlink{ref-morgan1998}{Morgan and Rutherford,
1998}): the use of mathematical economics and econometrics, and the
reliance on neoclassical theory as a benchmark for modelling.\footnote{Of
  course, this process of Americanisation did not go without conflicts:
  many ``local conflicts'' emerged between more ``nationally-trained''
  economists (generally locally trained) and ``internationally-trained
  economists'' who had been often trained in the US
  (\protect\hyperlink{ref-fourcade2006}{Fourcade, 2006}). These
  conflicts involved intellectual matters (for instance around the
  relevance of the neoclassical theory) as well as institutional issues,
  like the criteria to assess the quality of economists' work and thus
  to determine hiring and promotion.}

In parallel to this Americanisation, we observe a process of
`Europeanisation': many initiatives from the first issue of the
\emph{European Economic Review} (EER) in 1969 to the creation of the
\emph{European Economic Association} (EEA) in 1984 promoted the
development of intellectual exchanges between European
economists---while obviously keeping US economics as a model. The
simultaneous spreading of US standards in Europe after the 1970s and the
promotion of a European economics transcending national traditions bring
us back to Portes's 1987 question: could a distinct European approach to
economics develop and maintain a degree of autonomy from the US in the
years following the 1970s?

Portes pointed out some European ``comparative advantages''
(\protect\hyperlink{ref-portes1987}{Portes, 1987, p. 1332}), even though
some of these European specialities had been initially pioneered by US
economists. He mentioned the prominence in Europe of ``general
equilibrium theory,'' ``international macroeconomic policy
coordination,'' or ``Non-Walrasian macroeconomics'' (\emph{ibid}.).
Goutsmedt et al. (\protect\hyperlink{ref-goutsmedt2021}{2021}) have also
highlighted that within the \emph{International Seminar on
Macroeconomics} (ISoM), whose annual proceedings were published in the
EER, disequilibrium macroeconomics and large-scale macroeconometric
modelling constituted important rallying research programs until the
mid-1980s for European economists involved in the ISoM.\footnote{In the
  rest of the article, we follow Backhouse and Boianovsky and use the
  expression ``disequilibrium macroeconomics'' to designate a research
  program that has been labelled in many ways: ``non-Walrasian theory,
  disequilibrium theory, equilibrium with rationing,
  \emph{non-tâtonnement} theory, fixed-price models''
  (\protect\hyperlink{ref-backhouseboianovski2013}{Backhouse and
  Boianovski, 2013, pp. 8--9}).}

The purpose of our article is to investigate systematically and
quantitatively the development and persistence of European specificities
in macroeconomics. Macroeconomics constituted a substantial part of
EER's publications, even representing almost half of all the articles in
the early 1980s (Figure \ref{fig:plot-jel}). Macroeconomics was also
instrumental in fostering collaborations between European economists as
evidenced by the ISoM (see Section \ref{rising-journal}). Regarding
EER's history and its significance in the promotion of a European
macroeconomics, we believe that EER's publications constitutes a
compelling perspective from which to examine the evolution of a European
``mainstream'' macroeconomics, as well as of the emergence and
persistence of `European specialities' within the field (see Section
\ref{EER-creation} for further insights on this point).\footnote{We use
  ``mainstream'' as a convenient way to indicate that our focus is on a
  specific subset of macroeconomics, which adhered to certain standards
  and fundamental theoretical assumptions, inspired by US economists.
  Several alternative approaches, such as Marxian and Sraffian economics
  or British Keynesianism and post-Keynesianism, held strong roots in
  Europe. Yet, they were considerably less likely to be featured in
  publications within the EER.} We define European specialities as
\emph{(i)} prevalent research themes (i.e.~representing a substantial
portion of European macroeconomists' research) \emph{(ii)} distinct from
what US-based economists were doing and \emph{(iii)} embraced by
Europe-based economists affiliated with a diverse array of institutions
across different European countries. This final criterion resonates with
the idea of a ``Europeanisation'': in the subsequent analysis, we take
care to differentiate between research areas predominantly established
in a single European country and those encompassing multiple countries,
fostering collaborations among macroeconomists across Western
Europe.\footnote{Up until 2002, economists from Eastern Europe were
  scarcely represented. For the sake of simplicity, we will use the term
  Europe.} Employing a combination of bibliometric coupling, topic
modelling and content reading, we pinpoint European specialities in the
period spanning 1973 to 2002.\footnote{The corpus we use (see section
  \ref{methods}) has very few abstracts between 1969 (the date of the
  creation of the EER) and 1972. Besides, there is no JEL code for EER
  articles before 1973, preventing us for identifying macroeconomics
  articles. After 2002 and the creation of the
  \href{https://academic.oup.com/jeea}{\emph{Journal of the European
  Economic Association}}, the EER was not the official journal of the
  EEA any more.}

The interplay between the internationalisation of macroeconomics and the
persistence of specialities presents a compelling avenue to contribute
to the foundation of a history of European macroeconomics, an area that
remains largely unexplored. Over the last decade, many historical
contributions have documented the evolution of macroeconomics in the
1970s and 1980s. These contributions have identified the major
trajectories of the discipline's transformation (especially the changes
brought about by new classical economists' contributions) and examined
the extent to which macroeconomics' methodology has evolved
(\protect\hyperlink{ref-devroey2016}{De Vroey, 2016};
\protect\hyperlink{ref-duartelima2012a}{Duarte and Lima, 2012}).
Historians have also underlined the discontinuities within these
transformations, as well as the resistance against them
(\protect\hyperlink{ref-goutsmedt2021b}{Goutsmedt, 2021};
\protect\hyperlink{ref-goutsmedtetal2019}{Goutsmedt et al., 2019};
\protect\hyperlink{ref-renault2020a}{Renault, 2020}), their varying
impact on applied and empirical works
(\protect\hyperlink{ref-boumans2019}{Boumans and Duarte, 2019};
\protect\hyperlink{ref-qin2013a}{Qin, 2013};
\protect\hyperlink{ref-renault2022}{Renault, 2022}), but also the
existence of alternative theoretical research programmes
(\protect\hyperlink{ref-backhouseboianovski2013}{Backhouse and
Boianovski, 2013}; \protect\hyperlink{ref-cherrier2018c}{Cherrier and
Saïdi, 2018}; \protect\hyperlink{ref-hoover2012}{Hoover, 2012}).
Nevertheless, these historical contributions remained generally
US-centred. This can be readily attributed to the predominant influence
of US macroeconomists on the discipline, bolstered by the
internationalisation process previously discussed. Yet, it remains
essential to comprehend how European macroeconomics may have diverged
from the dominant US macroeconomics, how it has evolved at a distinct
pace, pursued alternative trajectories, and focussed on differing
issues. Furthermore, our article broadens the scope of historical
investigation to include the 1990s, a period that has yet not been
thoroughly explored by historians.

Our approach identifies different \emph{bibliometric clusters} and
\emph{topics} that are more associated to publication in the EER (rather
than in top 5 journals) and to Europe-based economists (see Section
\ref{methods} for details on method).\footnote{The article is also
  accompanied by a detailed methodological
  \protect\hyperlink{appendix}{Appendix}, as well as two online
  appendices listing the features of the bibliometric clusters
  (``Bibliographic information about the EER and details on the
  bibliographic coupling clusters'') and of the topics identified by our
  topic model (``Details on the topics'').} This approach provides
insights into the research areas European macroeconomists focused on
from the 1970s to the 1990s, in contrast to their US counterparts. A
meticulous examination of the detailed findings allows us to paint a
broader, albeit not complete, portrait of the evolution of European
macroeconomics since the 1970s. Consistently with Portes
(\protect\hyperlink{ref-portes1987}{1987}) and Goutsmedt et al.
(\protect\hyperlink{ref-goutsmedt2021}{2021}) claims, disequilibrium
theory appears as a unifying framework for European macroeconomics
between the mid-1970s and the mid-1980s. If it first constituted a
theoretical research program targeting the development of general
equilibrium theory (GET), it also became an interpretative framework to
explain the stagflation and the European unemployment problem after the
1970s (see Section \ref{disequilibrium}). Disequilibrium theory---and
notably Malinvaud's (\protect\hyperlink{ref-malinvaud1977}{1977})---was
a rallying point for European macroeconomists when discussing various
macroeconomic issues, and even those who disagreed with its utility made
their dissent explicit. Its influence was thus far more extended than
the contributions of new classical economists like Robert Lucas, Thomas
Sargent or Robert Barro. However, disequilibrium progressively
disappeared from references in the second part of the 1980s and issues
like European unemployment were tackled mainly using other types of
frameworks. If no unifying and consistent theoretical framework has
taken over the disequilibrium theory (at least to the same extent), the
new political economy inspired by Kydland and Prescott
(\protect\hyperlink{ref-kydland1977}{1977}) and Barro and Gordon
(\protect\hyperlink{ref-barro1983}{1983a},
\protect\hyperlink{ref-barro1983c}{1983b}) brought new questions and a
common language for many contributions of European economists (see
Section \ref{political-economics}). The pioneering contributions of this
literature were carried by US economists, but it became a truly European
way to tackle many macroeconomic issues in the 1990s.

\begin{figure}[h]

{\centering \includegraphics[width=0.8\linewidth]{../../../../../../../../Mon Drive/data/EER/pictures/Graphs/mean_jel} 

}

\caption{Share of articles with at least one macroeconomics JEL code}\label{fig:plot-jel}
\end{figure}

\hypertarget{EER-creation}{%
\section{The creation of the EER}\label{EER-creation}}

\hypertarget{the-birth-of-a-european-project}{%
\subsection{The birth of a european
project}\label{the-birth-of-a-european-project}}

In 1969, Jean Waelbroeck and Herbert Glejser, both from the
\emph{Université Libre de Bruxelles} (ULB), launched the \emph{European
Economic Review}. The new review was planned to be the official journal
of the European Scientific Association of Applied Economics (ASEPELT),
which had been created in 1961 by Waelbroeck and another ULB economist:
Etienne Kirschen. Before 1969, the association published in English a
bulletin gathering research in econometrics and mathematical economics
(\protect\hyperlink{ref-waelbroeck1969}{Waelbroeck and Glejser, 1969, p.
4}). The EER took up this torch by publishing the same type of research.
Articles had to be published in English, the new ``\emph{lingua franca}
of economics'' triggering the process of ``internationalisation of our
science'' as Waelbroeck and Glejser polemically stated in the
introduction of the first issue (\emph{ibid.}).

The fact that such a project was born in Belgium is no coincidence.
Indeed, the country displayed a high effervescence regarding the
internationalisation of the discipline. In 1966, Jacques Drèze had
established the Center for Operations Research and Econometrics (CORE)
at the \emph{Katholieke Universiteit Leuven} (before its split), on the
model of the Cowles Commission and the Carnegie Institute of Technology,
which Drèze had visited in the 1950s
(\protect\hyperlink{ref-duppe2017}{Düppe, 2017}).\footnote{KU Leuven was
  split in 1968 between a Flemish and a French-speaking part, the latter
  giving birth to the \emph{Université Catholique de Louvain} at
  Louvain-La-Neuve, where the CORE eventually moved in the mid-1970s.}
The CORE developed a research program around macroeconomic modelling and
GET, and quickly stimulated the establishment of a European research
network of economists, notably through its large visiting programme
(\protect\hyperlink{ref-duppe2017}{Düppe, 2017};
\protect\hyperlink{ref-maes2005}{Maes and Buyst, 2005}). Encouraged by
Waelbroeck, the ULB department of economics joined the CORE in its first
years of existence (\protect\hyperlink{ref-maes2005}{Maes and Buyst,
2005, p. 79}).

From the beginning, the EER was conceived as a European project and the
composition of the editorial board testifies of it (Figure
\ref{fig:plot-boards}). But the EER being a Belgian-centred initiative,
Belgian institutions represented one fourth of authors' affiliations in
EER articles in the first years (Figure
\ref{fig:plot-authors}).\footnote{This is an approximation, as the
  affiliation per author is not available in our corpus and we only have
  the affiliations per article (see
  \protect\hyperlink{author-affiliation}{Appendix B.2.} for more
  details).} Nonetheless, the EER authorship became increasingly diverse
in the 1970s in terms of geographic affiliation.

\begin{figure}[h]

{\centering \includegraphics[width=1\linewidth]{../../../../../../../../Mon Drive/data/EER/pictures/Graphs/Board_affiliations} 

}

\caption{Share of countries in EER editorial boards (Top 10)}\label{fig:plot-boards}
\end{figure}

\begin{figure}[h]

{\centering \includegraphics[width=1\linewidth]{../../../../../../../../Mon Drive/data/EER/pictures/Graphs/EER_affiliations} 

}

\caption{Share of countries of authors' affiliations in EER publications (Top 10)}\label{fig:plot-authors}
\end{figure}

The EER was one of these crucial initiatives that contributed to the
development of intellectual exchanges between European based economists
(\protect\hyperlink{ref-goutsmedt2021}{Goutsmedt et al., 2021}). The
centrality of the journal was strengthened in 1984 when the European
Economic Association was created, and the EER was established as the
official journal of the new association.

\hypertarget{rising-journal}{%
\subsection{A rising european journal}\label{rising-journal}}

Besides offering a common platform for European economists, the journal
initial goal was also to encourage the promotion of a US-style approach
to economics. An important dimension of the journal was thus the
progressive integration of US-based economists. The ``International
Seminar on Macroeconomics,'' (ISoM) co-organized by the French
\emph{Ecole des Hautes Etudes en Sciences Sociales} and the US National
Bureau of Economic Research, played a key role in that integration of US
economists, as the conference papers were published each year in a
special issue (\protect\hyperlink{ref-goutsmedt2021}{Goutsmedt et al.,
2021}). The ISoM therefore also contributed to the journal's major focus
on macroeconomics in the 1980s (Figure \ref{fig:plot-jel}).\footnote{The
  year 1979 in the data behind Figure \ref{fig:plot-jel} resulting from
  the confluence of two factors: fewer articles were published this year
  with only one volume against two in preceding and subsequent years;
  JEL codes were missing for most EER article this year.}

It also likely contributed to make the journal known on the other side
of the Atlantic. The share of US-based authors publishing in the journal
grew steadily in the 1970s and reached a third of all affiliations in
the early 1980s (Figure \ref{fig:plot-authors}). The increase of US
economists participation to the EER did not solely mean that more
articles were published by US authors, but also that the number of
collaborations between US- and Europe-based economists increased (Figure
\ref{fig:plot-collabs}). While there was no collaboration in the first
year of the journal, 10 percent of the articles published in 1980 mixed
institutions from the US and Europe.

\begin{figure}[h]

{\centering \includegraphics[width=1\linewidth]{../../../../../../../../Mon Drive/data/EER/pictures/Graphs/Collab_strict} 

}

\caption{Patterns of collaboration between the United States and European countries in EER}\label{fig:plot-collabs}
\end{figure}

In the mid-1980s, the journal thus emerged as a symbol of a more
integrated European economics, taking inspiration from the US standards
and enticing numerous US economists to contribute. Also, its
intellectual impact has seemingly broadened: the journal ascended as a
pre-eminent publication in macroeconomics, gradually surpassing other
prominent European journals regarding bibliographic citations (Figure
\ref{fig:plot-eer-importance-macro}).

\begin{figure}[h]

{\centering \includegraphics[width=0.9\linewidth]{../../../../../../../../Mon Drive/data/EER/pictures/Graphs/EER_importance_macro_bw} 

}

\caption{Share of total citations from macroeconomics articles going to EER}\label{fig:plot-eer-importance-macro}
\end{figure}

But did this whole process of internationalisation lead to the total
standardisation of a European macroeconomics on the US model, or did it
still leave room for the development and persistence of proper European
specialities?

\hypertarget{methods}{%
\section{Methods for identifying European specialities}\label{methods}}

To identify European specialities, we compare macroeconomics articles
published in the EER and in the Top-5 journals (\emph{American Economic
Review}, \emph{Journal of Political Economy}, \emph{Econometrica},
\emph{Quarterly Journal of Economics}, \emph{Review of Economic
Studies}). By focusing on the Top 5, we get only the most popular and
dominant trends in macroeconomics and we thus draw clearer comparisons
with what is published in the EER. Besides, the EER was created with the
intent to establish an elite leading journal for the European community
that would imitate the standards of US major journals. The Top 5 thus
seems an adequate benchmark to compare the EER to.

We identify macroeconomics articles by using the former and new JEL
codes classifications (\protect\hyperlink{ref-jel1991}{JEL,
1991}).\footnote{See the complete list of all the JEL codes we have used
  in \protect\hyperlink{eer-top5-macro}{Appendix B.1.}} Outside of JEL
codes data, we have used three different databases to collect different
types of information: outside of basic metadata (year of publication,
title, authors, \emph{etc.}), we have collected the list of
bibliographic references of the EER and Top 5 articles, the abstracts,
and authors affiliations.\footnote{Crossing databases has been necessary
  due to missing years and information in the different databases we
  have used (Web of Science, Scopus and Microsoft Academic Premier). See
  \protect\hyperlink{corpus}{Appendix B.1.} for more details on the
  building of our dataset.} Then, we have conducted two different types
of analysis to identify European specialities.

\hypertarget{bibliographic-coupling}{%
\subsection{Bibliographic coupling}\label{bibliographic-coupling}}

Bibliographic coupling connects articles together depending on the
bibliographic references they share. We build different relational
networks using EER and Top-5 articles (the nodes of the network),
connected together by weighted links (the edges of the network),
depending on the number of references two articles share
together.\footnote{For more details on the measure of weights, see
  \protect\hyperlink{network}{Appendix B.3.}.} We build networks on a
moving eight-year window (depending on the year of publication of the
articles). We thus have 23 networks from the 1973-1980 period, through
1974-1981, 1975-1982, \emph{etc.}, to the 1995-2002 period. For each
network, we use the Leiden algorithm
(\protect\hyperlink{ref-traag2019}{Traag et al., 2019}) to identify
bibliographic clusters, that is groups of articles that share many
significant references in common with articles in their cluster, and few
with articles outside their cluster. Articles which belongs to the same
cluster are more likely to share cognitive content (e.g., sharing
objects of study, methods, results or theory) even if disagreeing
(\protect\hyperlink{ref-claveau2016}{Claveau and Gingras, 2016};
\protect\hyperlink{ref-goutsmedt2021}{Goutsmedt et al., 2021};
\protect\hyperlink{ref-truc2021}{Truc et al., 2021}). Finally, we look
at the similarity of the clusters two by two for successive time
windows, and merge clusters from different windows together when they
are sufficiently close.\footnote{See
  \protect\hyperlink{network}{Appendix B.3.} for details on the merging
  criteria.}

This process allows us to obtain dynamic clusters. Indeed, citation
patterns are highly dependent of the date of publication of an article:
scholars tend to cite more recent works. Consequently, for large time
windows, clusters would likely be determined mainly by the publication
year, rather than by what they are talking about.\footnote{In other
  words, articles would be grouped together depending on the year of
  their publication and the clusterisation of the network would not say
  much of the economic content articles grouped together would share.}
By taking small time windows and then by merging clusters in different
windows together, we avoid this problem and are able to identify
clusters over longer period of time. We identify a total of 154 clusters
but only 33 that are \emph{(i)} present in at least 2 networks (i.e.~2
time windows) and \emph{(ii)} represent more than 0.04 percent of the
nodes of at least one of the network they belong.

A set of indicators allows us to understand what these clusters are
about---e.g.~the words used in abstracts and titles, the recurrent
authors, the most important nodes or the most cited
references.\footnote{There information are collected for each cluster in
  the online appendix ``Bibliographic information about the EER and
  details on the bibliographic coupling clusters''.} These indicators
guided us when naming the clusters. Then, for each cluster, we identify
the US or European oriented nature of its publications and authors. We
measured \emph{via} a log ratio the over/under representation of Europe-
and US-based authors in the cluster, and over/under representation of
the EER and top 5 journals in the cluster.\footnote{Our assumption is
  that the content of articles published in the Top 5 by European
  economists could be more largely influenced by the standards of Top 5
  journals and of US macroeconomics, and thus could be less
  representative of European economics than the articles published in
  the EER.} These two measures inform us on which are the most
`European' clusters, meaning those where relatively more articles are
published in the EER and by Europe-based economists.\footnote{Table
  \ref{tab:summary-communities} lists the 33 most significant clusters
  with their degree of `Europeanism' and
  \protect\hyperlink{network}{Appendix B.3.} explains the method. Figure
  \ref{fig:plot-cluster-flow} displays the distribution of clusters over
  the different time windows and the flows of articles between the
  different clusters from one time window to another.} Figure
\ref{fig:plot-community-diff} displays the position of each cluster
relatively to these two measures with the sum of the two being a
synthetic indicator of how much a cluster is European.

\begin{figure}[h]

{\centering \includegraphics[width=1\linewidth]{../../../../../../../../Mon Drive/data/EER/pictures/Graphs/Communities_europeanisation_colored} 

}

\caption{The most European clusters}\label{fig:plot-community-diff}
\end{figure}

\hypertarget{topic-modelling}{%
\subsection{Topic modelling}\label{topic-modelling}}

Topic modelling is a non-supervised machine learning method which
associates \emph{(i)} the \emph{ngrams} contained in a corpus to
\emph{k} topics and \emph{(ii)} the documents of the corpus to the same
\emph{k} topics. Our corpus is constituted of all the titles and
abstracts (when available) of the macroeconomic articles published in
the EER and Top 5.\footnote{From the documents of our corpus, we extract
  (or `tokenise') unique words (or unigrams), bigrams and trigrams. Stop
  words and other uninformative words are excluded and all words are
  `lemmatised'. See the \protect\hyperlink{topic}{Appendix B.4.} for
  more details on the preprocessing steps we use.} We use a variant of
the Latent Dirichlet Allocation model with the Correlated Topic Model
(\protect\hyperlink{ref-blei2007}{Blei and Lafferty, 2007}). The number
of topics \emph{k} is chosen by the modellers: after assessing
quantitatively and qualitatively different models, we choose to run the
model with 50 topics.\footnote{\protect\hyperlink{topic}{Appendix B.4.}
  gives more details on the different models we have tested and how we
  have set the number of topics.} For each topic, we can look at the
word with the highest `FREX' value
(\protect\hyperlink{ref-bischof2012}{Bischof and Airoldi,
2012}).\footnote{FREX is the weighted harmonic mean of the terms' rank
  regarding exclusivity and frequency scores. Exclusivity is a measure
  of how much a term is frequent in a topic in comparison to its
  frequency in others. In other words, a good topic model is a model
  where the words in topics are frequently used, but each topic can be
  easily dinstinguished from others, for the words associated to this
  topic are hardly used or not used in other topics.} Table
\ref{tab:summary-topics} displays the words with the highest FREX value
for each topic. We also use a set of indicators, like the most cited
references per topic crossed with the journal and affiliation variables,
to get a better picture of what the topics are and what are their
European and non-European dimensions.\footnote{There information are
  collected in the online Appendix ``Details on the topics''.}

Similarly to what we do for bibliometric coupling, we are interested in
the topics characteristics regarding the publications (EER \emph{vs.}
Top 5) and the countries of affiliations of the authors (the US
\emph{vs.} European countries). As each article has a `rate of
belonging' to each topic (the \emph{gamma} value), we keep only the
articles with a \emph{gamma} value above 0.1 to assess the over/under
representation of EER and of Europe-based authors in each
topic.\footnote{See \protect\hyperlink{topic}{Appendix B.4.} for details
  on the measure we use and for the results with an alternative measure
  that do not use \emph{gamma} as a threshold.}. The measures are the
coordinates of the 50 topics in Figure \ref{fig:plot-topic-diff}. When
we sum, we have an indicator of how much a topic is a European topic.

\begin{figure}[h]

{\centering \includegraphics[width=1\linewidth]{../../../../../../../../Mon Drive/data/EER/pictures/Graphs/logratio_diff_plot_new_bw} 

}

\caption{The most European topics (LogRatio method)}\label{fig:plot-topic-diff}
\end{figure}

\hypertarget{why-mixing-the-two-methods}{%
\subsection{Why mixing the two
methods?}\label{why-mixing-the-two-methods}}

To identify the main characteristics of a corpus, the existing
literature usually uses a topology-based approach (bibliographic
information and network analysis) or a topic-based approach (textual
information and topic modelling). In the first case, articles that share
the same bibliographic information (e.g., same references) are
considered similar and belong to the same speciality
(\protect\hyperlink{ref-claveau2016}{Claveau and Gingras, 2016}). In the
second case, using titles, abstracts or full-text, topics are assigned
to individual articles, and specialities are derived from these topics
(\protect\hyperlink{ref-ambrosino2018}{Ambrosino et al., 2018}). Both
methods have the advantage of organising a large corpus and allow the
user to `read' more than you can really read. It points to the most
influent articles and each cluster or topic can be identified by some
important articles. Besides furnishing synthetic indicators (like the
most cited references or most recurrent authors), these methods guide
the qualitative exploration of a corpus.

Within the infometrics literature, one can find some attempts to combine
both approaches (\protect\hyperlink{ref-dingCommunity2011}{Ding, 2011};
\protect\hyperlink{ref-liAdding2012}{Li et al., 2012};
\protect\hyperlink{ref-maoTopic2017}{Mao et al., 2017};
\protect\hyperlink{ref-yanTopics2012}{Yan et al., 2012}). To our
knowledge, this article is the first attempt to combine both methods to
describe the state of economics at different period. The combination of
both approaches provides different advantages. First, each method
enables to cross-check the result of the other and thus to test the
general robustness of the results. If a certain literature appears as
not `European' or very `European' in both the bibliometric and topic
modelling analyses, it gives us confidence in the robustness of this
result.\footnote{This is also strengthened by the fact that bibliometric
  data as text data can display some errors or be missing at some
  points. Crossing the two types of data reduces the role played by
  errors and missing values.}

Second, the combination of the two methods makes it possible to take
advantage of their complementarity, since they do not measure the same
thing. Clusters tend to reflect a more sociological dimension: we cite
economists who work on similar subjects but also who are
institutionally, geographically and intellectually closer to us. Topic
modelling partially abstracts from this dimension by looking at words
used. Several times in this article, topic modelling allows us to
complement bibliometric results by observing how some large bibliometric
cluster actually dealt with several topics, but also for a same topic
(and thus a relatively similar used vocabulary) what were the
differences between Europe-based and US-based economists.

\hypertarget{european-specialities}{%
\section{A broad picture of European
specialities}\label{european-specialities}}

In this section, we get a general idea of the different specialities
emerging from bibliometric coupling and topic modelling analyses. In the
two next sessions, we will sketch a more encompassing portrait of the
evolution of European macroeconomics from the late 1970s to the late
1990s, while leaving aside some of the specialities identified.

First, the two methods allow us to understand what European
macroeconomics \emph{was not}. A first consistent finding between the
two methods is that the literature about the life-cycle and permanent
income hypotheses, influenced by Friedman
(\protect\hyperlink{ref-friedman1957}{1957}) and Hall
(\protect\hyperlink{ref-hall1978b}{1978}), was far from popular for
European economists.\footnote{See clusters ``Intergenerational model,
  Savings \& Consumption'' and ``Permanent Income and Life-Cycle
  Hypotheses'', as well as topics 12 and 14.} Part of this literature
sought to introduce structural heterogenity in life cycle or permanent
income models (see Cherrier, Duarte and Saïdi, this issue). Other more
US-oriented areas were research about \emph{(i)} the demand for
money---for which Baumol (\protect\hyperlink{ref-baumol1952}{1952}) and
Friedman and Schwartz (\protect\hyperlink{ref-friedman1963}{1963}) were
central references---as well as \emph{(ii)} the ``new classical monetary
theory'' (\protect\hyperlink{ref-hoover1988}{Hoover, 1988, chap. 6}) of
the 1970s inspired by Sargent's, Bryant's and Wallace's works---see for
instance Bryant and Wallace (\protect\hyperlink{ref-bryant1979}{1979})
or Sargent and Wallace (\protect\hyperlink{ref-sargent1982}{1982})---or
\emph{(iii)} the more recent ``New Monetarist Economics'' of Kiyotaki
and Wright (\protect\hyperlink{ref-kiyotaki1989}{1989}), Kiyotaki and
Wright (\protect\hyperlink{ref-kiyotaki1993}{1993}), and Trejos and
Wright (\protect\hyperlink{ref-trejos1995}{1995})---see Frasser
(\protect\hyperlink{ref-frasser2020}{2020, chap. 2}), for a historical
reconstruction of this literature.\footnote{See clusters ``Monetary
  Economics \& Demand for Money'' and ``New Theory of Money: Search,
  Bargaining\ldots{}'', and, even if it is not as ``non-European'' as
  the two others, the cluster ``Demand for Money''. For topics, see
  topic 2 on the demand for money and money supply, which is one of the
  most non-European topic, but also topic 19 on demand for money and
  term structure of interest rates, influenced notably by Fama
  (\protect\hyperlink{ref-fama1975}{1975}).} The new classical monetary
theory of the 1970s is described by Hoover
(\protect\hyperlink{ref-hoover1988}{1988, p. 111}) as the research for
``microfoundations for the theory of money consistent with general
equilibrium and individual optimization'' promoted by new classical
economists (Lucas, Sargent, Barro, Kydland, Prescott, etc.). More
generally, it appears that the works of new classical economists that
contributed to reshaping macroeconomics in the late 1970s and early
1980s, and that are so central in many history of macroeconomics
(\protect\hyperlink{ref-devroey2016}{De Vroey, 2016}), were less
influential in Europe at the time. Articles like Lucas
(\protect\hyperlink{ref-lucas1972}{1972}), Lucas
(\protect\hyperlink{ref-lucas1973}{1973}), Sargent and Wallace
(\protect\hyperlink{ref-sargent1975}{1975}) or Barro
(\protect\hyperlink{ref-barro1976}{1976}) were constantly under-cited by
Europe-based macroeconomists in comparison to US economists in the 1970s
and 1980s (see Figure \ref{fig:plot-new-classical}).\footnote{We have to
  wait the 1982-1988 window to see some new classical contributions
  cited as much by Europeans as by US economists. The integration of
  these contributions obviously took some time in Europe and lagged
  behind the US.} This is consistent with the fact that European
macroeconomists favoured in the late 1970s and early 1980s an
alternative ``microfoundational programme''
(\protect\hyperlink{ref-hoover2012}{Hoover, 2012}): disequilibrium
theory (see Section \ref{disequilibrium}).

\begin{figure}[h]

{\centering \includegraphics[width=1\linewidth]{../../../../../../../../Mon Drive/data/EER/pictures/Graphs/g2_over_citations_bw} 

}

\caption{Citation of new classical works by European economists relatively to US-based economists (log of ratios on 7-year moving average)}\label{fig:plot-new-classical}
\end{figure}

Second, now regarding what European macroeconomics was, the EER appeared
as a more welcoming support for international macroeconomics. All
clusters and topics dealing with this kind of issues are relatively
over-represented in the EER (except one small topic on gold standard and
dollar reserves), and many topics are also relatively more authored by
European authors (see Figure \ref{fig:plot-community-diff} and
\ref{fig:plot-topic-diff}). Topics and clusters on the political economy
of central banking (see Section \ref{political-economics}) and on
unemployment---relying notably on Pissarides
(\protect\hyperlink{ref-pissarides1990}{1990}), Mortensen and Pissarides
(\protect\hyperlink{ref-mortensen1994}{1994}) and Layard et al.
(\protect\hyperlink{ref-layard1991a}{1991})---are also particularly
Europe-oriented both in terms of authorship and in publication
venues.\footnote{See the ``Political Economics of Central Banks''
  cluster and topic 8 for the first; and cluster ``Theory of
  Unemployment \& Job Dynamics'' and topic 37 for the second.} Lastly,
our analyses reveal a proper European approach of time series and
econometrics even if in a large part a UK speciality: the ``LSE
approach'' of David Hendry and his colleagues
(\protect\hyperlink{ref-qin2013a}{Qin, 2013, chap. 4}) and the treatment
of the cointegration issue.\footnote{See topic 46 and the cluster
  ``Business Cycles, Cointegration \& Trends''.}

The detailed analysis of clusters and topics offer a general panorama of
the different issues, methods and theoretical questions investigated by
European macroeconomists, in comparison to the US macroeconomics.
However, this only gives us a fragmented (and for now rather
a-historical) picture of European macroeconomics. In the two last
sections, we rather draw a more unified picture of the evolution of
European macroeconomics between the mid-1970s and the late 1990s. Even
if leaving aside some identified specialities, we consider that two
dynamics help to understand the interaction between European and US
macroeconomics, how European macroeconomics distinguished itself, as
well as how it has transformed itself since the 1970s. First,
disequilibrium theory represented in the 1970s and 1980s a theoretical
unifying research program for European macroeconomics (see Section
\ref{disequilibrium}). Second, if it did not constitute a similar
theoretical program, it appears to us that, in the 1990s, political
economy played a similar role, constituting a European speciality
touching to many different economic problems (see Section
\ref{political-economics}).

\hypertarget{disequilibrium}{%
\section{Disequilibrium theory as a landmark for European
macroeconomics}\label{disequilibrium}}

Disequilibrium theory constituted an important but often forgotten step
in the history of macroeconomics
(\protect\hyperlink{ref-backhouseboianovski2013}{Backhouse and
Boianovski, 2013}; \protect\hyperlink{ref-plassard2021}{Plassard et al.,
2021}). It contributed significantly to the renewal of interest for the
research for microfoundations in macroeconomics in the 1970s.\footnote{See
  Duarte and Lima (\protect\hyperlink{ref-duartelima2012a}{2012}) for a
  history of microfoundations in macroeconomics.} Anchored in the
general equilibrium theory (GET) tradition and influenced by the work of
Patinkin, Clower and Leijonhufvud, disequilibrium theory explored the
impact of non-walrasian price-setting (i.e.~without \emph{tâtonnement}),
fix-price and quantity rationing on macroeconomic outcomes. It
constituted an alternative to new classical contributions and the
``representative-agent microfoundational program'' of Lucas and Sargent
(\protect\hyperlink{ref-hoover2012}{Hoover, 2012}; see also
\protect\hyperlink{ref-renault2020a}{Renault, 2020}). Even if Barro and
Grossman's (\protect\hyperlink{ref-barro1971}{1971}) article was
fundamental in the popularisation of disequilibrium macroeconomics, the
research program was more deeply anchored in Europe, notably in France
and Belgium (\protect\hyperlink{ref-goutsmedt2021}{Goutsmedt et al.,
2021}, and Plassard and Renault, this issue).\footnote{Online appendices
  also display some statistics on countries and institutions for each
  cluster and topic.}

Bibliometric analysis shows that the ``Disequilibrium and Keynesian
economics'' cluster constituted the most significant cluster strongly
associated to the EER and developed by Europe-based economists. This
cluster also integrates other ``alternative research lines'' to Lucas's
research program (\protect\hyperlink{ref-devroey2016}{De Vroey, 2016,
chap. 14}), like Azariadis's
(\protect\hyperlink{ref-azariadis1975}{1975}) implicit contract model,
Hart's (\protect\hyperlink{ref-hart1982}{1982}) imperfect competition
model or Diamond's (\protect\hyperlink{ref-diamond1982}{1982}) search
model. It testifies that in the late 1970s and in the 1980s, connections
existed between US and European macroeconomists regarding both the
renewal of theoretical macroeconomics and the search for
microfoundations, and the opposition to new classical macroeconomics. As
a complement, topic modelling analysis allows us to observe how
widespread disequilibrium theory was for European macroeconomics in the
1980s and how it unified the treatment of different macroeconomic
issues.

First of all, part of the literature ``arose out of the internal
problems within general equilibrium theory''
(\protect\hyperlink{ref-backhouseboianovski2013}{Backhouse and
Boianovski, 2013, p. 105}), notably the need to break with
\emph{tâtonnement} and to build GET model with agents setting prices in
the model. Disequilibrium macroeconomics thus contributed to the
persistence of a lively research program around GET issues .\footnote{See
  topic 11.} However, disequilibrium macroeconomics also appeared as an
important framework to deal with the explanation of the contemporaneous
stagflation in the 1970s
(\protect\hyperlink{ref-backhouseboianovski2013}{Backhouse and
Boianovski, 2013, chap. 8}). Malinvaud's \emph{Theory of Unemployment
Reconsidered} (\protect\hyperlink{ref-malinvaud1977}{1977}) was a
decisive step in this direction by conceptualising the opposition
between ``Keynesian unemployment'', caused by excess supply in both
goods and labour markets, and classical unemployment'', triggered by
excess demand for goods, but excess supply on the labour
market---involving that real wages were too high.\footnote{Malinvaud
  (\protect\hyperlink{ref-malinvaud1977}{1977}) proposed a third regime,
  ``repressed inflation'', due to excess demand on both markets.} The
oil shock of 1973 and the simultaneous decrease of productivity
explained the rise of a classical unemployment in the 1970s. The big
issue for adherents to the three-regimes approach thus became to be able
to assess which part of European unemployment stemmed from Keynesian or
classical unemployment.

This framework to think about unemployment and the stagflation has been
used, discussed, or at least mentioned in many important works in
European macroeconomics in the 1980s.\footnote{``Important'' means here
  highly cited by European economists in one or several clusters or
  topics.} Drèze and Modigliani
(\protect\hyperlink{ref-dreze1981}{1981}) discussed in the EER the
``current state of underemployment in Belgium'' (p.~2) by analysing the
trade off between real wages and employment in the context of a small
open economy. Drèze and Modigliani explained that they mixed the
possibility of classical unemployment, inspired by Malinvaud
(\protect\hyperlink{ref-malinvaud1977}{1977}), and Modigliani and
Padoa-Schioppa's argument that, ``in an open economy, external balance
implies a constraining relationship between the levels of real wages and
employment'' (2). Similarly, Malinvaud's framework was linked in the
early 1980s to the debate, recurrent in the EER, about the ``wage gap'',
that is the assessment of whether real wages were too high (meaning a
positive wage gap) or too low. Bruno and Sachs were central characters
in this debate and relied explicitly on Malinvaud's
framework.\footnote{Bruno and Sachs's
  (\protect\hyperlink{ref-brunosachs1985}{1985}) book, \emph{Economics
  of Worldwide Stagflation}, offered a synthesis of their late 1970s and
  early 1980s works, and constituted a highly cited resource for
  European macroeconomists in the 1980s (see also
  \protect\hyperlink{ref-goutsmedt2021}{Goutsmedt et al., 2021, sec.
  3}).}

Outside of unemployment and stagflation, disequilibrium theory was also
extended to other macroeconomic issues. For instance, Avinash Dixit,
when at University of Warwick, extended Clower's
(\protect\hyperlink{ref-clower1965}{1965}) dual decision hypothesis and
Malinvaud's framework to international trade theory
(\protect\hyperlink{ref-dixit1978}{Dixit, 1978, p. 393}). It gave the
basis to Dixit for advocating a ``more satisfactory model of the balance
of trade'' than Frenkel and Johnson's
(\protect\hyperlink{ref-frenkel1976}{1976}) monetary approach which
``assumes instantaneous attainment of Walrasian equilibrium in commodity
and labour markets'' (\protect\hyperlink{ref-dixit1978}{Dixit, 1978, p.
393}). Dixit's model would form the basis for some parts of Dixit and
Norman's book on the \emph{Theory of International Trade}
(\protect\hyperlink{ref-dixit1980}{Dixit and Norman, 1980}), which
constituted an important reference for European economists working on
international trade.\footnote{See topic 39.}

This centrality of disequilibrium theory in European macroeconomics is
also visible through the fact that many macroeconomists had to position
themselves in comparison to disequilibrium theory, and notably to the
Keynesian versus classical unemployment framework. In May 1985 was held
a conference in Sussex about European unemployment, published in
\emph{Economica} the next year.\footnote{On this episode, see Backhouse,
  Forder and Laskaridis, as well as Plassard and Renault, both in this
  issue.} Macroeconomists from different countries presented their
analyses of European or national unemployment rates. While Sneessens and
Drèze estimated a ``two-market macroeconomic rationing (or
disequilibirum) model of the economy
(\protect\hyperlink{ref-sneessens1986}{Sneessens and Drèze, 1986, p.
S97}), Malinvaud (\protect\hyperlink{ref-malinvaud1986}{1986}) proposed
a more descriptive analysis to explain the rise of unemployment in
France, even if he claimed some proximity with Sneessens and Drèze
formalisation in the same issue. Malinvaud discussed some determinants
of''the classical component of unemployment''
(\protect\hyperlink{ref-malinvaud1986}{Malinvaud, 1986, p. S216}), but
also criticised the use of Phillips curve with a non-accelerating
inflation rate of unemployment (NAIRU) to deal with the causes of
unemployment. To the contrary, NAIRU was central in the model proposed
by Layard and Nickell to discuss unemployment in Britain and they
claimed that the ``labour demand function that we use cuts through the
fruitless debate now raging (especially in Europe) as to whether current
unemployment is `classical' or `Keynesian'\,''
(\protect\hyperlink{ref-layard1986}{Layard and Nickell, 1986, p. S121}).

If not totally consensual, disequilibrium theory and the
classical/Keynesian unemployment opposition were unavoidable in the
mid-1980s. However, they progressively lose their influence after that
period. Publications about disequilibrium continued to pop out
occasionally in the EER . But in quantitative terms, we observe a
decrease of disequilibrium importance both through the bibliometric and
topic modelling analyses.\footnote{See the destiny of the disequilibrium
  cluster and of topic 11 in Figure \ref{fig:plot-cluster-flow} and
  Figure \ref{fig:plot-topic-year}.} We can also observe that indirectly
in topic 25 on real wages and employment: while Malinvaud
(\protect\hyperlink{ref-malinvaud1977}{1977}) was an important reference
for the older article of the topic, it disappeared from the bibliography
of the most recent articles. \footnote{See online appendix on
  topic-modelling} Part of the research program on disequilibrium seems
to have persist in the 1990s through its most theoretical part and
developed closer links with the literature on coordination and
sunspots.\footnote{See cluster on ``Coordination \& Sunspots 2'' and
  Figure \ref{fig:plot-cluster-flow}}. This cluster was only slightly
over-represented by European economists, but gathered articles mainly
published in the Top 5, so it does not really constitute a European
speciality.{]} Regarding the European unemployment problem, new ways to
account for it progressively emerged and eclipsed the opposition between
Keynesian and classical unemployment. That is the case of Layard and
Nickell's approach (\protect\hyperlink{ref-grubb1982}{Grubb et al.,
1982}; \protect\hyperlink{ref-layard1991a}{Layard et al., 1991};
\protect\hyperlink{ref-layard1986}{Layard and Nickell, 1986}), relying
on NAIRU and wage-bargaining, and of the Diamond-Mortensen-Pissarides
equilibrium approach (\protect\hyperlink{ref-mortensen1994}{Mortensen
and Pissarides, 1994};
\protect\hyperlink{ref-pissarides1990}{Pissarides, 1990}), relying on
search.\footnote{See cluster ``Theory of Unemployment \& Job Dynamics''
  and topic 37.} The insider and outsider approach of the labour market
also gained some popularity in Europe.\footnote{The approach was
  developed notably by Nils Gottfries, Henrik Horn, Assar Lindbeck (all
  from the University of Stockholm) and Denis Snower from Birkbeck
  College (\protect\hyperlink{ref-gottfries1992}{Gottfries, 1992};
  \protect\hyperlink{ref-gottfries1987}{Gottfries and Horn, 1987};
  \protect\hyperlink{ref-lindbeck1987a}{Lindbeck and Snower, 1987},
  \protect\hyperlink{ref-lindbeck1986}{1986}).} We can observe that when
using the insider-outsider opposition to discuss European unemployment
in 1987, Gottfries and Horn still referred to the Keynesian/classical
opposition and argued in their paper that ``the present unemployment may
originally have arisen for Keynesian reasons, but once unemployment is
created it will change the conditions under which wages are formed, thus
persisting in a classical form''
(\protect\hyperlink{ref-gottfries1987}{Gottfries and Horn, 1987, p. 2}).
Lindbeck and Snower similarly cited Malinvaud
(\protect\hyperlink{ref-malinvaud1977}{1977}) and the ``boundary between
the `Keynesian' and `Classical' regimes''
(\protect\hyperlink{ref-lindbeck1987a}{Lindbeck and Snower, 1987, p.
408}). This reference to Malinvaud's framework disappear in the
following years in similar works (as in
\protect\hyperlink{ref-gottfries1992}{Gottfries, 1992}, for instance).
This reference to the classical versus Keynesian unemployment was also
missing in the literature exploring the role of firing costs and labour
market flexibility in European unemployment, which became popular in the
early 1990s (\protect\hyperlink{ref-bentolila1990}{Bentolila and
Bertola, 1990}; \protect\hyperlink{ref-bentolila1992a}{Bentolila and
Saint-Paul, 1992}; \protect\hyperlink{ref-bertola1990a}{Bertola,
1990}).\footnote{These four different approaches were central in
  significant European clusters and topics of the late 1980s and 1990s.
  See cluster ``Theory of Unemployment \& Job Dynamics'' or topic 37 and
  25.}

In the late 1980s, disequilibrium theory had lost its capacity to build
bridges between European macroeconomists and did not constitute a
unifying theoretical language any more. What tended to unify European
macroeconomists in the 1990s was not any more a theoretical framework
derived from the GET and the search for microfoundations, but rather a
new way to approach many macroeconomic problems through the lens of
political economy.

\hypertarget{political-economics}{%
\section{A new unifying language: political
economy}\label{political-economics}}

In its 2000 handbook, \emph{Political Economy in Macroeconomics}, Allan
Drazen defined the ``new political economy'' that had emerged since the
1970s by ``its use of the formal and technical tools of modern economic
analysis to look at the importance of politics for economics''
(\protect\hyperlink{ref-drazen2002}{Drazen, 2002, p. 4}).\footnote{For a
  history of the emergence of the ``new political economy'' or ``new
  political macroeconomics'' label, see Galvão de~Almeida
  (\protect\hyperlink{ref-galvaodealmeida2021}{2021}).} The main
question for political economy is to understand ``how political
constraints may explain the choice of policies (and thus economic
outcomes) that differ from optimal policies'' (p.~7). In Europe, a
detailed introduction to political economy was proposed by Torsten
Persson and Guido Tabellini
(\protect\hyperlink{ref-persson2002}{2002}).\footnote{Torsten Persson
  obtained his PhD in 1982 in Stockholm at the Institute for
  International Economic Studies under the supervision of Lars Svensson
  and became professor in Stockholm in 1987. Tabellini graduated in
  Torino before to move to UCLA for his PhD. After a first position at
  Stanford, he moved back in Italy in 1990. In their book, they used the
  term ``political economics'' rather than ``political economy'', as the
  latter is too much associated with ``an alternative analytical
  approach, as if the traditional tools of analysis in economics were
  not appropriate to study political phenomena''
  (\protect\hyperlink{ref-persson2002}{Persson and Tabellini, 2002, p.
  2}), idea which was the complete opposite of their point of view.}
They distinguished three traditions to which ``political economics'' can
be ``traced back'' (p.~2): ``the theory of macroeconomic policy''
inspired by Lucas, the public choice tradition of Buchanan, Tullock and
Olson, and the formal analysis in political analysis inspired by Riker.

That is the first tradition that lies at the core of European
macroeconomics in the 1990s. The integration of rational expectations in
the 1970s had raised attention for certain policy problems. The most
emblematic one is the time-consistency problem popularized by Kydland
and Prescott (\protect\hyperlink{ref-kydland1977}{1977}). The idea is
that the optimal policy in time \emph{t} is not the same as in \emph{t +
s} as the policymaker has some interest to mislead economic agents for
their own good. If agents are rational, they will anticipate in advance
the policymaker's incentive and the optimal policy is unattainable. This
work raised the question of the necessity to ``tie the hands'' of
policymakers and it has led to numerous extensions, notably about
central banks, on the concepts of credibility, reputation
(\protect\hyperlink{ref-barro1983}{Barro and Gordon, 1983a},
\protect\hyperlink{ref-barro1983c}{1983b}) or about the choice of
central bankers and of the formalisation of delegation
(\protect\hyperlink{ref-rogoff1985b}{Rogoff, 1985}). This literature had
clear origins in the academic US debates around rational expectations
and the efficiency of macroeconomic policies in the 1970s
(\protect\hyperlink{ref-hoover1988}{Hoover, 1988, pp. 80--86}). But the
articles cited above displayed an unusual citation trajectory: after
capturing a rising share of total citations in the first years after
their publication and then a decreasing share (like for many famous
contributions), they have encountered a rebound and a new wave of
popularity in the 1990s (Figure \ref{fig:plot-political-economy}). This
regain of interest is due to European economists who increasingly cited
these references more than their US colleagues (Figure
\ref{fig:plot-political-economy-europe}).

\begin{figure}[h]

{\centering \includegraphics[width=1\linewidth]{../../../../../../../../Mon Drive/data/EER/pictures/Graphs/g1_citations_bw} 

}

\caption{Share of articles citing political economy literature (5-year moving average)}\label{fig:plot-political-economy}
\end{figure}

\begin{figure}[h]

{\centering \includegraphics[width=1\linewidth]{../../../../../../../../Mon Drive/data/EER/pictures/Graphs/g1_over_citations_bw} 

}

\caption{Citation of political economy articles by European economists relatively to US-based economists (log of ratios on 7-year moving average)}\label{fig:plot-political-economy-europe}
\end{figure}

This interest of European economists for the new political economy
literature of the late 1970s and early 1980s is confirmed by both
bibliometric and topic modelling analyses. We find these references as
most cited references in different clusters and topics and they were
cited by many influential European contributions.\footnote{Again, we
  mean here articles written by European authors which were highly cited
  in one or several topics and clusters.} The cluster ``Political
Economics of Central Banks'' is one of the most European clusters while
being comparable in size to the cluster on ``Disequilibrium \& Keynesian
Macro''. Similarly, the topic 8 on credibility, optimal policy and
policy rule clearly represents a highly European topic. But the topic
modelling allows us to observe how the new political economy literature
infused many subjects in the 1990s.\footnote{We are able to observe that
  notably by looking for different topics what are the most cited
  references if the articles are written by European and if they are
  not. In many topics, the difference about the references cited is
  explained by the fact that Europeans refer more to political economy
  contributions.}

We can distinguish three areas where a political economy framework is
recurrently used by European macroeconomists in the 1990s and
constitutes a particularity of European macroeconomics in comparison to
the US. First, many discussions about the appropriate framework for
monetary policy involved political economy contributions. An important
contribution for European macroeconomics here is Giavazzi and Pagano's
(\protect\hyperlink{ref-giavazzi1988}{1988}) article published in the
EER. The authors questioned the advantages of adhering to the European
monetary system (EMS) for countries with higher rate of inflation. They
deal with the idea that the EMS would constitute a solution to the
time-consistency problem: it would ``tie the hands'' of high-inflation
countries which would have to keep their exchange rate stable, thus
reducing their incentive to generate surprise inflation and increasing
the credibility of monetary authorities in these countries. The adhesion
to the EMS thus ``parallels that in Rogoff
(\protect\hyperlink{ref-rogoff1985b}{1985}), who shows that the
non-cooperative rate of inflation can be reduced `through a system of
rewards and punishments which alters the incentives of the central
bank'\,'' (\protect\hyperlink{ref-giavazzi1988}{Giavazzi and Pagano,
1988, p. 1057}). The question was whether the adhesion to the EMS would
be ``welfare-improving'' for high-inflation countries
(\emph{ibid.}).\footnote{Giavazzi and Pagano's article constitutes a
  major reference for topic 3 and topic 8.} Still in the EER, Daniel
Laskar (\protect\hyperlink{ref-laskar1989}{1989}) also started from
Rogoff's (\protect\hyperlink{ref-rogoff1985b}{1985}) argument that
appointing a conservative central banker could be beneficial for
society, but extended the issue to a two-country model to discuss in
which cases appointing conservative central bankers in both countries
could be detrimental or beneficial to both countries.\footnote{We also
  observe another important way to approach the issue of the EMS, less
  empirical and a bit less framed in political economy terms, but still
  dealing with ``credibility'': the expectation from going out of the
  EMS and thus the credibility associated to some exchange rates in a
  target zone regime (\protect\hyperlink{ref-rose1994}{Rose and
  Svensson, 1994}; \protect\hyperlink{ref-svensson1993a}{Svensson,
  1993}).} Still regarding monetary policy framework, the monetary union
issue also stimulated contributions in the terms of political economy.
In his EER survey about the theoretical justification for the
convergence requirements of the Maastricht treaty, Paul distinguished
between two types of justification: \emph{(i)} ``the traditional theory
of optimum currency areas (OCA)'' and \emph{(ii)} ``the more recent `new
view' based on credibility issues''
(\protect\hyperlink{ref-degrauwe1996}{De Grauwe, 1996, pp. 1091--1092}).
Contrarily to the OCA theory, the second approach relied on the
intuition of the Barro-Gordon model. It analyses ``how countries can
gain (or loose) credibility by joining a monetary union'' and thus how
inflation rates would converge.\footnote{Whereas the OCA theory rather
  focused on the divergence in output and employment trends.} When
dealing with the monetary union issue, European macroeconomists favoured
the credibility approach and the OCA theory appeared less influential
(see notably topic 3).\footnote{Our goal is not to be fully
  comprehensive here. We can find another important discussion about
  monetary policy framework with a political economy taste on inflation
  targeting (see notably \protect\hyperlink{ref-svensson1997}{Svensson,
  1997}).}

A second area where a political economy framework was influential is the
issue of wage-setting. In the EER, Horn and Persson
(\protect\hyperlink{ref-horn1988}{1988}) studied the interaction between
exchange rate policy and the role of unions in wage-setting. If
devaluations used to maintain or increase competitiveness are followed
by compensatory wage increases, the effects on competitiveness are
cancelled and the economy is in a situation of a ``devaluation-wage
spiral'' (\protect\hyperlink{ref-horn1988}{Horn and Persson, 1988, p.
1621}). The point of departure of the authors' analysis is that ``if
wage setters are rational and forward-looking and understand the
objectives behind the government's exchange rate policy (\ldots) they
will anticipate exchange rate changes and take them into account in
their wage decisions'' (p.~1622). Their goal was thus to endogenise both
wage decisions and policy formation in a game-theoretic
framework.\footnote{Horn and Persson's
  (\protect\hyperlink{ref-horn1988}{1988}) article was an important
  reference for topic 3 (on monetary union), topic 6 (on exchange rate
  dynamics), topic 25 (on real wages, employment and contracts) and
  topic 8 (on strategic policy making issues).} Thorvadur Gylfasson and
Lindbeck's work on the links between wage-setting and monetary policy is
also enlightening here. In a 1984 article in the EER, they tried to
integrate together cost push and demand pull inflation in a Keynesian
framework taking into account the behaviour of aggregate supply and the
Phillips curve for wage formation
(\protect\hyperlink{ref-gylfason1984}{Gylfason and Lindbeck, 1984}). As
they acknowledged themselves, the issues raised by their model echoed
Malinvaud's (\protect\hyperlink{ref-malinvaud1977}{1977}) opposition
between Classical and Keynesian unemployment
(\protect\hyperlink{ref-gylfason1984}{Gylfason and Lindbeck, 1984, pp.
6--7}). Their article had a political economy flavour as they dealt with
``competing wage claims'' and framed their model as a duopoly problem
\emph{à la} Cournot. In their following article in the EER, they relied
explicitly on game theory to deal with the interaction of wages
determination and government spending
(\protect\hyperlink{ref-gylfason1986}{Gylfason and Lindbeck, 1986}).
Some years later, going back to the issue of wage setting and monetary
policy, they refer to the ``wage gaps'' debate of the 1970s and the
``cases where government efforts to reduce unemployment by bringing real
wages through price inflation were frustrated by subsequent nominal wage
increases'' (\protect\hyperlink{ref-gylfason1994}{Gylfason and Lindbeck,
1994, p. 34}), but no reference was made to classical and Keynesian
unemployment. They proposed a model in which wages are determined
``through collective bargaining among strong and well coordinated labor
unions'' (34) and explored its consequences for monetary policy in a
game-theoretic model similar to Barro and Gordon
(\protect\hyperlink{ref-barro1983}{1983a},
\protect\hyperlink{ref-barro1983c}{1983b}). To some extent, the
trajectory of Gylfason and Lindbeck's work is representative of the
transformation of European macroeconomics between the 1970s and the
1990s.

A third area concerns fiscal policy and European integration. Alberto
Alesina, Tabellini and Persson defended in the late 1980s and early
1990s the development of a ``positive theory'' of fiscal policy. The two
first explained in the AER that the goal was to ``{[}abandon{]} the
assumption that fiscal policy is set by a benevolent social planner who
maximizes the welfare of a representative consumer \ldots{} {[}for{]} an
economy with two policymakers with different objectives alternating in
office as a result of elections''
(\protect\hyperlink{ref-alesina1990}{Alesina and Tabellini, 1990}).
Persson and Tabellini (\protect\hyperlink{ref-persson1992}{1992})
defended a similar ``positive public finance'' research agenda. Their
goal was to understand how the rising European integration and the
removal of barriers to the mobility of capital, goods and labour could
affect the ``politico-economic equilibrium that determines fiscal
policy'' (\protect\hyperlink{ref-persson1992}{Persson and Tabellini,
1992, p. 689}).\footnote{There references are important for European
  economists in Topic 36, in comparison to US-based economists. We find
  a distinction similar in topic 22, where US economists mainly cited
  endogenous growth references, when Europeans were sticking to the
  political economy literature.}

As the three examples testify, after the late 1980s, (new) political
economy and its pioneering works
(\protect\hyperlink{ref-barro1983}{Barro and Gordon, 1983a},
\protect\hyperlink{ref-barro1983c}{1983b};
\protect\hyperlink{ref-kydland1977}{Kydland and Prescott, 1977};
\protect\hyperlink{ref-rogoff1985b}{Rogoff, 1985}) represented a
unifying framework for many European macroeconomists to deal with
different macroeconomic issues. It constituted a resource for tackling
the issues raised by the European integration and the building of a
European monetary system.

\hypertarget{conclusion}{%
\section*{Conclusion}\label{conclusion}}
\addcontentsline{toc}{section}{Conclusion}

Despite the widespread internationalisation and standardisation of
economics after the 1970s, European macroeconomics maintained
characteristic features distinct from US macroeconomics between the
1970s and 1990s. These features were of different nature. In the late
1970s and early 1980s, disequilibrium theory constituted a significant
part of the research undertaken by European macroeconomists, and did not
limit to the GET, but also represented a unifying framework to deal with
different macroeconomic issues (unemployment, stagflation, stabilization
policies, international trade, etc.). Disequilibrium theory represented
an alternative research program to the challenges raised by new
classical economists in the US and macroeconomics in Europe took for
some years another path. Nonetheless, disequilibrium theory did not
succeed in seducing US macroeconomists and, from a unifying research
framework in Europe, it progressively became a minor and declining
research program after the mid-1980s. In the same time, new classical
economists' contributions eventually encountered some success in Europe,
notably with the import of Real Business Cycle modelling, but above all
with the influence of the new political economy literature of Kydland
and Prescott (\protect\hyperlink{ref-kydland1977}{1977}) and Barro and
Gordon (\protect\hyperlink{ref-barro1983}{1983a},
\protect\hyperlink{ref-barro1983c}{1983b}).

In the 1990s, European macroeconomics appeared closer to its US
counterpart on the theoretical and methodological levels. However, it
does not mean that no difference existed and European macroeconomists
specialised on subjects that were only occasionally tackled (if at all)
by their US colleagues. The different macroeconomic situation of
European economies (particularly regarding unemployment), the issue of
economic interdependence between them, and the construction of the
European union probably pushed European economists towards different
avenues. And the importance of political economy in European
macroeconomics in the 1990s probably reflects these different incentives
for European researchers.

\newpage

\hypertarget{references}{%
\section*{References}\label{references}}
\addcontentsline{toc}{section}{References}

\hypertarget{refs}{}
\begin{CSLReferences}{1}{0}
\leavevmode\vadjust pre{\hypertarget{ref-alesina1990}{}}%
Alesina, A., Tabellini, G., 1990. A {Positive Theory} of {Fiscal
Deficits} and {Government Debt}. The Review of Economic Studies 57,
403--414. doi:\href{https://doi.org/10.2307/2298021}{10.2307/2298021}

\leavevmode\vadjust pre{\hypertarget{ref-ambrosino2018}{}}%
Ambrosino, A., Cedrini, M., Davis, J.B., Fiori, S., Guerzoni, M.,
Nuccio, M., 2018. What topic modeling could reveal about the evolution
of economics. Journal of Economic Methodology 25, 329--348.
doi:\href{https://doi.org/10.1080/1350178X.2018.1529215}{10.1080/1350178X.2018.1529215}

\leavevmode\vadjust pre{\hypertarget{ref-azariadis1975}{}}%
Azariadis, C., 1975. Implicit contracts and underemployment equilibria.
Journal of Political Economy 83, 1183--1202.

\leavevmode\vadjust pre{\hypertarget{ref-backhouse1997a}{}}%
Backhouse, R.E., 1997. The changing character of {British} economics.
History of Political Economy 28, 33--60.

\leavevmode\vadjust pre{\hypertarget{ref-backhouseboianovski2013}{}}%
Backhouse, R.E., Boianovski, M., 2013. Tranforming modern
macroeconomics. {Exploring} disequilibrium microfoundations (1956-2003).
{Cambridge University Press}, {Cambdrige}.

\leavevmode\vadjust pre{\hypertarget{ref-barro1976}{}}%
Barro, R.J., 1976. Rational expectations and the role of monetary
policy. Journal of Monetary Economics 2, 1--32.
doi:\href{https://doi.org/10.1016/0304-3932(76)90002-7}{10.1016/0304-3932(76)90002-7}

\leavevmode\vadjust pre{\hypertarget{ref-barro1983}{}}%
Barro, R.J., Gordon, D.B., 1983a. Rules, discretion and reputation in a
model of monetary policy. Journal of Monetary Economics 12, 101--121.
doi:\href{https://doi.org/10.1016/0304-3932(83)90051-X}{10.1016/0304-3932(83)90051-X}

\leavevmode\vadjust pre{\hypertarget{ref-barro1983c}{}}%
Barro, R.J., Gordon, D.B., 1983b. A {Positive Theory} of {Monetary
Policy} in a {Natural Rate Model}. Journal of Political Economy 91,
589--610.

\leavevmode\vadjust pre{\hypertarget{ref-barro1971}{}}%
Barro, R.J., Grossman, H.I., 1971. A {General Disequilibrium Model} of
{Income} and {Employment}. The American Economic Review 61, 82--93.

\leavevmode\vadjust pre{\hypertarget{ref-baumol1952}{}}%
Baumol, W.J., 1952. The transactions demand for cash: {An} inventory
theoretic approach. The Quarterly journal of economics 545--556.

\leavevmode\vadjust pre{\hypertarget{ref-bentolila1990}{}}%
Bentolila, S., Bertola, G., 1990. Firing costs and labour demand: How
bad is eurosclerosis? The Review of Economic Studies 57, 381--402.

\leavevmode\vadjust pre{\hypertarget{ref-bentolila1992a}{}}%
Bentolila, S., Saint-Paul, G., 1992. The macroeconomic impact of
flexible labor contracts, with an application to {Spain}. European
Economic Review 36, 1013--1047.
doi:\href{https://doi.org/10.1016/0014-2921(92)90043-V}{10.1016/0014-2921(92)90043-V}

\leavevmode\vadjust pre{\hypertarget{ref-bertola1990a}{}}%
Bertola, G., 1990. Job security, employment and wages. European Economic
Review 34, 851--879.
doi:\href{https://doi.org/10.1016/0014-2921(90)90066-8}{10.1016/0014-2921(90)90066-8}

\leavevmode\vadjust pre{\hypertarget{ref-bischof2012}{}}%
Bischof, J., Airoldi, E.M., 2012. Summarizing topical content with word
frequency and exclusivity. p. 201208.

\leavevmode\vadjust pre{\hypertarget{ref-blei2007}{}}%
Blei, D.M., Lafferty, J.D., 2007.
\href{https://www.jstor.org/stable/4537420}{A correlated topic model of
science}. The Annals of Applied Statistics 1, 17--35.

\leavevmode\vadjust pre{\hypertarget{ref-boumans2019}{}}%
Boumans, M., Duarte, P.G., 2019. The {History} of {Macroeconometric
Modeling} : {An Introduction}. History of Political Economy 51,
391--400.

\leavevmode\vadjust pre{\hypertarget{ref-brunosachs1985}{}}%
Bruno, M., Sachs, J.D., 1985. Economics of worldwide stagflation.
{National Bureau of Economic Research}, {Cambridge, Mass.}

\leavevmode\vadjust pre{\hypertarget{ref-bryant1979}{}}%
Bryant, J., Wallace, N., 1979. The inefficiency of interest-bearing
national debt. Journal of Political Economy 87, 365--381.

\leavevmode\vadjust pre{\hypertarget{ref-cherrier2018c}{}}%
Cherrier, B., Saïdi, A., 2018. The indeterminate fate of sunspots in
economics. History of Political Economy 50, 425--481.

\leavevmode\vadjust pre{\hypertarget{ref-cherrier2021}{}}%
Cherrier, B., Saïdi, A., 2021. Back to {Front}: {The Role} of
{Seminars}, {Conferences} and {Workshops} in the {History} of
{Economics}. {Preface} to the {Second Issue}. Revue d'economie politique
131, 723--727.

\leavevmode\vadjust pre{\hypertarget{ref-claveau2016}{}}%
Claveau, F., Gingras, Y., 2016.
\href{http://hope.dukejournals.org/cgi/content/short/48/4/551?rss=1}{Macrodynamics
of economics: A bibliometric history}. History of Political Economy.

\leavevmode\vadjust pre{\hypertarget{ref-clower1965}{}}%
Clower, R., 1965. The {Keynesian Counter-Revolution}: {A Theoretical
Appraisal}, in: The {Theory} of {Interest Rates}.

\leavevmode\vadjust pre{\hypertarget{ref-coats1996}{}}%
Coats, A.W., 1996. The Post-1945 Internationalization of Economics. Duke
University Press.

\leavevmode\vadjust pre{\hypertarget{ref-degrauwe1996}{}}%
De Grauwe, P., 1996. Monetary union and convergence economics. European
Economic Review, Papers and {Proceedings} of the {Tenth Annual Congress}
of the {European Economic Association} 40, 1091--1101.
doi:\href{https://doi.org/10.1016/0014-2921(95)00117-4}{10.1016/0014-2921(95)00117-4}

\leavevmode\vadjust pre{\hypertarget{ref-devroey2016}{}}%
De Vroey, M., 2016. A history of modern macroeconomics from keynes to
lucas and beyond. Cambridge University Press, Cambridge.

\leavevmode\vadjust pre{\hypertarget{ref-diamond1982}{}}%
Diamond, P.A., 1982. Aggregate demand management in search equilibrium.
Journal of political Economy 90, 881--894.

\leavevmode\vadjust pre{\hypertarget{ref-dingCommunity2011}{}}%
Ding, Y., 2011. Community detection: {Topological} vs. topical. Journal
of Informetrics 5, 498--514.
doi:\href{https://doi.org/10.1016/j.joi.2011.02.006}{10.1016/j.joi.2011.02.006}

\leavevmode\vadjust pre{\hypertarget{ref-dixit1978}{}}%
Dixit, A., 1978. The balance of trade in a model of temporary
equilibrium with rationing. The Review of Economic Studies 45, 393--404.

\leavevmode\vadjust pre{\hypertarget{ref-dixit1980}{}}%
Dixit, A., Norman, V., 1980. Theory of {International Trade}: {A Dual},
{General Equilibrium Approach}. {Cambridge University Press}.

\leavevmode\vadjust pre{\hypertarget{ref-drazen2002}{}}%
Drazen, A., 2002. Political {Economy} in {Macroeconomics}, Nouvelle. ed.
{Princeton University Press}, {New Haven}.

\leavevmode\vadjust pre{\hypertarget{ref-dreze1981}{}}%
Drèze, J.H., Modigliani, F., 1981. The trade-off between real wages and
employment in an open economy ({Belgium}). European Economic Review 15,
1--40.
doi:\href{https://doi.org/10.1016/0014-2921(81)90065-9}{10.1016/0014-2921(81)90065-9}

\leavevmode\vadjust pre{\hypertarget{ref-duartelima2012a}{}}%
Duarte, P.G., Lima, G.T., 2012. Microfoundations {Reconsidered}. {The
Relationship} of {Micro} and {Macroeconomics} in {Historical
Perspective}. {Edward Elgar Publishing}, {Cheltenham}.

\leavevmode\vadjust pre{\hypertarget{ref-duppe2017}{}}%
Düppe, T., 2017. How modern economics learned french: Jacques drèze and
the foundation of CORE. The European Journal of the History of Economic
Thought 24, 238--273.

\leavevmode\vadjust pre{\hypertarget{ref-fama1975}{}}%
Fama, E.F., 1975. Short-term interest rates as predictors of inflation.
American Economic Review 65.

\leavevmode\vadjust pre{\hypertarget{ref-fourcade2006}{}}%
Fourcade, M., 2006. The construction of a global profession: The
transnationalization of economics. American Journal of Sociology 112,
145194.

\leavevmode\vadjust pre{\hypertarget{ref-fourcade2009}{}}%
Fourcade, M., 2009. Economists and societies: Discipline and profession
in the united states, britain, and france, 1890s to 1990s. Princeton
University Press, Princeton.

\leavevmode\vadjust pre{\hypertarget{ref-frasser2020}{}}%
Frasser, C., 2020. Essays on liquidity-based asset classification and
illegal means of payment (PhD thesis). Université Paris 1
Panthéon-Sorbonne, Paris.

\leavevmode\vadjust pre{\hypertarget{ref-frenkel1976}{}}%
Frenkel, J.A., Johnson, H.G. (Eds.), 1976. The monetary approach to the
balance of payments. {G. Allen \& Unwin}, {London}.

\leavevmode\vadjust pre{\hypertarget{ref-friedman1957}{}}%
Friedman, M., 1957. A theory of the consumption function. Princeton
University Press, Princeton.

\leavevmode\vadjust pre{\hypertarget{ref-friedman1963}{}}%
Friedman, M., Schwartz, A.J., 1963. A {Monetary} history of the {United}
{States} 1867-1960, Studies in business cycles. Princeton university
press, Princeton.

\leavevmode\vadjust pre{\hypertarget{ref-galvaodealmeida2021}{}}%
Galvão de~Almeida, R., 2021. A {Macroeconomic View} of {Public Choice}:
{New Political Macroeconomics} as a {Separate Tradition} of {Public
Choice}. Œconomia. History, Methodology, Philosophy 77--105.
doi:\href{https://doi.org/10.4000/oeconomia.10402}{10.4000/oeconomia.10402}

\leavevmode\vadjust pre{\hypertarget{ref-giavazzi1988}{}}%
Giavazzi, F., Pagano, M., 1988. The advantage of tying one's hands:
{EMS} discipline and central bank credibility. European economic review
32, 1055--1075.

\leavevmode\vadjust pre{\hypertarget{ref-gottfries1992}{}}%
Gottfries, N., 1992. Insiders, outsiders, and nominal wage contracts.
Journal of Political Economy 100, 252--270.

\leavevmode\vadjust pre{\hypertarget{ref-gottfries1987}{}}%
Gottfries, N., Horn, H., 1987. Wage formation and the persistence of
unemployment. The Economic Journal 97, 877--884.

\leavevmode\vadjust pre{\hypertarget{ref-goutsmedt2021b}{}}%
Goutsmedt, A., 2021. From the {Stagflation} to the {Great Inflation}:
{Explaining} the {US} economy of the 1970s. Revue d'Economie Politique
131, 557--582.

\leavevmode\vadjust pre{\hypertarget{ref-goutsmedtetal2019}{}}%
Goutsmedt, A., Pinzon-Fuchs, E., Renault, M., Sergi, F., 2019. Reacting
to the lucas critique: {The} keynesians' replies. History of Political
Economy 51, 533--556.
doi:\href{https://doi.org/10.1215/00182702-7551912}{10.1215/00182702-7551912}

\leavevmode\vadjust pre{\hypertarget{ref-goutsmedt2021}{}}%
Goutsmedt, A., Renault, M., Sergi, F., 2021. European {Economics} and
the {Early Years} of the {``{International Seminar} on
{Macroeconomics}.''} Revue d'Economie Politique 131, 693--722.

\leavevmode\vadjust pre{\hypertarget{ref-grubb1982}{}}%
Grubb, D., Jackman, R., Layard, R., 1982. Causes of the {Current
Stagflation}. The Review of Economic Studies 49, 707--730.
doi:\href{https://doi.org/10.2307/2297186}{10.2307/2297186}

\leavevmode\vadjust pre{\hypertarget{ref-gylfason1984}{}}%
Gylfason, T., Lindbeck, A., 1984. Competing wage claims, cost inflation,
and capacity utilization. European Economic Review 24, 1--21.
doi:\href{https://doi.org/10.1016/0014-2921(84)90010-2}{10.1016/0014-2921(84)90010-2}

\leavevmode\vadjust pre{\hypertarget{ref-gylfason1986}{}}%
Gylfason, T., Lindbeck, A., 1986. Endogenous unions and governments: {A}
game-theoretic approach. European Economic Review 30, 5--26.
doi:\href{https://doi.org/10.1016/0014-2921(86)90029-2}{10.1016/0014-2921(86)90029-2}

\leavevmode\vadjust pre{\hypertarget{ref-gylfason1994}{}}%
Gylfason, T., Lindbeck, A., 1994. The interaction of monetary policy and
wages. Public Choice 79, 33--46.

\leavevmode\vadjust pre{\hypertarget{ref-hagemann2011a}{}}%
Hagemann, H., 2011. European émigrés and the {``{Americanization}''} of
economics. The European Journal of the History of Economic Thought 18,
643--671.
doi:\href{https://doi.org/10.1080/09672567.2011.629056}{10.1080/09672567.2011.629056}

\leavevmode\vadjust pre{\hypertarget{ref-hall1978b}{}}%
Hall, R.E., 1978. Stochastic implications of the life cycle-permanent
income hypothesis: Theory and evidence. Journal of political economy 86,
971--987.

\leavevmode\vadjust pre{\hypertarget{ref-hart1982}{}}%
Hart, O., 1982. A model of imperfect competition with {Keynesian}
features. The Quarterly Journal of Economics 97, 109--138.

\leavevmode\vadjust pre{\hypertarget{ref-hesse2012}{}}%
Hesse, J.-O., 2012. The {``{Americanisation}''} of {West German}
economics after the {Second World War}: {Success}, failure, or something
completely different? The European Journal of the History of Economic
Thought 19, 67--98.
doi:\href{https://doi.org/10.1080/09672567.2010.487283}{10.1080/09672567.2010.487283}

\leavevmode\vadjust pre{\hypertarget{ref-hoover1988}{}}%
Hoover, K.D. (Ed.), 1988. The new classical macroeconomics, The
{International} library of critical writings in economics. E. Elgar,
Aldershot (GB) Brookfield (Vt.).

\leavevmode\vadjust pre{\hypertarget{ref-hoover2012}{}}%
Hoover, K.D., 2012. Microfoundational programs, in: Duarte, P.G., Lima,
G.T. (Eds.), Microfoundations {Reconsidered}. Edward Elgar Publishing,
Cheltenham, pp. 19--61.

\leavevmode\vadjust pre{\hypertarget{ref-horn1988}{}}%
Horn, H., Persson, T., 1988. Exchange rate policy, wage formation and
credibility. European Economic Review 32, 1621--1636.
doi:\href{https://doi.org/10.1016/0014-2921(88)90021-9}{10.1016/0014-2921(88)90021-9}

\leavevmode\vadjust pre{\hypertarget{ref-jel1991}{}}%
JEL, 1991. \href{https://www.jstor.org/stable/2727351}{Classification
system: Old and new categories}. Journal of Economic Literature 29,
xviii--xxviii.

\leavevmode\vadjust pre{\hypertarget{ref-kiyotaki1989}{}}%
Kiyotaki, N., Wright, R., 1989. On money as a medium of exchange.
Journal of political Economy 97, 927--954.

\leavevmode\vadjust pre{\hypertarget{ref-kiyotaki1993}{}}%
Kiyotaki, N., Wright, R., 1993. A search-theoretic approach to monetary
economics. The American Economic Review 63--77.

\leavevmode\vadjust pre{\hypertarget{ref-kydland1977}{}}%
Kydland, F.E., Prescott, E.C., 1977. Rules {Rather} than {Discretion}:
{The Inconsistency} of {Optimal Plans}. Journal of Political Economy 85,
473--491.

\leavevmode\vadjust pre{\hypertarget{ref-laskar1989}{}}%
Laskar, D., 1989. Conservative central bankers in a two-country world.
European Economic Review 33, 1575--1595.
doi:\href{https://doi.org/10.1016/0014-2921(89)90079-2}{10.1016/0014-2921(89)90079-2}

\leavevmode\vadjust pre{\hypertarget{ref-layard1986}{}}%
Layard, R., Nickell, S., 1986. Unemployment in {Britain}. Economica 53,
S121--S169. doi:\href{https://doi.org/10.2307/2554377}{10.2307/2554377}

\leavevmode\vadjust pre{\hypertarget{ref-layard1991a}{}}%
Layard, R., Nickell, S., Jackman, R., 1991. Unemployment: {Macroeconomic
Performance} and the {Labour Market}. {Oxford University Press},
{Oxford}.
doi:\href{https://doi.org/10.1093/acprof:oso/9780199279166.001.0001}{10.1093/acprof:oso/9780199279166.001.0001}

\leavevmode\vadjust pre{\hypertarget{ref-liAdding2012}{}}%
Li, D., Ding, Y., Shuai, X., Bollen, J., Tang, J., Chen, S., Zhu, J.,
Rocha, G., 2012. Adding community and dynamic to topic models. Journal
of Informetrics 6, 237--253.
doi:\href{https://doi.org/10.1016/j.joi.2011.11.004}{10.1016/j.joi.2011.11.004}

\leavevmode\vadjust pre{\hypertarget{ref-lindbeck1986}{}}%
Lindbeck, A., Snower, D.J., 1986. Wage setting, unemployment, and
insider-outsider relations. The American Economic Review 76, 235--239.

\leavevmode\vadjust pre{\hypertarget{ref-lindbeck1987a}{}}%
Lindbeck, A., Snower, D.J., 1987. Efficiency wages versus insiders and
outsiders. European Economic Review 31, 407--416.
doi:\href{https://doi.org/10.1016/0014-2921(87)90058-4}{10.1016/0014-2921(87)90058-4}

\leavevmode\vadjust pre{\hypertarget{ref-lucas1973}{}}%
Lucas, R.E., 1973. \href{http://www.jstor.org/stable/1914364}{Some
international evidence on output-inflation tradeoffs}. The American
Economic Review 326--334.

\leavevmode\vadjust pre{\hypertarget{ref-lucas1972}{}}%
Lucas, R.E., Jr., 1972. Expectations and the neutrality of money.
Journal of Economic Theory 4, 103--124.
doi:\href{https://doi.org/10.1016/0022-0531(72)90142-1}{10.1016/0022-0531(72)90142-1}

\leavevmode\vadjust pre{\hypertarget{ref-maes2005}{}}%
Maes, I., Buyst, E., 2005. Migration and americanization: The special
case of belgian economics. The European Journal of the History of
Economic Thought 12, 7388.

\leavevmode\vadjust pre{\hypertarget{ref-malinvaud1977}{}}%
Malinvaud, E., 1977. The theory of unemployment reconsidered.
{Basil-Blackwell}, {Oxford}.

\leavevmode\vadjust pre{\hypertarget{ref-malinvaud1986}{}}%
Malinvaud, E., 1986. The rise of unemployment in {France}. Economica 53,
S197--S217.

\leavevmode\vadjust pre{\hypertarget{ref-maoTopic2017}{}}%
Mao, J., Cao, Y., Lu, K., Li, G., 2017. Topic scientific community in
science: A combined perspective of scientific collaboration and topics.
Scientometrics 112, 851--875.
doi:\href{https://doi.org/10.1007/s11192-017-2418-7}{10.1007/s11192-017-2418-7}

\leavevmode\vadjust pre{\hypertarget{ref-morgan1998}{}}%
Morgan, M.S., Rutherford, M., 1998.
\href{http://search.ebscohost.com/login.aspx?direct=true\&db=bth\&AN=7752144\&lang=fr\&site=ehost-live}{American
economics: The character of the transformation}. History of Political
Economy 30, 1--26.

\leavevmode\vadjust pre{\hypertarget{ref-mortensen1994}{}}%
Mortensen, D.T., Pissarides, C.A., 1994. Job creation and job
destruction in the theory of unemployment. The review of economic
studies 61, 397415.

\leavevmode\vadjust pre{\hypertarget{ref-persson1992}{}}%
Persson, T., Tabellini, G., 1992. The {Politics} of 1992: {Fiscal
Policy} and {European Integration}. The Review of Economic Studies 59,
689--701. doi:\href{https://doi.org/10.2307/2297993}{10.2307/2297993}

\leavevmode\vadjust pre{\hypertarget{ref-persson2002}{}}%
Persson, T., Tabellini, G.E., 2002. Political economics: Explaining
economic policy. {MIT press}.

\leavevmode\vadjust pre{\hypertarget{ref-pissarides1990}{}}%
Pissarides, C.A., 1990. Equilibrium unemployment theory. MIT press.

\leavevmode\vadjust pre{\hypertarget{ref-plassard2021}{}}%
Plassard, R., Renault, M., Rubin, G., 2021. Modelling market dynamics :
{Jean-Pascal Bénassy}, {Edmond Malinvaud}, and the development of
disequilibrium macroeconomics. Modelling market dynamics : Jean-Pascal
Bénassy, Edmond Malinvaud, and the development of disequilibrium
macroeconomics 83--114.
doi:\href{https://doi.org/10.19272/202106101005}{10.19272/202106101005}

\leavevmode\vadjust pre{\hypertarget{ref-portes1987}{}}%
Portes, R., 1987. Economics in europe. European Economic Review 31,
1329--1340.
doi:\href{https://doi.org/10.1016/S0014-2921(87)80021-1}{10.1016/S0014-2921(87)80021-1}

\leavevmode\vadjust pre{\hypertarget{ref-qin2013a}{}}%
Qin, D., 2013. {A History of Econometrics: The Reformation from the
1970s}. {Oxford University Press}, {Oxford}.

\leavevmode\vadjust pre{\hypertarget{ref-renault2020a}{}}%
Renault, M., 2020. Edmond {Malinvaud}'s {Criticisims} of the {New
Classical Economics}: {Restoring} the {Nature} and the {Rationale} of
the {Old Keynesians}' {Opposition}. Journal of the History of Economic
Thought 42, 563--585.
doi:\href{https://doi.org/10.1017/S1053837219000610}{10.1017/S1053837219000610}

\leavevmode\vadjust pre{\hypertarget{ref-renault2022}{}}%
Renault, M., 2022. Theory to the {Rescue} of the {Large-Scale Models}:
{Edmond Malinvaud}'s {Alternative View} on the {Search} for
{Microfoundations}. History of Political Economy 54.

\leavevmode\vadjust pre{\hypertarget{ref-rogoff1985b}{}}%
Rogoff, K., 1985. The optimal degree of commitment to a monetary target.
Quarterly Journal of Economics 100, 1169--1190.

\leavevmode\vadjust pre{\hypertarget{ref-rose1994}{}}%
Rose, A.K., Svensson, L.E.O., 1994. European exchange rate credibility
before the fall. European Economic Review 38, 1185--1216.
doi:\href{https://doi.org/10.1016/0014-2921(94)90067-1}{10.1016/0014-2921(94)90067-1}

\leavevmode\vadjust pre{\hypertarget{ref-sandelin1997}{}}%
Sandelin, B., Ranki, S., 1997. Internationalization or americanization
of swedish economics? The European Journal of the History of Economic
Thought 4, 284--298.
doi:\href{https://doi.org/10.1080/10427719700000040}{10.1080/10427719700000040}

\leavevmode\vadjust pre{\hypertarget{ref-sargent1975}{}}%
Sargent, T.J., Wallace, N., 1975.
\href{http://econpapers.repec.org/article/ucpjpolec/v_3A83_3Ay_3A1975_3Ai_3A2_3Ap_3A241-54.htm}{"{Rational}"
{Expectations}, the {Optimal} {Monetary} {Instrument}, and the {Optimal}
{Money} {Supply} {Rule}}. Journal of Political Economy 83, 241--54.

\leavevmode\vadjust pre{\hypertarget{ref-sargent1982}{}}%
Sargent, T.J., Wallace, N., 1982.
\href{http://www.jstor.org/stable/1830945}{The real-bills doctrine
versus the quantity theory: {A} reconsideration}. The Journal of
Political Economy 90, 1212--1236.

\leavevmode\vadjust pre{\hypertarget{ref-shen2019}{}}%
Shen, S., Zhu, D., Rousseau, R., Su, X., Wang, D., 2019. A refined
method for computing bibliographic coupling strengths. Journal of
Informetrics 13, 605--615.
doi:\href{https://doi.org/10.1016/j.joi.2019.01.012}{10.1016/j.joi.2019.01.012}

\leavevmode\vadjust pre{\hypertarget{ref-sneessens1986}{}}%
Sneessens, H.R., Drèze, J.H., 1986. A {Discussion} of {Belgian
Unemployment}, {Combining Traditional Concepts} and {Disequilibrium
Econometrics}. Economica 53, S89--S119.
doi:\href{https://doi.org/10.2307/2554376}{10.2307/2554376}

\leavevmode\vadjust pre{\hypertarget{ref-svensson1993a}{}}%
Svensson, L.E.O., 1993. Assessing target zone credibility: {Mean}
reversion and devaluation expectations in the {ERM},
1979\textendash 1992. European Economic Review 37, 763--793.
doi:\href{https://doi.org/10.1016/0014-2921(93)90087-Q}{10.1016/0014-2921(93)90087-Q}

\leavevmode\vadjust pre{\hypertarget{ref-svensson1997}{}}%
Svensson, L.E.O., 1997. Inflation forecast targeting: {Implementing} and
monitoring inflation targets. European Economic Review 41, 1111--1146.
doi:\href{https://doi.org/10.1016/S0014-2921(96)00055-4}{10.1016/S0014-2921(96)00055-4}

\leavevmode\vadjust pre{\hypertarget{ref-traag2019}{}}%
Traag, V.A., Waltman, L., van Eck, N.J., 2019. From louvain to leiden:
Guaranteeing well-connected communities. Scientific reports 9, 112.

\leavevmode\vadjust pre{\hypertarget{ref-trejos1995}{}}%
Trejos, A., Wright, R., 1995. Search, bargaining, money, and prices.
Journal of political Economy 103, 118--141.

\leavevmode\vadjust pre{\hypertarget{ref-truc2021}{}}%
Truc, A., Claveau, F., Santerre, O., 2021. Economic methodology: a
bibliometric perspective. Journal of Economic Methodology 1--12.
doi:\href{https://doi.org/10.1080/1350178X.2020.1868774}{10.1080/1350178X.2020.1868774}

\leavevmode\vadjust pre{\hypertarget{ref-waelbroeck1969}{}}%
Waelbroeck, J.L., Glejser, H., 1969. Editor's introduction. European
Economic Review 1, 3--6.
doi:\href{https://doi.org/10.1016/0014-2921(69)90016-6}{10.1016/0014-2921(69)90016-6}

\leavevmode\vadjust pre{\hypertarget{ref-yanTopics2012}{}}%
Yan, E., Ding, Y., Milojević, S., Sugimoto, C.R., 2012. Topics in
dynamic research communities: {An} exploratory study for the field of
information retrieval. Journal of Informetrics 6, 140--153.
doi:\href{https://doi.org/10.1016/j.joi.2011.10.001}{10.1016/j.joi.2011.10.001}

\end{CSLReferences}

\newpage

\hypertarget{appendices}{%
\section*{Appendices}\label{appendices}}
\addcontentsline{toc}{section}{Appendices}

\hypertarget{a---summary-tables}{%
\subsection*{A - Summary Tables}\label{a---summary-tables}}
\addcontentsline{toc}{subsection}{A - Summary Tables}

Here are the tables listing the different clusters and topics, with
their synthetic indicator of how much they are ``European'' (see
\protect\hyperlink{network}{Appendix B.3.} and
\protect\hyperlink{topic}{Appendix B.4.} for explanations on the
calculus of the indicator).

\begin{table}[!h]

\caption{\label{tab:summary-communities}Summary of Bibliographic Clusters}
\centering
\fontsize{9}{11}\selectfont
\begin{tabular}[t]{lr}
\toprule
Communities & Differences\\
\midrule
\cellcolor{gray!6}{Modeling Consumption \& Production} & \cellcolor{gray!6}{0.6998876}\\
Disequilibrium \& Keynesian Macro & 0.6245388\\
\cellcolor{gray!6}{Optimal Taxation 1} & \cellcolor{gray!6}{0.4222809}\\
International Macroeconomics \& Target Zone & 0.3608973\\
\cellcolor{gray!6}{Political Economics of Central Banks} & \cellcolor{gray!6}{0.3153792}\\
\addlinespace
Exchange Rate Dynamics & 0.2332856\\
\cellcolor{gray!6}{Taxation, Tobin's Q \& Monetarism} & \cellcolor{gray!6}{0.2194353}\\
Theory of Unemployment \& Job Dynamics & 0.1543966\\
\cellcolor{gray!6}{Capital \& Income Taxation} & \cellcolor{gray!6}{0.1212048}\\
Coordination \& Sunspots 2 & 0.1072461\\
\addlinespace
\cellcolor{gray!6}{Target Zone \& Currency Crises} & \cellcolor{gray!6}{0.0804988}\\
Monetary Policy, Financial Transmission \& Cycles 2 & 0.0681915\\
\cellcolor{gray!6}{Business Cycles, Cointegration \& Trends} & \cellcolor{gray!6}{0.0681889}\\
Optimal Taxation 2 & 0.0120000\\
\cellcolor{gray!6}{Taxation, Debt \& Growth} & \cellcolor{gray!6}{0.0054730}\\
\addlinespace
Terms of Trade \& Devaluation & -0.0325807\\
\cellcolor{gray!6}{Endogenous Growth} & \cellcolor{gray!6}{-0.0373527}\\
RBC & -0.0719855\\
\cellcolor{gray!6}{Monetary Policy, Financial Transmission \& Cycles 1} & \cellcolor{gray!6}{-0.0862415}\\
Coordination \& Sunspots 1 & -0.1172546\\
\addlinespace
\cellcolor{gray!6}{Exchange Rate Dynamics \& Expectations} & \cellcolor{gray!6}{-0.1589905}\\
Monetary Policy, Target \& Output Gap & -0.1934961\\
\cellcolor{gray!6}{REH, Monetary Policy \& Business Cycles} & \cellcolor{gray!6}{-0.2509395}\\
Inflation, Interest Rates \& Expectations & -0.2813913\\
\cellcolor{gray!6}{Monetary Approach of Balance of Payments} & \cellcolor{gray!6}{-0.2918580}\\
\addlinespace
Inflation \& Rigidities & -0.2939254\\
\cellcolor{gray!6}{Credit Rationing, Rational Expectations \& Imperfect Information} & \cellcolor{gray!6}{-0.2970840}\\
Demand for Money & -0.3392704\\
\cellcolor{gray!6}{New Theory of Money: Search, Bargaining...} & \cellcolor{gray!6}{-0.4094658}\\
Permanent Income Hypothesis \& Life-Cycle & -0.5461016\\
\addlinespace
\cellcolor{gray!6}{Monetary Economics \& Demand for Money} & \cellcolor{gray!6}{-0.6221023}\\
Intergenerational Model, Savings and Consumption & -1.1926593\\
\cellcolor{gray!6}{Marginal Taxation} & \cellcolor{gray!6}{-1.2846364}\\
\bottomrule
\end{tabular}
\end{table}

\newpage

\begingroup\fontsize{9}{11}\selectfont

\begin{longtable}[t]{>{}l>{}r>{\raggedright\arraybackslash}m{25em}}
\caption{\label{tab:summary-topics}Summary of Topics}\\
\toprule
Topics & Differences & Terms with the highest frex value\\
\midrule
\endfirsthead
\caption[]{Summary of Topics \textit{(continued)}}\\
\toprule
Topics & Differences & Terms with the highest frex value\\
\midrule
\endhead

\endfoot
\bottomrule
\endlastfoot
Topic 3 & 2.037 & system;
monetary
expansion;
monetary
system;
union;
expansion;
\cellcolor{gray!6}{stability}\\
Topic 4 & 1.695 & macroeconomics;
rich;
history;
robert;
divide;
lead\\
Topic 46 & 1.244 & german;
money
demand;
germany;
unite
kingdom;
unite;
\cellcolor{gray!6}{kingdom}\\
Topic 22 & 1.033 & political;
oecd
country;
oecd;
world;
country;
index\\
Topic 37 & 0.994 & unemployment;
job;
unemployment
rate;
creation;
flow;
phillips
\cellcolor{gray!6}{curve}\\
\addlinespace
Topic 6 & 0.767 & real
exchange;
real
exchange
rate;
exchange
rate;
flexible
exchange;
flexible
exchange
rate;
target
zone\\
Topic 25 & 0.756 & real
wage;
contract;
employment;
capacity;
wage;
\cellcolor{gray!6}{stickiness}\\
Topic 39 & 0.745 & trade
balance;
trade;
wealth;
relative
price;
balance;
external\\
Topic 8 & 0.708 & credibility;
strategic;
policy;
maker;
economic
policy;
policy
\cellcolor{gray!6}{rule}\\
Topic 28 & 0.699 & exchange
market;
foreign
exchange
market;
intervention;
foreign
exchange;
transaction;
transaction
cost\\
\addlinespace
Topic 13 & 0.615 & unanticipated;
activity;
economic
activity;
national;
economy;
\cellcolor{gray!6}{gap}\\
Topic 44 & 0.454 & level;
national
income;
inflationary;
equation;
price
level;
money
balance\\
Topic 31 & 0.374 & welfare
cost;
cost;
welfare;
bear;
survey;
\cellcolor{gray!6}{household}\\
Topic 43 & 0.369 & federal;
signal;
monetary
policy;
federal
reserve;
revision;
feed\\
Topic 23 & 0.349 & short
run;
run;
short;
burden;
indirect;
\cellcolor{gray!6}{externality}\\
\addlinespace
Topic 20 & 0.340 & inflation
target;
target;
stabilize;
central
bank;
central;
length\\
Topic 26 & 0.333 & inflation;
inflation
rate;
relative;
evidence;
dispersion;
nominal
\cellcolor{gray!6}{price}\\
Topic 45 & 0.258 & power
parity;
purchase
power
parity;
purchase
power;
power;
purchase;
parity\\
Topic 40 & 0.256 & economic
growth;
growth
rate;
productivity
growth;
growth;
fast;
\cellcolor{gray!6}{region}\\
Topic 36 & 0.208 & spend;
government
spend;
deficit;
fiscal;
government;
government
debt\\
\addlinespace
Topic 17 & 0.139 & indexation;
distortion;
labor
market;
labor;
product;
\cellcolor{gray!6}{corporate}\\
Topic 11 & 0.134 & walrasian;
competitive;
temporary;
existence;
search;
equilibrium\\
Topic 35 & 0.033 & likelihood;
variable;
estimation;
autoregressive;
variance;
endogenous
\cellcolor{gray!6}{variable}\\
Topic 24 & 0.002 & term;
spread;
short
term;
term
structure;
premium;
structure\\
Topic 15 & -0.031 & production;
class;
factor;
identical;
preference;
\cellcolor{gray!6}{input}\\
\addlinespace
Topic 50 & -0.097 & budget
constraint;
constraint;
project;
budget;
bad;
loan\\
Topic 30 & -0.146 & business
cycle;
business;
cycle;
real
business
cycle;
real
business;
\cellcolor{gray!6}{volatility}\\
Topic 47 & -0.160 & investment;
monopolistic;
dynamic;
competition;
macroeconomic;
replace\\
Topic 38 & -0.253 & asset
price;
financial
market;
stock
market;
return;
asset
market;
\cellcolor{gray!6}{stock}\\
Topic 21 & -0.258 & process;
procedure;
property;
incentive;
build;
endogenous\\
\addlinespace
Topic 42 & -0.296 & stationary;
rational
expectation
equilibrium;
expectation
equilibrium;
expectation;
unique;
rational
\cellcolor{gray!6}{expectation}\\
Topic 5 & -0.310 & price
adjustment;
oil;
price;
commodity
price;
sticky;
import\\
Topic 9 & -0.333 & skill;
asymmetric
information;
program;
change;
research;
\cellcolor{gray!6}{complementarity}\\
Topic 29 & -0.425 & capital
market;
mobility;
capital
mobility;
capital;
imperfect;
intensity\\
Topic 18 & -0.603 & generation;
overlap;
overlap
generation;
social
security;
live;
generation
\cellcolor{gray!6}{model}\\
\addlinespace
Topic 34 & -0.605 & plan;
stage;
multiple
equilibrium;
option;
crisis;
currency\\
Topic 33 & -0.664 & depression;
theory;
subsequent;
classical;
pure;
\cellcolor{gray!6}{principle}\\
Topic 48 & -0.715 & liquidity;
credit;
debt;
insurance;
access;
investor\\
Topic 10 & -0.799 & tax;
capital
income;
income
tax;
tax
system;
redistribution;
income
\cellcolor{gray!6}{taxation}\\
Topic 27 & -0.808 & perfect
foresight;
foresight;
time
vary;
time;
perfect;
continuous
time\\
\addlinespace
Topic 7 & -0.812 & control;
stochastic;
game;
equivalence;
equivalent;
\cellcolor{gray!6}{solution}\\
Topic 16 & -0.845 & public;
strategy;
finance;
local;
provision;
desirable\\
Topic 41 & -1.025 & optimal;
optimal
tax;
growth
model;
function;
optimal
policy;
optimal
\cellcolor{gray!6}{taxation}\\
Topic 12 & -1.339 & lm;
risk
aversion;
utility
function;
aversion;
intertemporal;
risk\\
Topic 49 & -1.427 & report;
composition;
regime;
critique;
puzzle;
\cellcolor{gray!6}{profit}\\
\addlinespace
Topic 1 & -1.608 & inventory;
hold;
association;
century;
create;
rationally\\
Topic 14 & -1.645 & income
distribution;
labor
income;
permanent;
sensitivity;
permanent
income;
\cellcolor{gray!6}{income}\\
Topic 32 & -1.746 & standard;
gold;
dollar;
reserve;
price
level;
size\\
Topic 2 & -1.916 & money
supply;
money
stock;
money;
fix
exchange;
supply;
fix
exchange
\cellcolor{gray!6}{rate}\\
Topic 19 & -1.927 & expect
rate;
cash;
sargent;
nominal;
expect
inflation;
expect\\*
\end{longtable}
\endgroup{}

\newpage

\hypertarget{appendix}{%
\subsection*{B - Information on the Methods}\label{appendix}}
\addcontentsline{toc}{subsection}{B - Information on the Methods}

\hypertarget{corpus}{%
\subsubsection*{B.1. Corpus Creation}\label{corpus}}
\addcontentsline{toc}{subsubsection}{B.1. Corpus Creation}

For the present study we used two different corpora. The first corpus is
composed of all EER articles and allows us to track how publications,
citations, references and authors affiliations evolved since the
creation of the journal in 1969 up to 2002. The second corpus is
composed of all macroeconomic articles published in the top five
economics journals (\emph{American Economic Review}, \emph{Journal of
Political Economy}, \emph{Econometrica}, \emph{Quarterly Journal of
Economics}, \emph{Review of Economic Studies}) and the EER.
Macroeconomic articles are identified thanks to the former and new
classification of the JEL codes (\protect\hyperlink{ref-jel1991}{JEL,
1991}).\footnote{See \ref{eer-top5-macro} for the list of JEL codes
  used.} This corpus is used as the basis for topic modelling and
bibliographic coupling analysis to contrast macroeconomics publications
authored by Europe-based and US-based authors, and/or published in top 5
journals and in the EER.

\hypertarget{eer-publications}{%
\paragraph*{EER Publications}\label{eer-publications}}
\addcontentsline{toc}{paragraph}{EER Publications}

For the creation of the first corpus composed of all EER articles, we
used a mix of \emph{Web of Science} (WoS) and \emph{Scopus}. While WoS
has all articles of the EER between 1969-1970 and 1974-2002, it is
missing most articles published between 1971 and 1973. To make up for
the missing data, we use Scopus to complete the dataset. This operation
required normalization of the Scopus dataset, and manual cleaning of
variables that were missing from Scopus compared to WoS. This mostly
includes cleaning the references to match \emph{Scopus} references with
WoS ones, and identification of author's affiliation.

\hypertarget{eer-top5-macro}{%
\paragraph*{EER and Top 5 Macroeconomics
Articles}\label{eer-top5-macro}}
\addcontentsline{toc}{paragraph}{EER and Top 5 Macroeconomics Articles}

The construction of this corpus is made in multiple steps:

\begin{enumerate}
\def\labelenumi{\arabic{enumi}.}
\item
  Identifying macroeconomics articles

  \begin{itemize}
  \item
    We identified all articles published in macroeconomics using JEL
    codes related to macroeconomics (we get JEL codes of Top 5 and EER
    articles thanks to the Econlit database). We consider that an
    article is a macroeconomics article if it has one of the following
    codes:

    \begin{itemize}
    \tightlist
    \item
      For old JEL codes (pre-1991): 023, 131, 132, 133, 134, 223, 311,
      313, 321, 431, 813, 824.
    \item
      For new JEL codes (1991 onward): all E, F3 and F4.\footnote{The
        new classification has a clear categorisation of Macroeconomics
        (the letter `E'), but we had F3 and F4 as they deal with
        international macroeconomics. For the older JEL codes, we use
        the table of correspondence produce by the \emph{Journal of
        Economic Literature} itself
        (\protect\hyperlink{ref-jel1991}{JEL, 1991}).}.
    \end{itemize}
  \end{itemize}
\item
  Using these JEL codes, we match econlit articles with WoS articles
  using the following matching variables:

  \begin{itemize}
  \tightlist
  \item
    Journal, Volume, First Page
  \item
    Year, Journal, First Page, Last Page
  \item
    Year, Volume, First Page, Last Page
  \item
    First Author, Year, Volume, First Page
  \item
    First Author, Title, Year
  \item
    Title, Year, First Page
  \end{itemize}
\item
  We then kept articles published in the EER (Corpus 1 improved with
  Scopus), and in the top five journals between 1973 and 2002. Out of
  the 3592 articles in econlit, we matched 3428. \footnote{Most of the
    unmatched articles are not `articles' properly speaking: they often
    are reply and comments on other published articles.}
\item
  Finally, we were able to collect abstracts:

  \begin{itemize}
  \tightlist
  \item
    using \emph{Scopus} for the EER. All abstracts have been matched
    with the EER corpus.
  \item
    using \emph{Microsoft Academics} to collect the highest number of
    available abstracts for the Top 5 as too many abstracts were missing
    in WoS or \emph{Scopus}. The abstracts extracted from this database
    are matched with our WoS Top 5 corpus using
  \end{itemize}
\end{enumerate}

Moreover, given that the size of our corpus is modest, we made an
extensive semi-automatic cleaning of references to improve references
identification by adding the most commonly cited books, book chapter,
and articles that are not otherwise identified in WoS when possible.

\hypertarget{b.2.-variable-creation}{%
\subsubsection*{B.2. Variable creation}\label{b.2.-variable-creation}}
\addcontentsline{toc}{subsubsection}{B.2. Variable creation}

\hypertarget{author-affiliation}{%
\paragraph*{Authors' affiliation}\label{author-affiliation}}
\addcontentsline{toc}{paragraph}{Authors' affiliation}

Authors' affiliations information were extracted from WoS. However, the
affiliations are not per author, but instead per institutional
departments per paper. For example, in the case of an article with two
authors from the same department, the department (and institution or
country associated with it) is only counted once. Similarly, a
single-authored article where the author has three affiliations can
result in one article having three affiliations. While in some cases we
can inferred the institutional affiliation for each author (e.g., one
institution, multiple authors), in others we cannot (e.g., two
institutions, three authors). For example, in an article with two
authors from Princeton and one author from Stanford, we only know that
the article was written by at least one author from Princeton and at
least one from Stanford, but not that the individual ratio was two
third.

For the descriptive analysis we simply use the count of unique
combinations of institution and country per article, and use occurrences
as an approximation of affiliation. However, for the more detailed
network and topic analysis, we restructured the information. given that
we are mostly interested in the relationship between Europe and US
economics, we simply looked at the share of papers authored by
Europe-based and US-based economists. While we do not have individual
affiliation, we know with certainty when a paper has only European
authors, only American authors, or a mix of the two. For this reason,
while the share of institutions within the corpus is only an estimation
based on the occurrences of affiliation, the information generated to
identify US authored papers and European authored paper is certain.

\hypertarget{network}{%
\subsubsection*{B.3. Bibliographic Coupling and Cluster
Detection}\label{network}}
\addcontentsline{toc}{subsubsection}{B.3. Bibliographic Coupling and
Cluster Detection}

A first way to identify potential differences between European and
American macroeconomics is to find articles written by Europeans and
published in a European journal, the EER, resembling each others but
dissimilar to American articles. To do that, we used bibliographic
coupling techniques. In a bibliographic coupling network, a link is
created between two articles when they have one or more references in
common. The more references that two articles have in common, the
stronger the link. Bibliographic coupling is one way to measure how
similar two articles are in a corpus. To normalize and weight the link
between two articles, we used the refined bibliographic coupling
strength of Shen et al.
(\protect\hyperlink{ref-shen2019}{2019}).\footnote{We have implemented
  this method in the \emph{biblionetwork} R package: Aurélien Goutsmedt,
  François Claveau and Alexandre Truc (2021). biblionetwork: A Package
  For Creating Different Types of Bibliometric Networks. R package
  version 0.0.0.9000.} This method normalized and weight the strength
between articles by taking into account two important elements

\begin{enumerate}
\def\labelenumi{\arabic{enumi}.}
\tightlist
\item
  The size of the bibliography of the two linked articles. It means that
  common references between two articles with long bibliography are
  weighted as less significant since the likeliness of potential common
  references is higher. Conversely, common references between two
  articles with a short bibliography is weighted as more significant.
\item
  The number of occurrences of each reference in the overall corpus.
  When a reference is shared between two articles, it is weighted as
  less significant if it is a very common reference across the entire
  corpus and very significant if it is scarcely cited. The assumption is
  that a very rare common reference points to a higher content
  similarity between two articles than a highly cited reference.
\end{enumerate}

For all macroeconomics articles published in the EER and in the Top 5,
we build the networks with 8-year overlapping windows. This results in
23.

We use Leiden detection algorithm
(\protect\hyperlink{ref-traag2019}{Traag et al., 2019}) that optimize
the modularity on each network to identify groups of articles that are
similar to each other and dissimilar to the rest of the network. We use
a resolution of 1 with 1000 iterations. This results in 466 clusters
across all networks. Because networks have a lot of overlaps, many
clusters between two periods are composed of the same articles. To
identify these clusters that are very similar between two time windows,
we considered that \emph{(i)} if at least 55\% of the articles in a
cluster of the first time window where in the same cluster in the second
time window, and that \emph{(ii)} if the cluster was also composed by at
least 55\% of articles of the first time window, \emph{then} it is the
same cluster. Simply put, if two clusters share a high number of
articles, and are both mostly composed by these shared articles, they
are considered the same cluster.

This gives us 154 clusters, with 33 that represent at least 4\% of a
network and are stable enough to exists for at least 2 time windows. We
are thus able to project the composition of each network and how nodes
circulated between clusters from one time window to the following one.

\begin{figure}[h]

{\centering \includegraphics[width=1\linewidth]{../../../../../../../../Mon Drive/data/EER/pictures/Graphs/Intertemporal_communities} 

}

\caption{The distribution of bibliographic clusters over time}\label{fig:plot-cluster-flow}
\end{figure}

For each cluster, we identify the US or European oriented nature of its
publications and authors. A first measure we used is the over/under
representation of European/US authors in the cluster. For each cluster,
and for articles written solely by European or by American authors, we
measured the log of the ratio of the share of European authored articles
in the cluster on the share of European authored articles in the
networks for the same time length the cluster exists:

\bigskip

\({\scriptstyle \text{Author EU/US Orientation}=\log(\frac{\text{Share Of European Authored Articles In The Cluster}}{\text{Share Of European Authored Articles In The Time Window}})}\)
\bigskip

We then use a second similar index for the publication venue of the
articles in the cluster. For each cluster, we subtracted the relative
share of EER publications to Top 5 publications in the cluster, to the
relative share of EER publications to Top 5 publications on the same
time window of the cluster:

\bigskip

\({\scriptstyle \text{Journal EU/US Orientation}=\log(\frac{\text{Share Of EER Articles In The Cluster}}{\text{Share Of EER  Articles In Time Window}})}\)
\bigskip

To get an overall index score of the European/US orientation of
clusters, we simply sum the two previous index:

\bigskip

\({\scriptstyle \text{Overall EU/US Orientation}=\text{Author EU/US Orientation} + \text{Journal EU/US Orientation}}\)
\bigskip

Finally, clusters are placed on a scatterplot with the Y-axis for the
EER vs Top 5 score, and the X-Axis for the American vs European authors
score. The size of the points captures the size of the cluster with the
number of articles that are in it, and the color of the cluster is
simply the sum of the two Y and X scores (see Figure
\ref{fig:plot-community-diff}).

Supplementary information about each cluster can be found in the online
appendix ``Bibliographic information about the EER and details on the
bibliographic coupling clusters''.

\hypertarget{topic}{%
\subsubsection*{B.4. Topic Modelling}\label{topic}}
\addcontentsline{toc}{subsubsection}{B.4. Topic Modelling}

\hypertarget{preprocessing}{%
\paragraph*{Preprocessing}\label{preprocessing}}
\addcontentsline{toc}{paragraph}{Preprocessing}

Our text corpus is composed of the titles and abstracts (when available)
of macroeconomics articles published in the Top 5 and EER. We have
several steps to clean our corpus before running our topic models:

\begin{enumerate}
\def\labelenumi{\arabic{enumi}.}
\tightlist
\item
  Titles and abstracts are merged together for all EER and Top 5
  articles.
\item
  We use the \emph{tidytext} and \emph{tokenizers} R packages to
  `tokenise' the resulting texts (when there is no abstract, only the
  title is thus tokenise)?\footnote{See Silge J, Robinson D (2016).
    ``tidytext: Text Mining and Analysis Using Tidy Data Principles in
    R.'' \emph{JOSS}, \emph{1}(3) and Lincoln A. Mullen et al., ``Fast,
    Consistent Tokenization of Natural Language Text,'' Journal of Open
    Source Software 3, no.23 (2018): 655.} Tokenisation is the process
  of transforming human-readable text into machine readable objects.
  Here, the text is split in unique words (unigrams), bigrams (pair of
  words) and trigrams. In other words, to each article is now associated
  a list of unigrams, bigrams and trigrams, some appearing several times
  in the same title \emph{plus} abstract.
\item
  Stop words are removed using the \emph{Snowball}
  dictionary.\footnote{See
    \url{http://snowball.tartarus.org/algorithms/english/stop.txt}.} We
  add to this dictionary some common verbs in abstract like
  ``demonstrate'', ``show'', ``explain''. Such verbs are likely to be
  randomly distributed in abstracts, and we want to limit the noise as
  much as possible.
\item
  We lemmatise the words using the \emph{textstem} package.\footnote{Rinker,
    T. W. (2018). textstem: Tools for stemming and lemmatizing text
    version 0.1.4. Buffalo, New York.} The lemmatisation is the process
  of grouping words together according to their ``lemma'' which depends
  on the context. For instance, different form of a verb are reduced to
  its infinitive form. The plural of nouns are reduced to the singular.
\end{enumerate}

\hypertarget{choosing-the-number-of-topics}{%
\paragraph*{Choosing the number of
topics}\label{choosing-the-number-of-topics}}
\addcontentsline{toc}{paragraph}{Choosing the number of topics}

We use the Correlated Topic Model (\protect\hyperlink{ref-blei2007}{Blei
and Lafferty, 2007}) method implemented in the \emph{STM} R
package.\footnote{Roberts ME, Stewart BM, Tingley D (2019). ``stm: An R
  Package for Structural Topic Models.'' \emph{Journal of Statistical
  Software}, \emph{91}(2), 1-40.}

From the list of words we have tokenised, cleaned and lemmatised, we
test different thresholds and choices by running different models:

\begin{itemize}
\tightlist
\item
  by exluding trigrams or not;
\item
  by removing the terms that are present in less than 0.6\% of the
  Corpus (20 articles), 0.8\% (27) and 1\% (34);
\item
  by removing articles with less than 8 words or with less than 12
  words.\footnote{Here, only articles with no abstract are impacted.}
\end{itemize}

Crossing all these criteria, we thus have 12 different possible
combinations. For each of these 12 different combinations, we have run
topic models for different number of topics from 20 to 110 with a gap of
5. The chosen model integrates trigrams, removes only terms that appear
in less than 0,6\% of the documents and keep all articles if they have
more than 8 words in their title \emph{plus} abstract. We choose to keep
the model with 50 topics.

We have chosen the criteria and the number of topics by comparing the
performance of the different models in terms of the FREX value
(\protect\hyperlink{ref-bischof2012}{Bischof and Airoldi, 2012}). We
have tested alternative specification for preprocessing steps and
different number of topics when the performance regarding FREX values
was similar. It seems to us that 50 topics allows us to have a model
with easily understandable topics and an interesting level of ``zoom''.
Indeed, increasing the number of topics just splits some topics in two,
but did not lead to fundamentally different results.

For each cluster, we are able to plot the distribution of the years of
publications of article, depending on their \emph{gamma} value for the
corresponding topic.

\begin{figure}[h]

{\centering \includegraphics[width=1\linewidth]{../../../../../../../../Mon Drive/data/EER/pictures/Graphs/topic_per_year} 

}

\caption{The distribution of topics over time}\label{fig:plot-topic-year}
\end{figure}

\hypertarget{studying-the-european-character-of-topics}{%
\paragraph*{Studying the European character of
topics}\label{studying-the-european-character-of-topics}}
\addcontentsline{toc}{paragraph}{Studying the European character of
topics}

To look at the features of the topics regarding our two variables of
interest (EER \emph{vs.} Top 5 publications and US authors \emph{vs.}
European authors), we use two methods. The first one keeps all articles
and, for each topic, calculate the average gamma value for articles
published in the EER and in the Top 5. We subtract the two means. We do
the same for articles written by European authors only and by US authors
only. The two resulting differences are plot in the following Figure
\ref{fig:plot-topic-diff-alternative}.

\begin{figure}[h]

{\centering \includegraphics[width=1\linewidth]{../../../../../../../../Mon Drive/data/EER/pictures/Graphs/mean_diff_plot_new_bw} 

}

\caption{The most European topics (Differences of mean method)}\label{fig:plot-topic-diff-alternative}
\end{figure}

In Figure \ref{fig:plot-topic-diff} in the text above, we are only
keeping, for each topic, articles with a \emph{gamma} value above 0.1.
We then calculate the log ratio of EER and Top 5 articles for each
topic:

\bigskip

\({\scriptstyle \text{Journal EU/US Orientation}=\log(\frac{\frac{\text{Share Of EER Articles In The Topic}}{\text{Total Share Of EER Articles}}} {\frac{\text{Share Of Top 5 Articles In The Topic}}{\text{Total Share Of Top 5 Articles}}})}\)

\bigskip

We do the same for articles written by Europe-based and US-based
authors:

\bigskip

\({\scriptstyle \text{Author EU/US Orientation}=\log(\frac{\frac{\text{Share Of European Authored Articles In The Topic}}{\text{Total Share Of European Authored Articles}}} {\frac{\text{Share Of US Authored Articles In The Topic}}{\text{Total Share Of US Authored Articles}}})}\)

\end{document}
