% Options for packages loaded elsewhere
\PassOptionsToPackage{unicode}{hyperref}
\PassOptionsToPackage{hyphens}{url}
%
\documentclass[
  12pt,
  onecolumn]{article}
\usepackage{amsmath,amssymb}
\usepackage{lmodern}
\usepackage{setspace}
\usepackage{iftex}
\ifPDFTeX
  \usepackage[T1]{fontenc}
  \usepackage[utf8]{inputenc}
  \usepackage{textcomp} % provide euro and other symbols
\else % if luatex or xetex
  \usepackage{unicode-math}
  \defaultfontfeatures{Scale=MatchLowercase}
  \defaultfontfeatures[\rmfamily]{Ligatures=TeX,Scale=1}
\fi
% Use upquote if available, for straight quotes in verbatim environments
\IfFileExists{upquote.sty}{\usepackage{upquote}}{}
\IfFileExists{microtype.sty}{% use microtype if available
  \usepackage[]{microtype}
  \UseMicrotypeSet[protrusion]{basicmath} % disable protrusion for tt fonts
}{}
\makeatletter
\@ifundefined{KOMAClassName}{% if non-KOMA class
  \IfFileExists{parskip.sty}{%
    \usepackage{parskip}
  }{% else
    \setlength{\parindent}{0pt}
    \setlength{\parskip}{6pt plus 2pt minus 1pt}}
}{% if KOMA class
  \KOMAoptions{parskip=half}}
\makeatother
\usepackage{xcolor}
\usepackage[a4paper,bindingoffset=0mm,inner=30mm,outer=30mm,top=30mm,bottom=30mm]{geometry}
\usepackage{graphicx}
\makeatletter
\def\maxwidth{\ifdim\Gin@nat@width>\linewidth\linewidth\else\Gin@nat@width\fi}
\def\maxheight{\ifdim\Gin@nat@height>\textheight\textheight\else\Gin@nat@height\fi}
\makeatother
% Scale images if necessary, so that they will not overflow the page
% margins by default, and it is still possible to overwrite the defaults
% using explicit options in \includegraphics[width, height, ...]{}
\setkeys{Gin}{width=\maxwidth,height=\maxheight,keepaspectratio}
% Set default figure placement to htbp
\makeatletter
\def\fps@figure{htbp}
\makeatother
\setlength{\emergencystretch}{3em} % prevent overfull lines
\providecommand{\tightlist}{%
  \setlength{\itemsep}{0pt}\setlength{\parskip}{0pt}}
\setcounter{secnumdepth}{5}
\newlength{\cslhangindent}
\setlength{\cslhangindent}{1.5em}
\newlength{\csllabelwidth}
\setlength{\csllabelwidth}{3em}
\newlength{\cslentryspacingunit} % times entry-spacing
\setlength{\cslentryspacingunit}{\parskip}
\newenvironment{CSLReferences}[2] % #1 hanging-ident, #2 entry spacing
 {% don't indent paragraphs
  \setlength{\parindent}{0pt}
  % turn on hanging indent if param 1 is 1
  \ifodd #1
  \let\oldpar\par
  \def\par{\hangindent=\cslhangindent\oldpar}
  \fi
  % set entry spacing
  \setlength{\parskip}{#2\cslentryspacingunit}
 }%
 {}
\usepackage{calc}
\newcommand{\CSLBlock}[1]{#1\hfill\break}
\newcommand{\CSLLeftMargin}[1]{\parbox[t]{\csllabelwidth}{#1}}
\newcommand{\CSLRightInline}[1]{\parbox[t]{\linewidth - \csllabelwidth}{#1}\break}
\newcommand{\CSLIndent}[1]{\hspace{\cslhangindent}#1}
\ifLuaTeX
\usepackage[bidi=basic]{babel}
\else
\usepackage[bidi=default]{babel}
\fi
\babelprovide[main,import]{british}
% get rid of language-specific shorthands (see #6817):
\let\LanguageShortHands\languageshorthands
\def\languageshorthands#1{}
\usepackage{booktabs}
\usepackage{longtable}
\usepackage{array}
\usepackage{multirow}
\usepackage{wrapfig}
\usepackage{float}
\usepackage{colortbl}
\usepackage{pdflscape}
\usepackage{tabu}
\usepackage{threeparttable}
\usepackage{threeparttablex}
\usepackage[normalem]{ulem}
\usepackage{makecell}
\usepackage{xcolor}
\ifLuaTeX
  \usepackage{selnolig}  % disable illegal ligatures
\fi
\IfFileExists{bookmark.sty}{\usepackage{bookmark}}{\usepackage{hyperref}}
\IfFileExists{xurl.sty}{\usepackage{xurl}}{} % add URL line breaks if available
\urlstyle{same} % disable monospaced font for URLs
\hypersetup{
  pdftitle={An Independent European Macroeconomics? A History of European Macroeconomics through the Lens of the European Economic Review},
  pdfauthor={Aurélien Goutsmedt; Alexandre Truc },
  pdflang={en-UK},
  hidelinks,
  pdfcreator={LaTeX via pandoc}}

\title{An Independent European Macroeconomics? A History of European
Macroeconomics through the Lens of the European Economic Review}
\author{Aurélien Goutsmedt\footnote{UCLouvain, ISPOLE; F.R.S.-FNRS.} \and Alexandre
Truc \footnote{Université Côte d'Azur, CNRS GREDEG France.}}
\date{2023-04-28}

\begin{document}
\maketitle
\begin{abstract}
Economics in Europe has encountered a process of internationalisation
since the 1970s. To a certain extent, this internationalisation is also
an `Americanisation' and many European departments and economics have
adopted the standards of US economics, notably mathematical modelling,
the use of econometrics, and the neoclassical theory as a modelling
benchmark. Regarding this process, we can wonder if European economics
has just been mimicking US economics since the 1970s, or if some
European specialities have survived or emerged.

In this article, we use topic modelling and bibliometric coupling to
identify what have been some European specialities between 1969 and
2002. We focus on macroeconomics, and we use the articles published in
the European Economic Review and compare their bibliographic references
and textual content (\emph{via} titles and abstracts) to what has been
published in the top 5 journals.

In the late 1970s and early 1980s, disequilibrium theory constituted a
significant part of the research undertaken by European macroeconomists,
and did not limit to the general equilibrium theory, but also
represented a unifying framework to deal with different macroeconomic
issues. After it lose its influence in the second part of the 1980s,
political economy occupied this role. It constituted a resource for
tackling the issues raised by the European integration and the building
of a European monetary system, and constituted a common language for
many European macroeconomists.
\end{abstract}

\setstretch{1.5}
\hypertarget{introduction}{%
\section{Introduction}\label{introduction}}

In 1987 in the \emph{European Economic Review}, the director of the
Centre for Economic Policy Research, Richard Portes, attempted to assess
the ``state and status of economics in Europe''. He regarded ``the
standard of comparison {[}as{]} obvious: the United States, by far the
dominant producer'' (\protect\hyperlink{ref-portes1987}{Portes, 1987, p.
1329}). He then asked ``whether there is now any economics outside and
independent of the United States.'' (p.~1330) He listed many clues
testifying of the US domination, ending it by the observation that ``the
leaders of the economics profession in Europe were trained as
postgraduates in the United States. Many take from the US their
professional standards, their views of what are the interesting
problems, and their approaches to them'' (\emph{ibid.}).

Indeed, in the early 1970s in many Western European countries, economics
had entered a process of internationalisation
(\protect\hyperlink{ref-fourcade2009}{Fourcade, 2009, chap. 3} and 4;
\protect\hyperlink{ref-fourcade2006}{Fourcade, 2006}).\footnote{Of
  course, the circulation of economic ideas has always been relatively
  internationalised and examples of knowledge and economists circulation
  before the 1970s abound (e.g.
  \protect\hyperlink{ref-hagemann2011a}{Hagemann, 2011};
  \protect\hyperlink{ref-hesse2012}{Hesse, 2012}). However, one observe
  a significant acceleration of this process after the 1970s
  (\protect\hyperlink{ref-coats1996}{Coats, 1996}), even though national
  rhythms may differ (e.g.
  \protect\hyperlink{ref-backhouse1997a}{Backhouse, 1997}).} To some
extent, such process was also a form of ``Americanisation''
(\protect\hyperlink{ref-coats1996}{Coats, 1996};
\protect\hyperlink{ref-goutsmedt2021}{Goutsmedt et al., 2021}): US
professional and intellectual standards were progressively adopted in
European countries, mimicking the functioning of the US academic field.
English gradually spread as the dominant language in economics
(\protect\hyperlink{ref-sandelin1997}{Sandelin and Ranki, 1997}) and
publications in peer-review journals became the norm for assessing
research productivity. The organisation of international events were
encouraged to boost research centres visibility
(\protect\hyperlink{ref-cherrier2021}{Cherrier and Saïdi, 2021};
\protect\hyperlink{ref-goutsmedt2021}{Goutsmedt et al., 2021}). In terms
of content, the Americanisation of the discipline in Europe favoured the
intellectual standards that had become widespread in the US in the
postwar era (\protect\hyperlink{ref-morgan1998}{Morgan and Rutherford,
1998}): the use of mathematical economics and econometrics, and the
reliance on neoclassical theory as a benchmark for modelling.\footnote{Of
  course, this process of Americanisation did not go without conflicts:
  many ``local conflicts'' emerged between more ``nationally-trained''
  economists (generally locally trained) and ``internationally-trained
  economists'' who had been often trained in the US
  (\protect\hyperlink{ref-fourcade2006}{Fourcade, 2006}). These
  conflicts involved intellectual matters (for instance around the
  relevance of the neoclassical theory) as well as institutional issues,
  like the criteria to assess the quality of economists' work and thus
  to determine hiring and promotion.}

In parallel to this Americanisation, we observe a process of
`Europeanisation': many initiatives from the first issue of the
\emph{European Economic Review} (EER) in 1969 to the creation of the
\emph{European Economic Association} (EEA) in 1984 promoted the
development of intellectual exchanges between European
economists---while obviously keeping US economics as a model. The
simultaneous spreading of US standards in Europe after the 1970s and the
promotion of a European economics transcending national traditions bring
us back to Portes's 1987 question: could a distinct European approach to
economics develop and maintain a degree of autonomy from the US in the
years following the 1970s?

Portes pointed out some European ``comparative advantages''
(\protect\hyperlink{ref-portes1987}{Portes, 1987, p. 1332}), even though
some of these European specialities had been initially pioneered by US
economists. He mentioned the prominence in Europe of ``general
equilibrium theory,'' ``international macroeconomic policy
coordination,'' or ``Non-Walrasian macroeconomics'' (\emph{ibid}.).
Goutsmedt et al. (\protect\hyperlink{ref-goutsmedt2021}{2021}) have also
highlighted that within the \emph{International Seminar on
Macroeconomics} (ISoM), whose annual proceedings were published in the
EER, disequilibrium macroeconomics and large-scale macroeconometric
modelling constituted important rallying research programs until the
mid-1980s for European economists involved in the ISoM.\footnote{In the
  rest of the article, we follow Backhouse and Boianovsky and use the
  expression ``disequilibrium macroeconomics'' to designate a research
  program that has been labelled in many ways: ``non-Walrasian theory,
  disequilibrium theory, equilibrium with rationing,
  \emph{non-tâtonnement} theory, fixed-price models''
  (\protect\hyperlink{ref-backhouseboianovski2013}{Backhouse and
  Boianovski, 2013, pp. 8--9}).}

The purpose of our article is to investigate systematically and
quantitatively the development and persistence of European specificities
in macroeconomics. Macroeconomics constituted a substantial part of
EER's publications, even representing almost half of all the articles in
the early 1980s (Figure \ref{fig:plot-jel}). Macroeconomics was also
instrumental in fostering collaborations between European economists as
evidenced by the ISoM (see Section \ref{rising-journal}). Regarding
EER's history and its significance in the promotion of a European
macroeconomics, we believe that EER's publications constitutes a
compelling perspective from which to examine the evolution of a European
``mainstream'' macroeconomics, as well as of the emergence and
persistence of `European specialities' within the field (see Section
\ref{EER-creation} for further insights on this point).\footnote{We use
  ``mainstream'' as a convenient way to indicate that our focus is on a
  specific subset of macroeconomics, which adhered to certain standards
  and fundamental theoretical assumptions, inspired by US economists.
  Several alternative approaches, such as Marxian and Sraffian economics
  or British Keynesianism and post-Keynesianism, held strong roots in
  Europe. Yet, they were considerably less likely to be featured in
  publications within the EER.} We define European specialities as
\emph{(i)} prevalent research themes (i.e.~representing a substantial
portion of European macroeconomists' research) \emph{(ii)} distinct from
what US-based economists were doing and \emph{(iii)} embraced by
Europe-based economists affiliated with a diverse array of institutions
across different European countries. This final criterion resonates with
the idea of a ``Europeanisation'': in the subsequent analysis, we take
care to differentiate between research areas predominantly established
in a single European country and those encompassing multiple countries,
fostering collaborations among macroeconomists across Western
Europe.\footnote{Up until 2002, economists from Eastern Europe were
  scarcely represented. For the sake of simplicity, we will use the term
  Europe.} Employing a combination of bibliometric coupling, topic
modelling and content reading, we pinpoint European specialities in the
period spanning 1973 to 2002.\footnote{The corpus we use (see section
  \ref{methods}) has very few abstracts between 1969 (the date of the
  creation of the EER) and 1972. Besides, there is no JEL code for EER
  articles before 1973, preventing us for identifying macroeconomics
  articles. After 2002 and the creation of the
  \href{https://academic.oup.com/jeea}{\emph{Journal of the European
  Economic Association}}, the EER was not the official journal of the
  EEA any more.}

The interplay between the internationalisation of macroeconomics and the
persistence of specialities presents a compelling avenue to contribute
to the foundation of a history of European macroeconomics, an area that
remains largely unexplored. Over the last decade, many historical
contributions have documented the evolution of macroeconomics in the
1970s and 1980s. These contributions have identified the major
trajectories of the discipline's transformation (especially the changes
brought about by new classical economists' contributions) and examined
the extent to which macroeconomics' methodology has evolved
(\protect\hyperlink{ref-devroey2016}{De Vroey, 2016};
\protect\hyperlink{ref-duartelima2012a}{Duarte and Lima, 2012}).
Historians have also underlined the discontinuities within these
transformations, as well as the resistance against them
(\protect\hyperlink{ref-goutsmedt2021b}{Goutsmedt, 2021};
\protect\hyperlink{ref-goutsmedtetal2019}{Goutsmedt et al., 2019};
\protect\hyperlink{ref-renault2020a}{Renault, 2020}), their varying
impact on applied and empirical works
(\protect\hyperlink{ref-boumans2019}{Boumans and Duarte, 2019};
\protect\hyperlink{ref-qin2013a}{Qin, 2013};
\protect\hyperlink{ref-renault2022}{Renault, 2022a}), but also the
existence of alternative theoretical research programmes
(\protect\hyperlink{ref-backhouseboianovski2013}{Backhouse and
Boianovski, 2013}; \protect\hyperlink{ref-cherrier2018c}{Cherrier and
Saïdi, 2018}; \protect\hyperlink{ref-hoover2012}{Hoover, 2012}).
Nevertheless, these historical contributions remained generally
US-centred. This can be readily attributed to the predominant influence
of US macroeconomists on the discipline, bolstered by the
internationalisation process previously discussed. Yet, it remains
essential to comprehend how European macroeconomics may have diverged
from the dominant US macroeconomics, how it has evolved at a distinct
pace, pursued alternative trajectories, and focussed on differing
issues. Furthermore, our article broadens the scope of historical
investigation to include the 1990s, a period that has yet not been
thoroughly explored by historians.

Our approach identifies different \emph{bibliometric clusters} and
\emph{topics} that are more associated to publication in the EER (rather
than in top 5 journals) and to Europe-based economists (see Section
\ref{methods} for details on method).\footnote{The article is also
  accompanied by a detailed methodological
  \protect\hyperlink{appendix}{Appendix}, as well as two online
  appendices listing the features of the bibliometric clusters
  (``Bibliographic information about the EER and details on the
  bibliographic coupling clusters'') and of the topics identified by our
  topic model (``Details on the topics''). Online appendices also
  display some statistics on countries and institutions for each cluster
  and topic.} This approach provides insights into the research areas
European macroeconomists focused on from the 1970s to the 1990s, in
contrast to their US counterparts. A meticulous examination of the
detailed findings allows us to paint a broader, albeit not complete,
portrait of the evolution of European macroeconomics since the 1970s.
Consistently with Portes (\protect\hyperlink{ref-portes1987}{1987}) and
Goutsmedt et al. (\protect\hyperlink{ref-goutsmedt2021}{2021}) claims,
disequilibrium theory appears as a unifying framework for European
macroeconomics between the mid-1970s and the mid-1980s. If it first
constituted a theoretical research program targeting the development of
general equilibrium theory, it also became an interpretative framework
to explain the stagflation and the European unemployment problem after
the 1970s (see Section \ref{disequilibrium}). Disequilibrium
theory---and notably Malinvaud's
(\protect\hyperlink{ref-malinvaud1977}{1977})---was a rallying point for
European macroeconomists when discussing various macroeconomic issues,
and even those who disagreed with its utility made their dissent
explicit. Its influence was thus far more extended than the
contributions of new classical economists like Robert Lucas, Thomas
Sargent or Robert Barro. However, disequilibrium progressively
disappeared from references in the second part of the 1980s and issues
like European unemployment were tackled mainly using other types of
frameworks. If no unifying and consistent theoretical framework has
taken over the disequilibrium theory (at least to the same extent), the
new political economy inspired by Kydland and Prescott
(\protect\hyperlink{ref-kydland1977}{1977}) and Barro and Gordon
(\protect\hyperlink{ref-barro1983}{1983a},
\protect\hyperlink{ref-barro1983c}{1983b}) brought new questions and a
common language for many contributions of European economists (see
Section \ref{political-economics}). The pioneering contributions of this
literature were carried by US economists, but it became a truly European
way to tackle many macroeconomic issues in the 1990s.

\begin{figure}[h]

{\centering \includegraphics[width=1\linewidth]{../../../../../../../../Mon Drive/data/EER/pictures/Graphs/mean_jel} 

}

\caption{Share of articles with at least one macroeconomics JEL code}\label{fig:plot-jel}
\end{figure}

\hypertarget{EER-creation}{%
\section{The creation of the EER}\label{EER-creation}}

\hypertarget{the-birth-of-a-european-project}{%
\subsection{The birth of a european
project}\label{the-birth-of-a-european-project}}

In 1969, Jean Waelbroeck and Herbert Glejser, both from the
\emph{Université Libre de Bruxelles} (ULB), launched the \emph{European
Economic Review}. The new review was planned to be the official journal
of the European Scientific Association of Applied Economics (ASEPELT),
which had been created in 1961 by Waelbroeck and another ULB economist:
Etienne Kirschen. Before 1969, this association published in English a
bulletin gathering research in econometrics and mathematical economics
(\protect\hyperlink{ref-waelbroeck1969}{Waelbroeck and Glejser, 1969, p.
4}). The EER took up this torch by publishing the same type of research.
Articles had to be published in English, the new ``\emph{lingua franca}
of economics'' triggering the process of ``internationalisation of our
science'' as Waelbroeck and Glejser polemically stated in the
introduction of the first issue (\emph{ibid.}).

The birth of such a project in Belgium is far from coincidental. Indeed,
the country exhibited a high effervescence regarding the
internationalisation of the discipline. In 1966, Jacques Drèze had
established the Center for Operations Research and Econometrics (CORE)
at the \emph{Katholieke Universiteit Leuven} (before its split), drawing
inspiration from the Cowles Commission and the Carnegie Institute of
Technology, which Drèze had visited in the 1950s
(\protect\hyperlink{ref-duppe2017}{Düppe, 2017}).\footnote{KU Leuven was
  split in 1968 between a Flemish and a French-speaking part, the latter
  giving birth to the \emph{Université Catholique de Louvain} at
  Louvain-La-Neuve, where the CORE eventually moved in the mid-1970s.}
The CORE developed a research program around macroeconomic modelling and
general equilibrium theory, and quickly stimulated the establishment of
a European research network of economists, notably through its large
visiting programme (\protect\hyperlink{ref-duppe2017}{Düppe, 2017};
\protect\hyperlink{ref-maes2005}{Maes and Buyst, 2005}). Encouraged by
Waelbroeck, the ULB department of economics joined the CORE in its first
years of existence (\protect\hyperlink{ref-maes2005}{Maes and Buyst,
2005, p. 79}).

From the beginning, the EER was conceived as a European project and the
composition of the editorial board testifies of it (Table
\ref{table:plot-boards}). But the EER being a Belgian-centred
initiative, Belgian institutions represented one fourth of authors'
affiliations in EER articles in the first years (Figure
\ref{fig:plot-authors}).\footnote{This is an approximation, as the
  affiliation per author is not available in our corpus and we only have
  the affiliations per article (see
  \protect\hyperlink{author-affiliation}{Appendix B.2.} for more
  details).{]}} Nonetheless, the geographical diversity of EER
authorship expanding significantly during the 1970s. When comparing
authors' affiliations data from \emph{The Economic Journal} and
\emph{Economica} as calculated in Backhouse
(\protect\hyperlink{ref-backhouse1997a}{1997, fig. 7} and 8), we see
that European countries excluding the UK were far better represented in
the EER.\footnote{In contrast to \emph{The Economic Journal} and
  \emph{Economica}, where British and US affiliations accounted
  approximately for 80\% of authors from the 1970s to the 1990s, these
  affiliations represented merely 30 to 40\% in the EER. Meanwhile,
  European countries (excluding the UK) represented nearly 50\%.}

\begingroup\fontsize{3}{5}\selectfont

\begin{longtable}[t]{>{}c>{}l>{}l>{}l>{}l>{}l>{}l>{}l}
\caption{\label{tab:table-boards}Share of countries in EER editorial boards (Top 10)}\\
\toprule
Rank & 1969-1973 & 1974-1978 & 1979-1983 & 1984-1988 & 1989-1993 & 1994-1998 & 2002-2002\\
\midrule
\endfirsthead
\caption[]{Share of countries in EER editorial boards (Top 10) \textit{(continued)}}\\
\toprule
Rank & 1969-1973 & 1974-1978 & 1979-1983 & 1984-1988 & 1989-1993 & 1994-1998 & 2002-2002\\
\midrule
\endhead

\endfoot
\bottomrule
\endlastfoot
\cellcolor{gray!6}{1} & \cellcolor{gray!6}{FRANCE (18.18\%)} & \cellcolor{gray!6}{FRANCE (20.31\%)} & \cellcolor{gray!6}{FRANCE (23.08\%)} & \cellcolor{gray!6}{FRANCE (20.86\%)} & \cellcolor{gray!6}{FRANCE (20.69\%)} & \cellcolor{gray!6}{UK (18.67\%)} & \cellcolor{gray!6}{FRANCE (18.92\%)}\\
2 & UK (12.99\%) & BELGIUM (12.5\%) & BELGIUM (15.38\%) & BELGIUM (14.11\%) & UK (19.31\%) & FRANCE (13.33\%) & USA (13.51\%)\\
\cellcolor{gray!6}{3} & \cellcolor{gray!6}{GERMANY (11.69\%)} & \cellcolor{gray!6}{UK (10.94\%)} & \cellcolor{gray!6}{ITALY (9.62\%)} & \cellcolor{gray!6}{ITALY (11.04\%)} & \cellcolor{gray!6}{NETHERLANDS (8.28\%)} & \cellcolor{gray!6}{USA (12\%)} & \cellcolor{gray!6}{SPAIN (8.11\%)}\\
4 & SWITZERLAND (11.04\%) & ITALY (8.98\%) & UK (7.69\%) & GERMANY (7.98\%) & NORWAY (7.59\%) & GERMANY (10\%) & ITALY (8.11\%)\\
\cellcolor{gray!6}{5} & \cellcolor{gray!6}{BELGIUM (9.09\%)} & \cellcolor{gray!6}{GERMANY (8.59\%)} & \cellcolor{gray!6}{GERMANY (5.77\%)} & \cellcolor{gray!6}{UK (7.36\%)} & \cellcolor{gray!6}{BELGIUM (6.9\%)} & \cellcolor{gray!6}{NETHERLANDS (9.33\%)} & \cellcolor{gray!6}{SWEDEN (8.11\%)}\\
\addlinespace
6 & NETHERLANDS (8.44\%) & NETHERLANDS (7.42\%) & NETHERLANDS (5.77\%) & SWEDEN (5.52\%) & SWEDEN (6.21\%) & ITALY (7.33\%) & GERMANY (8.11\%)\\
\cellcolor{gray!6}{7} & \cellcolor{gray!6}{ITALY (7.14\%)} & \cellcolor{gray!6}{SWITZERLAND (7.03\%)} & \cellcolor{gray!6}{SPAIN (5.77\%)} & \cellcolor{gray!6}{NETHERLANDS (5.52\%)} & \cellcolor{gray!6}{USA (4.83\%)} & \cellcolor{gray!6}{SPAIN (6.67\%)} & \cellcolor{gray!6}{NETHERLANDS (5.41\%)}\\
8 & HUNGARY (5.19\%) & HUNGARY (4.69\%) & HUNGARY (3.85\%) & SWITZERLAND (4.91\%) & GERMANY (4.83\%) & SWEDEN (4.67\%) & UK (5.41\%)\\
\cellcolor{gray!6}{9} & \cellcolor{gray!6}{DENMARK (3.25\%)} & \cellcolor{gray!6}{SPAIN (3.91\%)} & \cellcolor{gray!6}{SWITZERLAND (3.85\%)} & \cellcolor{gray!6}{NORWAY (4.29\%)} & \cellcolor{gray!6}{ITALY (4.14\%)} & \cellcolor{gray!6}{DENMARK (4\%)} & \cellcolor{gray!6}{SWITZERLAND (5.41\%)}\\
10 & LUXEMBURG (2.6\%) & AUSTRIA (3.91\%) & AUSTRIA (3.85\%) & ISRAEL (4.29\%) & SPAIN (3.45\%) & NORWAY (3.33\%) & ISRAEL (5.41\%)\\*
\end{longtable}
\endgroup{}

\begingroup\fontsize{3}{5}\selectfont

\begin{longtable}[t]{>{}c>{}l>{}l>{}l>{}l>{}l>{}l>{}l}
\caption{\label{tab:table-authors}Share of countries of authors' affiliations in EER publications (Top 10)}\\
\toprule
Rank & 1969-1973 & 1974-1978 & 1979-1983 & 1984-1988 & 1989-1993 & 1994-1998 & 1999-2002\\
\midrule
\endfirsthead
\caption[]{Share of countries of authors' affiliations in EER publications (Top 10) \textit{(continued)}}\\
\toprule
Rank & 1969-1973 & 1974-1978 & 1979-1983 & 1984-1988 & 1989-1993 & 1994-1998 & 1999-2002\\
\midrule
\endhead

\endfoot
\bottomrule
\endlastfoot
\cellcolor{gray!6}{1} & \cellcolor{gray!6}{USA (26.76\%)} & \cellcolor{gray!6}{USA (27.37\%)} & \cellcolor{gray!6}{USA (31.82\%)} & \cellcolor{gray!6}{USA (27.82\%)} & \cellcolor{gray!6}{USA (26.55\%)} & \cellcolor{gray!6}{USA (25.71\%)} & \cellcolor{gray!6}{UK (23.83\%)}\\
2 & BELGIUM (19.72\%) & BELGIUM (15.08\%) & UK (18.75\%) & UK (16.13\%) & UK (19.87\%) & UK (22.22\%) & USA (23.67\%)\\
\cellcolor{gray!6}{3} & \cellcolor{gray!6}{NETHERLANDS (14.08\%)} & \cellcolor{gray!6}{UK (11.73\%)} & \cellcolor{gray!6}{FRANCE (7.67\%)} & \cellcolor{gray!6}{FRANCE (8.47\%)} & \cellcolor{gray!6}{FRANCE (9.06\%)} & \cellcolor{gray!6}{FRANCE (9.95\%)} & \cellcolor{gray!6}{FRANCE (8.21\%)}\\
4 & FRANCE (8.45\%) & NETHERLANDS (8.38\%) & BELGIUM (7.39\%) & BELGIUM (7.46\%) & GERMANY (6.68\%) & BELGIUM (5.17\%) & GERMANY (5.48\%)\\
\cellcolor{gray!6}{5} & \cellcolor{gray!6}{HUNGARY (4.23\%)} & \cellcolor{gray!6}{NORWAY (5.03\%)} & \cellcolor{gray!6}{ISRAEL (7.39\%)} & \cellcolor{gray!6}{GERMANY (5.24\%)} & \cellcolor{gray!6}{BELGIUM (5.56\%)} & \cellcolor{gray!6}{ITALY (4.26\%)} & \cellcolor{gray!6}{ITALY (5.48\%)}\\
\addlinespace
6 & SWEDEN (4.23\%) & SWEDEN (5.03\%) & GERMANY (5.97\%) & NETHERLANDS (4.64\%) & CANADA (5.25\%) & GERMANY (4.26\%) & BELGIUM (4.67\%)\\
\cellcolor{gray!6}{7} & \cellcolor{gray!6}{IRELAND (2.82\%)} & \cellcolor{gray!6}{GREECE (5.03\%)} & \cellcolor{gray!6}{NETHERLANDS (4.55\%)} & \cellcolor{gray!6}{SWEDEN (4.23\%)} & \cellcolor{gray!6}{NETHERLANDS (3.82\%)} & \cellcolor{gray!6}{NETHERLANDS (4.01\%)} & \cellcolor{gray!6}{SPAIN (4.51\%)}\\
8 & UK (2.82\%) & ISRAEL (5.03\%) & AUSTRALIA (3.41\%) & CANADA (3.83\%) & ITALY (3.18\%) & SWEDEN (3.36\%) & SWEDEN (3.54\%)\\
\cellcolor{gray!6}{9} & \cellcolor{gray!6}{GERMANY (2.82\%)} & \cellcolor{gray!6}{CANADA (5.03\%)} & \cellcolor{gray!6}{CANADA (3.12\%)} & \cellcolor{gray!6}{ISRAEL (3.63\%)} & \cellcolor{gray!6}{SWITZERLAND (3.02\%)} & \cellcolor{gray!6}{CANADA (3.23\%)} & \cellcolor{gray!6}{SWITZERLAND (3.38\%)}\\
10 & CANADA (2.82\%) & FRANCE (2.23\%) & JAPAN (1.99\%) & ITALY (2.82\%) & ISRAEL (2.54\%) & SWITZERLAND (3.23\%) & NETHERLANDS (3.38\%)\\*
\end{longtable}
\endgroup{}

The EER was one of these crucial initiatives that contributed to a
Europeanisation of economics and the development of intellectual
exchanges between European based economists
(\protect\hyperlink{ref-goutsmedt2021}{Goutsmedt et al.,
2021}).\footnote{Our focus is on the geographical localisation of the
  institutions with which economists are affiliated, rather than their
  nationality (which is difficult to ascertain). Consequently,
  throughout the rest of the article, we will refer to ``European-based
  authors'' or ``US-based authors''. However, for the sake of brevity,
  we may occasionally employ more concise phrasing (such as ``Europeans
  economists'') to prevent the text from becoming overly cumbersome.}
The centrality of the journal was strengthened in 1984 when the European
Economic Association was created, and the EER was established as the
official journal of the new association.

\hypertarget{rising-journal}{%
\subsection{A rising european journal}\label{rising-journal}}

Besides offering a common platform for European economists, the journal
initial goal was also to encourage the promotion of a US-style approach
to economics. An important dimension of the journal's evolution was thus
the progressive integration of US-based economists. The ``International
Seminar on Macroeconomics,'' (ISoM) co-organized by the French
\emph{Ecole des Hautes Etudes en Sciences Sociales} and the US National
Bureau of Economic Research, played a key role in that integration of US
economists, as the conference papers were published each year in a
special issue (\protect\hyperlink{ref-goutsmedt2021}{Goutsmedt et al.,
2021}). The ISoM also contributed to the journal's major focus on
macroeconomics in the 1980s (Figure \ref{fig:plot-jel}).\footnote{The
  year 1979 in the data behind Figure \ref{fig:plot-jel} resulting from
  the confluence of two factors: fewer articles were published this year
  with only one volume against two in preceding and subsequent years;
  JEL codes were missing for most EER article this year.}

During the 1970s, the share of US-based authors publishing in the
journal grew steadily and reached a third of all affiliations in the
early 1980s (Figure \ref{fig:plot-authors}). The rising involvement of
US economists in the EER signified not just an increase in articles
published by US-based authors, but also a rise in collaborations between
US and European economists (Figure \ref{fig:plot-collabs}). Although
there were no collaborations in the journal's inaugural year, by 1980,
10 percent of the published articles featured joint writing between
institutions from the US and Europe.

\begin{figure}[h]

{\centering \includegraphics[width=1\linewidth]{../../../../../../../../Mon Drive/data/EER/pictures/Graphs/Intra_vs_US_collab} 

}

\caption{Patterns of collaboration between the United States and European countries in EER (smoothed using local polynomial regression)\protect\footnotemark}\label{fig:plot-collabs}
\end{figure}

\footnotetext{While the 'Europe Only', 'USA Only' and 'EU-US Collab' categories are computed on the overall corpus, the 'Intra EU Collab category' is computed on EER papers written by European authors ('Europe Only' category).}

In the mid-1980s, the journal thus emerged as a symbol of a more
integrated European economics, taking inspiration from the US standards
and enticing numerous US economists to contribute. Also, its
intellectual impact has seemingly broadened: the journal ascended as a
pre-eminent publication in macroeconomics, gradually surpassing other
prominent European journals regarding bibliographic citations (Figure
\ref{fig:plot-eer-importance-macro}).

\begin{figure}[h]

{\centering \includegraphics[width=1\linewidth]{../../../../../../../../Mon Drive/data/EER/pictures/Graphs/EER_importance_macro_minimal_bw} 

}

\caption{Share of citations from macroeconomics articles to the EER and the main european journals (smoothed using local polynomial regression)\protect\footnotemark}\label{fig:plot-eer-importance-macro}
\end{figure}

\footnotetext{We count only references to articles published after 1969, the inaugural year of the EER. We have also looked at the \textit{Scandinavian Journal of Economics}, \textit{Oxford Economic Papers}, and \textit{Weltwirtschaftliches Archiv}, which remain steadily under 0.5\%. Graphs including these journals can be found in the online appendix.}

The question that remains is whether this process of
internationalisation resulted in the complete standardisation of
European macroeconomics based on the US model, or if it allowed for the
cultivation and persistence of distinctly European specialities.

\hypertarget{methods}{%
\section{Methods for identifying European specialities}\label{methods}}

To identify European specialities, we compare macroeconomics articles
published in the EER and in the Top 5 journals (\emph{American Economic
Review}, \emph{Journal of Political Economy}, \emph{Econometrica},
\emph{Quarterly Journal of Economics}, \emph{Review of Economic
Studies}). Choosing a list of journals for a comparison is always, to a
certain extent, arbitrary. However, choosing the Top 5 journals offers
certain benefits. First, even though these journals may not have
consistently been regarded as the ``Top 5'' over our entire
investigation period, they have maintained their status as major
economic journals, where prominent macroeconomists used to publish.
Consequently, macroeconomic papers published in these journals are more
likely to represent a ``mainstream'' in macroeconomics, widely accepted
and disseminated.\footnote{Including more specialised journals would
  rather reveal emerging trends that have not yet become mainstream.}.
Second, these 5 journals are diverse enough to accommodate a broad range
of macroeconomic articles, such as highly theoretical or empirical
pieces. In sum, the Top 5 journals serve as a stable reference point
between 1969 and 2002 to draw comparisons with the EER, facilitating a
deeper understanding of EER publications' main
characteristics.\footnote{Incorporating more journals might have
  introduced the risks of adding noise, stemming from significant
  variations in editorship and primary features of some journals used
  for comparison.} Moreover, the EER was founded with the intention of
establishing an elite, leading journal for the European community that
would mimic the standards of the leading US journals. Therefore, the Top
5 seems a suitable benchmark for comparison with the EER.\footnote{It
  should be noted that we also examine the content published in
  macroeconomics in the EER independently of the comparison with the Top
  5, meaning the comparison is not the sole driver of our analysis.}

We identify macroeconomics articles by using the former and new JEL
codes classifications (\protect\hyperlink{ref-jel1991}{JEL,
1991}).\footnote{See the complete list of all the JEL codes we have used
  in \protect\hyperlink{eer-top5-macro}{Appendix B.1.} Using JEL codes
  as a classification method is more conventional and less arbitrary
  than creating our own system. This approach occasionally incorporates
  articles that may not seem overtly ``macroeconomic'', either because
  of misclassification or due to old JEL codes mixing subjects that were
  not ``macroeconomics'' in the new classification. However,
  bibliometric coupling and topic modelling isolate articles that
  deviate significantly from the ``average'' macroeconomic content or
  group them based on their similarities. Consequently, articles that do
  not directly pertain to macroeconomics can be set aside and do not
  affect our general results.} In addition to JEL codes data, we have
employed three distinct databases to collect various types of
information: outside of basic metadata (such year of publication, title,
authors, \emph{etc.}), we have collected bibliographic references of EER
and Top 5 articles, abstracts, and authors' affiliations.\footnote{Crossing
  databases has been necessary due to missing years and information in
  the different databases we have used (\emph{Web of Science},
  \emph{Scopus} and \emph{Microsoft Academic Premier}). See
  \protect\hyperlink{corpus}{Appendix B.1.} for more details on the
  building of our dataset.} Then, we have conducted two different types
of analysis to identify European specialities.

\hypertarget{bibliographic-coupling}{%
\subsection{Bibliographic coupling}\label{bibliographic-coupling}}

Bibliographic coupling connects articles together depending on the
bibliographic references they share. We build different relational
networks using EER and Top 5 articles (the nodes of the network),
connected by weighted links (the edges of the network), depending on the
number of references shared between two articles.\footnote{For more
  details on the measure of weights, see
  \protect\hyperlink{network}{Appendix B.3.}} To closely examine the
evolution of macroeconomic content, we build networks using a moving
eight-year window (based on the publication year of the articles). We
thus have 23 networks from the 1973-1980 period, through 1974-1981,
1975-1982, and so on, up to 1995-2002. For each network, we employ the
Leiden algorithm (\protect\hyperlink{ref-traag2019}{Traag et al., 2019})
to identify bibliographic clusters, i.e.~groups of articles sharing
numerous significant references with other articles of their cluster,
and fewer with articles outside their cluster. Articles within the same
cluster are more likely to share cognitive content (e.g., sharing
objects of study, methods, results or theory) even if disagreeing
(\protect\hyperlink{ref-claveau2016}{Claveau and Gingras, 2016};
\protect\hyperlink{ref-goutsmedt2021}{Goutsmedt et al., 2021};
\protect\hyperlink{ref-truc2021}{Truc et al., 2021}). Finally, we look
at the similarity of the clusters two by two for successive time
windows, and merge clusters from different windows together when they
are sufficiently close.\footnote{See
  \protect\hyperlink{network}{Appendix B.3.} for details on the merging
  criteria.}

This process allows us to obtain intertemporal clusters. Academic
articles predominantly cite references published in recent years.
Building a network spanning the entire period (1973-2002) would likely
lead to group articles based on their publication date rather than
shared intellectual content. By employing short time windows and then by
merging clusters from different windows, we circumvent this problem and
can identify clusters spanning longer periods of time, thus allowing for
a comprehensive historical analysis. We identify a total of 154
intertemporal clusters but only 33 that are \emph{(i)} present in at
least 2 networks (i.e.~2 time windows) and \emph{(ii)} represent more
than 0.04 percent of the nodes of at least one of the network they
belong.\footnote{We confine our investigation to these prominent
  intertemporal clusters.}

A range of indicators helps us to discern the content of these
intertemporal clusters---e.g.~the words used in abstracts and titles,
recurring authors, the most important nodes or the most cited
references.\footnote{This information is collected for each cluster in
  the online appendix ``Bibliographic information about the EER and
  details on the bibliographic coupling clusters''.} These indicators
informed our labeling of the clusters.

Then, for each intertemporal cluster, we identify the US or European
oriented nature of its publications and authors. We measured \emph{via}
the log of ratios the over/under representation of Europe- and US-based
authors in the cluster, and over/under representation of the EER and top
5 journals in the cluster.\footnote{Our assumption is that the content
  of articles published in the Top 5 by European economists could be
  more largely influenced by the standards of Top 5 journals and of US
  macroeconomics, and thus could be less representative of European
  economics than the articles published in the EER.} These two measures
inform us on which are the most `European' clusters, meaning those where
relatively more articles are published in the EER and by Europe-based
economists.\footnote{Table \ref{tab:summary-communities} lists the 33
  most significant clusters with the two measures.
  \protect\hyperlink{network}{Appendix B.3.} explains the method. Figure
  \ref{fig:plot-cluster-flow} displays the distribution of clusters over
  the different time windows and the flows of articles between the
  different clusters from one time window to another.} Figure
\ref{fig:plot-community-diff} displays the position of each cluster
relatively to these two measures.

\begin{figure}[h]

{\centering \includegraphics[width=1\linewidth]{../../../../../../../../Mon Drive/data/EER/pictures/Graphs/Communities_europeanisation_odds_windows_bw} 

}

\caption{The most European clusters (the size of the points captures the number of articles in the cluster)}\label{fig:plot-community-diff}
\end{figure}

\hypertarget{topic-modelling}{%
\subsection{Topic modelling}\label{topic-modelling}}

We apply topic modelling to our corpus, which is constituted of all the
titles and abstracts (when available) of the macroeconomic articles
published in the EER and Top 5. As a first step, we extract (or
`tokenise') ``ngrams'': unique words (or unigrams), bigrams and
trigrams. We exclude stop words and other uninformative words and
`lemmatise' remaining ngrams.\footnote{See the
  \protect\hyperlink{topic}{Appendix B.4.} for more details on the
  preprocessing steps we use.} Topic modelling is a unsupervised machine
learning method that identifies patterns and themes in the corpus by
clustering similar words and phrases into \(k\) ``topics''.
Specifically, for each topic, the method returns a set of probabilities
\(\beta\) for a ngram to be used in the topic. The \emph{ngrams} with
the highest \(\beta\) for a topic allows to understand what the topic is
about. Second, for each document in the corpus (i.e.~each article title
and abstract), the method returns a set of probabilities \(\gamma\) for
each topic to be mentioned in the document. The topics with the highest
\(\gamma\) for a document allows to understand what the document is
talking about. To estimate our topic model, we use a variant of the
Latent Dirichlet Allocation model called the Correlated Topic Model
(\protect\hyperlink{ref-blei2007}{Blei and Lafferty, 2007}). The number
of topics \emph{k} is chosen after evaluating different models
quantitatively and qualitatively. We choose to run the model with 50
topics, which allows us to better capture the diversity of topics
discussed in the corpus.\footnote{\protect\hyperlink{topic}{Appendix
  B.4.} gives more details on the different models we have tested and
  how we have set the number of topics.}

Table \ref{tab:summary-topics} lists all the topics identified with
their most representative words and expressions.\footnote{To select the
  most representative words for each topic, we use \emph{FREX} rather
  than \(\beta\) values (\protect\hyperlink{ref-bischof2012}{Bischof and
  Airoldi, 2012}). FREX is the weighted harmonic mean of the ngram's
  exclusivity and frequency. Exclusivity is a measure of how much a term
  is used in a topic compared to its frequency in others. By using FREX
  rather than \(\beta\) values, we can highlight the words that most
  strongly identify each topic, rather than highlighting more common
  words that may also appear in other topics.} We also use a set of
indicators, like the most cited references per topic crossed with the
journal and affiliation variables, to get a better picture of what the
topics are and what are their European and non-European
dimensions.\footnote{This information is collected in the online
  Appendix ``Details on the topics''.}

Similarly to bibliometric coupling, we are interested in the topics
characteristics regarding the publications (EER \emph{vs.} Top 5) and
the countries of affiliations of the authors (the US \emph{vs.} European
countries). For each topic, we only keep the articles that are the most
associated to the topic, with \(\gamma > 0.1\). We then assess the
over/under representation of EER articles for each topic, by calculating
the log odds ratio in comparison to top 5 articles:

\[
\text{log odds ratio EER/Top5 for topic }i = ln(\frac{[\frac{N_{\text{EER in topic }i}+1}{N_{\text{topic }i} + 1}]_{EER}}{[\frac{N_{\text{Top5 in topic }i}+1}{N_{\text{topic }i} + 1}]_{Top5}})
\]

with \(N_{\text{EER in topic }i}\) the number of EER articles in the
topic and \(N_{\text{topic }i}\) the total number of articles in the
topic. We perform the same calculation for articles written exclusively
by Europe-based authors or US-based authors. The two log odds ratios are
plotted in a two-dimensional graph (Figure \ref{fig:plot-topic-diff}),
which allows us to observe which topics are over-represented in the EER
and more frequently mentioned by Europe-based authors.\footnote{Refer to
  \protect\hyperlink{topic}{Appendix B.4.} for details on the measure
  and the results using an alternative measure that does not employ
  \(\gamma\) as a threshold.}

\begin{figure}[h]

{\centering \includegraphics[width=1\linewidth]{../../../../../../../../Mon Drive/data/EER/pictures/Graphs/logratio_diff_plot_new_bw} 

}

\caption{The most European topics (the size of nodes capture the topics prevalence)}\label{fig:plot-topic-diff}
\end{figure}

\bigskip

The combination of two distinct unsupervised methods enables us to
systematically identify potential European specialities.\footnote{To our
  knowledge, this article represents the first attempt to combine both
  methods to describe the evolution of the state of economics. Utilising
  both approaches is essential to ensure the robustness of our results
  and systematically verify if a distinctly ``European'' cluster can be
  associated with one or several ``European'' topics, and conversely.}
The in-depth qualitative analysis of clusters and topics then provide a
comprehensive overview of the various issues, methods and theoretical
questions investigated by European macroeconomists, in comparison to US
macroeconomics. In the final two sections, we thus explore the evolving
characteristics of European macroeconomics in the EER, in comparison to
macroeconomics in Top 5 journals, published by US-based economists. Two
main periods (1970s to mid-1980s; mid-1980s to late 1990s) may be
discerned.

\hypertarget{the-1970s-to-mid-1980s-a-progressive-europeanisation-in-opposition-to-us-macroeconomics}{%
\section{The 1970s to mid-1980s: A progressive Europeanisation in
opposition to US
macroeconomics}\label{the-1970s-to-mid-1980s-a-progressive-europeanisation-in-opposition-to-us-macroeconomics}}

While the literature in the history of macroeconomics insist on the
transformations of macroeconomics following debates on microfoundations,
and the rise of the microfoundational program of new classical
economics, European macroeconomics have followed a different path.
European macroeconomists stay far from most of the theoretical debates
triggered by microfoundations, whether on the Rational expectations and
the Phillips curve, the demand for money, or the consumption function.
Rather, they developed more empirical studies, but also their own
microfoundational program.

\hypertarget{opposition-to-new-trends-in-us-macroeconomics-opposition-trends}{%
\subsection{Opposition to New Trends in US macroeconomics
(\#opposition-trends)}\label{opposition-to-new-trends-in-us-macroeconomics-opposition-trends}}

The analysis of topics and clusters allows us to understand what
European macroeconomics \emph{was not} in this period. Several
literatures seem to be ignored by the Europeans in the EER. That is the
case of the debates around the life-cycle and permanent income
hypotheses, influenced by Friedman
(\protect\hyperlink{ref-friedman1957}{1957}) and Hall
(\protect\hyperlink{ref-hall1978b}{1978}).\footnote{See clusters
  ``Intergenerational model, Savings \& Consumption'' and ``Permanent
  Income and Life-Cycle Hypotheses'', as well as topics 12 and 14.} This
literature sought to introduce structural heterogenity in life cycle or
permanent income models (see Cherrier, Duarte and Saïdi, this issue).

Other more US-oriented areas were research about \emph{(i)} the demand
for money---for which Baumol (\protect\hyperlink{ref-baumol1952}{1952})
and Friedman and Schwartz (\protect\hyperlink{ref-friedman1963}{1963})
were central references---as well as \emph{(ii)} the ``new classical
monetary theory'' (\protect\hyperlink{ref-hoover1988}{Hoover, 1988,
chap. 6}) of the 1970s inspired by Sargent's, Bryant's and Wallace's
works (\protect\hyperlink{ref-bryant1979}{Bryant and Wallace, 1979};
\protect\hyperlink{ref-sargent1982}{Sargent and Wallace,
1982}).\footnote{See clusters ``Monetary Economics \& Demand for Money''
  and, even if it is not as ``non-European'' as the first one, the
  cluster ``Demand for Money''. For topics, see topic 2 on the demand
  for money and money supply, which is one of the most non-European
  topic, but also topic 19 on demand for money and term structure of
  interest rates, influenced notably by Fama
  (\protect\hyperlink{ref-fama1975}{1975}). More recently, the ``New
  Monetarist Economics'' of Kiyotaki and Wright
  (\protect\hyperlink{ref-kiyotaki1989}{1989}), Kiyotaki and Wright
  (\protect\hyperlink{ref-kiyotaki1993}{1993}), and Trejos and Wright
  (\protect\hyperlink{ref-trejos1995}{1995}) has been also ignored by
  Europeans (see Frasser (\protect\hyperlink{ref-frasser2020}{2020,
  chap. 2}), for a historical reconstruction of this literature.} The
new classical monetary theory of the 1970s is described by Hoover
(\protect\hyperlink{ref-hoover1988}{1988, p. 111}) as the research for
``microfoundations for the theory of money consistent with general
equilibrium and individual optimization'' promoted by new classical
economists (Lucas, Sargent, Barro, Kydland, Prescott, etc.).

More generally, it appears that the works of new classical economists
that contributed to reshaping macroeconomics in the late 1970s and early
1980s, and that are so central in many histories of macroeconomics
(\protect\hyperlink{ref-devroey2016}{De Vroey, 2016};
\protect\hyperlink{ref-snowdon2005}{Snowdon and Vane, 2005}), were less
influential in Europe at the time. Articles like Lucas
(\protect\hyperlink{ref-lucas1972}{1972}), Lucas
(\protect\hyperlink{ref-lucas1973}{1973}), Sargent and Wallace
(\protect\hyperlink{ref-sargent1975}{1975}) or Barro
(\protect\hyperlink{ref-barro1976}{1976}) were constantly under-cited by
Europe-based macroeconomists in comparison to US economists in the 1970s
and 1980s (see Figure \ref{fig:plot-new-classical}).\footnote{We have to
  wait the 1982-1988 window to see some new classical contributions
  cited as much by Europeans as by US economists. The integration of
  these contributions obviously took some time in Europe and lagged
  behind the US.} This is consistent with the fact that European
macroeconomists favoured in the late 1970s and early 1980s an
alternative ``microfoundational programme''
(\protect\hyperlink{ref-hoover2012}{Hoover, 2012}): disequilibrium
theory (see Section \ref{disequilibrium}).

\begin{figure}[h]

{\centering \includegraphics[width=1\linewidth]{../../../../../../../../Mon Drive/data/EER/pictures/Graphs/g2_over_citations_bw} 

}

\caption{Citation of new classical works by European economists relatively to US-based economists (log of ratios on 7-year moving average)}\label{fig:plot-new-classical}
\end{figure}

But this rejection of new theoretical research in line with the search
for microfoundations, and coming from the US, may also be linked to the
fact that European macroeconomics as it appeared in the EER, was
relatively more oriented toward applied and empirical contributions.

\hypertarget{the-importance-of-applied-and-empirical-works}{%
\subsection{The importance of applied and empirical
works}\label{the-importance-of-applied-and-empirical-works}}

That is not easy to assess some major trends in Europeans publications
in the EER in the first years, neither to observe convergence between
European macroeconomists. Indeed, as fewer papers were published in the
first issues and we may also imagine that the editorial line was
fluctuating. However, a first characteristic emerges: the propensity to
publish applied and empirical works. For instance, in the 1970s issues,
we find contributions providing empirical analysis of Dutch pensions
funds, UK income distribution, Belgium wage variations, using principal
components analysis, building indicators of capacity utilization or
larger macroeconometric models (see Table X).

As an emerging journal, the EER was still less likely in the mid-1970s
to publish very influential contributions. But some of the early
contributions belong to some already well established literature and
connect with two important research programs that influenced these
contributions in the ERR.\footnote{These contributions are grouped in
  topic 46 and cluster ``Modeling Consumption \& Production'', which
  partially overlap.} These two research programs were at the juncture
of econometrics and macroeconomics.

The first one is the ``LSE approach'' of econometrics, under the
leadership of David Hendry, involving notably the work of James
Davidson, Grayham Mizon and Neil Ericsson
(\protect\hyperlink{ref-qin2013a}{Qin, 2013, chap. 4}). A major goal of
the LSE approach and of Hendry was to build some bridges between the
Cowles Commission structural approach of econometrics and the
Box-Jenkins time-series approach. Heavy emphasis was put on the search
for appropriate model specification, to design the best representation
of the true ``data generation process''.

A large part of the LSE approach work was devoting to the modelling of
consumption. Davidson and Hendry, accompanied by two co-authors, tackles
consumption modelling by inspecting the existing ``plethora of
substantially different quarterly regression equations'' and criticised
the ``proliferation of non-tested models''
(\protect\hyperlink{ref-davidson1978}{Davidson et al., 1978, pp.
661--662}). This article was the occasion for them to highlight the
major principles of the LSE approach, and notably to illustrate the use
of error-correction (\protect\hyperlink{ref-qin2013a}{Qin, 2013, pp.
63--64}). During the third edition of the ISoM, published in 1981 in the
EER, they came back to the chosen equations of Davidson et al.
(\protect\hyperlink{ref-davidson1978}{1978}) to test the integration of
the permanent income/life cycle hypothesis in Hall's -Hall
(\protect\hyperlink{ref-hall1978b}{1978}) formulation
(\protect\hyperlink{ref-davidson1981}{Davidson and Hendry, 1981}).

Econometric research on demand and consumption did not limit to the LSE
in the UK. Angus Deaton's and John Muellbauer's contributions also
appear to have had impact on research published in the EER. Their focus
was on the aggregation problem and the weaknesses of a ``representative
consumer'' (see Cherrier et al., this issue). In 1980 in the
\emph{American Economic Review}, they proposed a new system of demand
equations, modelling the relationship between consumer demand, relative
prices and income, that they called the ``Almost Ideal Demand System''
(\protect\hyperlink{ref-deaton1980}{Deaton and Muellbauer, 1980}). They
claimed the superiority of this model to competitor like the translog
model, or the Rotterdam model developed by Henri Theil and Anton Barten.

The research around the Rotterdam model constituted a second research
program that appeared central in the 1970s for the European
contributions published in the EER. Theil had been appointed in 1953 as
Professor of Econometrics at the Netherlands School of Economics in
Rotterdam, to take the succession of Jan Tinbergen
(\protect\hyperlink{ref-kloek2001}{Kloek, 2001}). In 1966, he eventually
moved to Chicago, one year after having published ``The Information
Approach to Demand Analysis'' (\protect\hyperlink{ref-theil1965}{Theil,
1965}), an essential pillar of the Rotterdam model, modelling the role
played by marginal budget shares
(\protect\hyperlink{ref-clements2015a}{Clements and Gao, 2015}). The
other essential contribution was the one of Barten
(\protect\hyperlink{ref-barten1964}{1964}).

Barten had indeed started a Ph.D.~thesis on consumer behavior under the
supervision of Theil in Rotterdam
(\protect\hyperlink{ref-kuleuven2016}{Leuven, 2016}). Barten was hired
in 1966 by KU Leuven to join the just-born CORE. His most cited paper is
the very first article published in the EER
(\protect\hyperlink{ref-barten1969}{Barten, 1969}): Barten developed a
new estimation procedure of a demand equations system that would gather
a lot of attention in the first years of the EER.\footnote{Barten
  tackled broader project than demand equations. With other CORE
  economists, he contributed to develop a macroeconometric model of the
  European Economic Community (EEC), which was presented in the EER
  (\protect\hyperlink{ref-barten1976}{Barten et al., 1976}). The next
  year, André Dramais from the \emph{Université Libre de Bruxelles}
  presented empirical results on inflation transmission between EEC
  members, using a competing macroeconometric model, DESMOS. Both models
  constituted crucial steps in the development of macroeconometric model
  within the European Commission (see Acosta et al., this issue).} The
Rotterdam model recurrently appeared in EER's publications, somewhat
culminating in 1984 when two synthesis articles were devoted to it
(\protect\hyperlink{ref-barnett1984}{Barnett, 1984};
\protect\hyperlink{ref-byron1984}{Byron, 1984}).

These two lines of research (the LSE approach and the Rotterdam model)
as well as the contributions of researchers gravitating around them,
appear as original and distinguished from the developments of
macroeconomics occurring on the other side of the Atlantic.\footnote{This
  is even more true for the British side, that devoted time to criticise
  many new trends in the US. For instance, Deaton and Muellbauer
  attacked the search for theoretical microfoundations to
  macroeconomics, which ignored aggregation issues (Cherrier et al.,
  this issue).} Nonetheless, we cannot speak of European specialities
are these research lines were very nationally anchored (or just connect
Belgium and the Netherlands in the second case) and did not yet
participate to a true process of Europeanisation. Some transnational
connections existed. For instance, Hendry visited the CORE during
1980-1981, at the time where Angus Deaton was also there
(\protect\hyperlink{ref-ericsson2004}{Ericsson, 2004}). Hendry started
to work with Jean-François Richard, a CORE econometrician, and Robert
Engle, then at University of California San Diego. The resulting article
developed the concept of ``strong exogeneity'' and ``super exogeneity''
(\protect\hyperlink{ref-engle1983}{Engle et al., 1983}).

The LSE approach and Hendry had a delayed impact on the development of
European specialities. The cointegration concept
(\protect\hyperlink{ref-engle1987}{Engle and Granger, 1987}) had some
roots in the LSE approach and in Granger's and Engle's discussion with
Hendry (\protect\hyperlink{ref-diebold2003}{Diebold, 2003, pp.
1173--1174;@qin2013a, 68}). Both the LSE approach and Granger and
Engle's contributions, as well as the work of University of Copenhagen
econometrician, Soren Johansen
(\protect\hyperlink{ref-johansen1988}{Johansen, 1988}), would deeply
influence research published by European macroeconomists in the EER
after the mid-1980s.\footnote{See Topic 46 and Community ``Business
  Cycles, Cointegration \& Trends''.} Many European macroeconomists
engaged in the 1990s in empirical research on business cycles,
detrending methods and shocks identifications. The German Horst Entorf,
the Italians Marco Lippi and Lucrezia Reichlin, the Spanish Juan José
Dolado, or the LSE economist Dany Quah were essential characters
\footnote{The French Olivier Blanchard was of course another important
  researcher in this domain, but was working in the US during the
  period.}

However, before this period, the true European speciality and most
unifying line of research from the 1970s to the mid-1980s was the
disequilibrium theory.

\hypertarget{disequilibrium}{%
\subsection{Disequilibrium theory as a landmark for European
macroeconomics}\label{disequilibrium}}

Research around the disequilibrium theory represents a significant yet
often overlooked milestone in the history of macroeconomics
(\protect\hyperlink{ref-backhouseboianovski2013}{Backhouse and
Boianovski, 2013}; \protect\hyperlink{ref-plassard2021}{Plassard et al.,
2021}). It played a crucial role in the 1970s in rekindling interest in
the research finding appropriate microfoundations for macroeconomics in
the 1970s.\footnote{See Duarte and Lima
  (\protect\hyperlink{ref-duartelima2012a}{2012}) for a history of
  microfoundations in macroeconomics.} Rooted in the tradition of the
general equilibrium theory and influenced by the work of Patinkin,
Clower and Leijonhufvud, disequilibrium theory investigated the effects
of non-walrasian price-setting (i.e.~without \emph{tâtonnement}),
fixed-price and quantity rationing on macroeconomic outcomes. This
approach provided an alternative to the new classical contributions and
the Lucas and Sargent's ``representative-agent microfoundational
program'' (\protect\hyperlink{ref-hoover2012}{Hoover, 2012}), which many
proponents of disequilibrium research explicitly rejected
(\protect\hyperlink{ref-renault2020a}{Renault, 2020}). Although Barro
and Grossman's (\protect\hyperlink{ref-barro1971}{1971}) article helped
popularising disequilibrium macroeconomics, this research line was
deeply anchored in France and Belgium, and it continued to expand
throughout the 1970s and 1980s (Plassard and Renault, this issue).

Bibliometric analysis reveals that the ``Disequilibrium and Keynesian
economics'' cluster constituted the most significant cluster closely
associated with the EER and fostered by Europe-based economists. This
cluster also encompasses other ``alternative research lines'' to Lucas
and Sargent's research program (\protect\hyperlink{ref-devroey2016}{De
Vroey, 2016, chap. 14}), such as Azariadis's
(\protect\hyperlink{ref-azariadis1975}{1975}) implicit contract model,
Hart's (\protect\hyperlink{ref-hart1982}{1982}) imperfect competition
model or Diamond's (\protect\hyperlink{ref-diamond1982}{1982}) search
model. This testifies that in the late 1970s and in the 1980s,
connections existed between US and European macroeconomists regarding
the renewal of theoretical macroeconomics and the search for
microfoundations, as well as the opposition to new classical
macroeconomics. \footnote{As a complement, topic modelling demonstrates
  the extent to which disequilibrium theory pervaded European
  macroeconomics in the 1980s, unifying the treatment of diverse
  macroeconomic issues (see notably Topic 25, Topic 39, and Topic 11.}

First of all, part of the literature ``arose out of the internal
problems within general equilibrium theory''
(\protect\hyperlink{ref-backhouseboianovski2013}{Backhouse and
Boianovski, 2013, p. 105}), notably the need to break with
\emph{tâtonnement} and to build general equilibrium model with agents
setting prices in the model. Disequilibrium macroeconomics thus
contributed to the persistence of a lively research program centred on
general equilibrium theory issues.\footnote{See topic 11.} Moreover,
disequilibrium macroeconomics also emerged as a pivotal framework for
accounting for the 1970s stagflation
(\protect\hyperlink{ref-backhouseboianovski2013}{Backhouse and
Boianovski, 2013, chap. 8}). Malinvaud's \emph{Theory of Unemployment
Reconsidered} (\protect\hyperlink{ref-malinvaud1977}{1977}) marked a
decisive development in this direction by opposing ``Keynesian
unemployment'', which resulted from excess supply in both goods and
labour markets, and ``classical unemployment'', caused by excess demand
for goods but excess supply in the labour market---implying excessively
high real wages.\footnote{Malinvaud
  (\protect\hyperlink{ref-malinvaud1977}{1977}) proposed a third regime,
  ``repressed inflation'', resulting from excess demand on both markets.}
The 1973 oil shock and the concurrent decline in productivity would
explain the rise of a classical unemployment in the 1970s. Consequently,
the primary concern for proponents of the three-regime approach became
determining the extent to which European unemployment could be
attributed to either Keynesian or classical unemployment.

This framework for understanding unemployment and stagflation was
featured, discussed, or at least mentioned in numerous influential works
in European macroeconomics in the 1980s.\footnote{``Influential'' means
  here highly cited by European economists in one or several clusters or
  topics.} In the EER, Drèze and Modigliani examined the ``current state
of underemployment in Belgium'' by analysing the trade off between real
wages and employment in the context of a small open economy
(\protect\hyperlink{ref-dreze1981}{Drèze and Modigliani, 1981, p. 2}).
They combined the possibility of classical unemployment, inspired by
Malinvaud (\protect\hyperlink{ref-malinvaud1977}{1977}), with Modigliani
and Padoa-Schioppa's argument that, ``in an open economy, external
balance implies a constraining relationship between the levels of real
wages and employment'' (\emph{ibid.}). In the early 1980s, Malinvaud's
framework was also connected to the reccurring EER debate surrounding
the ``wage gap''. Question was to determine whether real wages were too
high (indicating a positive wage gap). Bruno and Sachs were central
figures in this debate and explicitly relied on Malinvaud's
framework.\footnote{Bruno and Sachs's
  (\protect\hyperlink{ref-brunosachs1985}{1985}) book, \emph{Economics
  of Worldwide Stagflation}, provided a synthesis of their late 1970s
  and early 1980s works, and served as a highly cited resource for
  European macroeconomists in the 1980s (see also
  \protect\hyperlink{ref-goutsmedt2021}{Goutsmedt et al., 2021, sec.
  3}).}

Outside of unemployment and stagflation, disequilibrium theory was also
extended to other macroeconomic issues. For instance, Avinash Dixit,
when at University of Warwick, extended Clower's
(\protect\hyperlink{ref-clower1965}{1965}) dual decision hypothesis and
Malinvaud's framework to international trade theory
(\protect\hyperlink{ref-dixit1978}{Dixit, 1978, p. 393}). According to
Dixit, this provided the foundation for a ``more satisfactory model of
the balance of trade'' than Frenkel and Johnson's
(\protect\hyperlink{ref-frenkel1976}{1976}) monetary approach, which
``assumes instantaneous attainment of Walrasian equilibrium in commodity
and labour markets'' (\protect\hyperlink{ref-dixit1978}{Dixit, 1978, p.
393}). Dixit's model would later form the basis for parts of his joint
book with Victor Norman, from the Norwegian School of Economics and
Business Administration, on the \emph{Theory of International Trade}
(\protect\hyperlink{ref-dixit1980}{Dixit and Norman, 1980}). The book
representend a pivotal reference for European economists working on
international trade.\footnote{See topic 39.}

This centrality of disequilibrium macroeconomics in European
macroeconomics is also evident in the extent that European
macroeconomists had to position themselves relative to it, particularly
to the Keynesian versus classical unemployment framework. In May 1985
was held a conference in Sussex about European unemployment, published
in \emph{Economica} the next year.\footnote{On this episode, see
  Backhouse, Forder and Laskaridis, as well as Plassard and Renault,
  both in this issue.} Macroeconomists from different countries
presented their analyses of European national unemployment rates. While
Sneessens and Drèze estimated a ``two-market macroeconomic rationing (or
disequilibirum) model of the economy
(\protect\hyperlink{ref-sneessens1986}{Sneessens and Drèze, 1986, p.
S97}), Malinvaud (\protect\hyperlink{ref-malinvaud1986}{1986}) proposed
a more descriptive analysis to explain the rise of unemployment in
France, although he acknowledged proximities with Sneessens and Drèze
formalisation in the same issue. Malinvaud discussed determinants
of''the classical component of unemployment''
(\protect\hyperlink{ref-malinvaud1986}{Malinvaud, 1986, p. S216}), but
also criticised the use of Phillips curve with a non-accelerating
inflation rate of unemployment (NAIRU) to account for the causes of
unemployment. In contrast, NAIRU was central to the model proposed by
Layard and Nickell to discuss British unemployment. They argued that the
``labour demand function that {[}they{]} use cuts through the fruitless
debate now raging (especially in Europe) as to whether current
unemployment is `classical' or `Keynesian'\,''
(\protect\hyperlink{ref-layard1986}{Layard and Nickell, 1986, p. S121}).

While not completely consensual, disequilibrium theory and the
classical/Keynesian unemployment distinction were unavoidable in the
mid-1980s. They shaped the treatment of various macroeconomic issues
such as international trade, inflation, unemployment, wage-setting, and
spurred new theoretical developments, refinements existing models, as
well as econometric innovations
(\protect\hyperlink{ref-renault2019}{Renault, 2022b}). Furthermore, this
line of research was embraced by macroeconomists across various European
countries and fostered transnational collaborations. In line with the
broader disregard or rejection of new trends in US macroeconomics by
many Europe-based macroeconomists (Section \ref{opposition-trends}),
disequilibrium macroeconomics brought forward an alternative theoretical
framework to new classical macroeconomics that was emerging in the US
during the same period.

\hypertarget{the-mid-1980s-to-the-late-1990s-remaining-specialities-despite-theoretical-convergence}{%
\section{The Mid-1980s to the Late 1990s: Remaining Specialities despite
Theoretical
Convergence}\label{the-mid-1980s-to-the-late-1990s-remaining-specialities-despite-theoretical-convergence}}

The relationship of European macroeconomics with its US counterpart, as
well as its connection to new classical economics, gradually transformed
after the mid-1980s. This shift was likely due to the dwindling dynamism
of research surrounding disequilibrium theory. In the late 1980s,
disequilibrium theory had lost its ability to build bridges between
European macroeconomists and no longer served as a unifying theoretical
language. Instead, what brought European macroeconomists together in the
1990s was not a common theoretical framework, but rather distinctly
European matters (such as high unemployment or European integration), as
well as a new approach to address many macroeconomic issues: political
economy. This later trend served as the belated entry point for new
classical economics to eventually influence European macroeconomics, and
this was achieved not through Lucas and Sargent's works, but through the
time consistency literature based on the contributions of Kydland,
Prescott and Barro.

\hypertarget{the-fall-of-disequilibrium}{%
\subsection{The Fall of
Disequilibrium}\label{the-fall-of-disequilibrium}}

After the mid-1980s, publications about disequilibrium continued to pop
out occasionally in the EER. But in quantitative terms, we observe a
decrease of disequilibrium importance both through the bibliometric and
topic modelling analyses.\footnote{See the destiny of the disequilibrium
  cluster and of topic 11 in Figure \ref{fig:plot-cluster-flow} and
  Figure \ref{fig:plot-topic-year}.} We can also observe that indirectly
in topic 25 on real wages and employment: while Malinvaud
(\protect\hyperlink{ref-malinvaud1977}{1977}) was an important reference
for the older article of the topic, it disappeared from the bibliography
of the most recent articles.

Part of the research program on disequilibrium seems to have persist in
the 1990s through its most theoretical part and developed closer links
with the literature on coordination and sunspots.\footnote{See cluster
  on ``Coordination \& Sunspots 2'' and Figure
  \ref{fig:plot-cluster-flow}}. This cluster was only slightly
over-represented by European economists, but gathered articles mainly
published in the Top 5, so it does not really constitute a European
speciality.{]}

The fall of the disequilibrium program is also visible in the place of
the Keynesian/classical unemployment distinction. For instance, we
observe that when using the insider-outsider opposition to discuss
European unemployment in 1987, Nils Gottfries and Henrik Horn still
referred to the Keynesian/classical opposition and argued in their paper
that ``the present unemployment may originally have arisen for Keynesian
reasons, but once unemployment is created it will change the conditions
under which wages are formed, thus persisting in a classical form''
(\protect\hyperlink{ref-gottfries1987}{Gottfries and Horn, 1987, p.
2}).\footnote{See Section \ref{unemployment} below for analyses on the
  various treatments of the unemployment issue by European
  macroeconomists.} Similarly, Assar Lindbeck and Denis Snower cited
Malinvaud (\protect\hyperlink{ref-malinvaud1977}{1977}) and the
``boundary between the `Keynesian' and `Classical' regimes''
(\protect\hyperlink{ref-lindbeck1987a}{Lindbeck and Snower, 1987, p.
408}). This reference to Malinvaud's framework disappear in the
following years in similar works (as in
\protect\hyperlink{ref-gottfries1992}{Gottfries, 1992}, for instance).
More generally, the reference to the classical versus Keynesian
unemployment was most of the time missing in the large literature that
developed after the mid-1980s to understand the problem of large
unemployment in Europe.

\hypertarget{unemployment}{%
\subsection{New Approaches to Deal with European
Unemployment}\label{unemployment}}

Unemployment became a major object of study for European macroeconomists
after the mid-1980s and the EER published plethora of articles on the
subject.\footnote{See Topic 37 and Community ``Theory of Unemployment
  and Job Dynamics''} With the stagflation and the following
disinflation of the early 1980s, unemployment has risen at rare levels
in European countries. However, once inflation stabilised, unemployment
did not decrease to go back to levels relatively similar to the early
1970s, as it has been the case in the US. This macroeconomic situation
was coined the ``Eurosclerosis'', which raised a new puzzle to solve for
macroeconomists.

This puzzle was the starting point of numerous contributions by
Europe-based economists. It fostered the Europeanisation of
macroeconomics by pushing forward comparisons between European
countries, to identify common features that would explain European high
unemployment. This was happening notably through the organisation of
conferences, like the May 1985 conference in Sussex, published in
\emph{Economica} the next year (Backhouse et al., this issue).
Macroeconomists also proposed various empirical studies using data of
various OECD countries to understand Europe peculiarities.

That has been the case in Bruno and Sachs'
(\protect\hyperlink{ref-brunosachs1985}{1985}) book, which analysed the
differences in terms of nominal and real wage rigidities between the US
and Europe. The higher unemployment cost of the stagflation in Europe
would be the result of larger real rigidities. Bruno and Sachs approach
of the unemployment issue, that was developed in a series of articles
since 1979 and culminated in their book, remained influential for
European economists during the 1980s. It constituted a point of
departure of another important line of research developed by LSE
economists Richard Layard, Richard Jackman, Stephen Nickell, and their
various co-authors.

In a conference held in Cambridge in July 1981, Grubb et al.
(\protect\hyperlink{ref-grubb1982}{1982}) proposed an explanation of the
``Causes of the Current Stagflation'' through a comparison of 19 OECD
countries (see also Backhouse et al., this issue). If they want in the
direction of too high real wages in Europe, they consider them as ``a
consequence rather than a prime cause of the difficulty''
(\protect\hyperlink{ref-grubb1982}{Grubb et al., 1982, p. 707}). High
real wages were the symptom of a rise in the NAIRU, caused by an
increase in the relative prices of raw materials and a sharp fall in the
rate of productivity growth. Grubb, Jackman and Layard pursued this
analysis of real and nominal rigidities in OECD countries the next year
in the ISoM in Mannheim, in a paper then published in the EER
(\protect\hyperlink{ref-grubb1983a}{Grubb et al., 1983}). During a
decade, the LSE team will provided a series of articles with different
specifications and estimations of European unemployment. This culminated
in 1991 with Layard, Nickell and Jackman's book, ``Unemployment:
Macroeconomic Performance and the Labour Market'', which proposed an
extensive analysis of unemployment in the OECD, relying on a theoretical
framework of inflation dynamics and wage bargaining
(\protect\hyperlink{ref-layard1991a}{Layard et al., 1991}). Three years
later, Charles Bean, also at the LSE, and who had collaborated at
different occasion with Layard and Nickell, published two surveys of the
treatment of European unemployment, in the \emph{Journal of Economic
Literature} and the EER (\protect\hyperlink{ref-bean1994}{Bean, 1994a},
\protect\hyperlink{ref-bean1994a}{1994b}).

Beyond comparing macroeconomic data of different European countries,
European macroeconomists also adopted more institutional comparisons,
looking at differences in wage bargaining organisation, labor market
regulations, etc. A central study cited in EER articles was Lars
Calmfors (University of Stockholm) and John Driffil (University of
Southampton) analysis of European countries corporatism
(\protect\hyperlink{ref-calmfors1988}{Calmfors and Driffill, 1988}).
They estimated the impact on wages and unemployment of different
measures of corporatism offered by economists or political scientists.
Calmfors and Driffil argue that countries with low centralisation of
wage bargaining (like the US) and with high centralisation (the Nordic
Countries) would do better in terms of employment than middle
centralisation countries (most countries of the European Community at
the time).

In a similar vein, the insider and outsider approach of the labour
market also gained some popularity in Europe. The approach was developed
by University of Stockholm economists Nils Gottfries, Henrik Horn, Assar
Lindbeck (all from the University of Stockholm), in collaboration with
Denis Snower from Birkbeck College
(\protect\hyperlink{ref-gottfries1992}{Gottfries, 1992};
\protect\hyperlink{ref-gottfries1987}{Gottfries and Horn, 1987};
\protect\hyperlink{ref-lindbeck1987a}{Lindbeck and Snower, 1987},
\protect\hyperlink{ref-lindbeck1986}{1986}).{]}. The insider-outsider
literature argues that the insiders (the employed or the unionised
workers) have the lead on wage-setting. Employment being determined in
function of wages, too high wages maintain outsiders excluded from the
labour market. In the case of Europe after the stagflation, it means
that an increase in labour demand would rather lead to rising wages
without rising employment.

The interest for explaining the high European unemployment did not limit
to UK and Swedish macroeconomists. Another influential path of research
were notably developed by the Spanish Samuel Bentolila and the French
Gilles Saint-Paul (EHESS), in collaboration with Giuseppe Bertola from
Princeton (\protect\hyperlink{ref-bentolila1990}{Bentolila and Bertola,
1990}; \protect\hyperlink{ref-bentolila1992a}{Bentolila and Saint-Paul,
1992}; \protect\hyperlink{ref-bertola1990a}{Bertola, 1990}). They
developed a series of articles studying the role that job security and
firing costs played in the stock of unemployment and its variation.

All these lines of research were central for European macroeconomists
publishing in the EER, as well as they constitute a true specificity in
comparison to the US. They spread across different European countries
and fostered transnational collaborations, even if the UK appears as a
more dynamic producer of contributions on European unemployment. Another
approach, also partly centred in the UK and the LSE, was quite
influential in the 1990s: the literature on search and matching models,
developed by Dale Mortensen from Northwestern University and Christopher
Pissarides from the LSE (\protect\hyperlink{ref-mortensen1994}{Mortensen
and Pissarides, 1994}). Mortensen and Pissarides' theoretical framework
contributed to shape the orientation of the research agenda in Europe,
whether on job flows, firing costs, skills and geographical mismatching.
However, this approach appears to have found some echo among US
macroeconomists and Top 5 journals, and does not limit to European
macroeconomics.

\hypertarget{political-economics}{%
\subsection{A new unifying language: political
economy}\label{political-economics}}

European macroeconomists' research in the 1980s and 1990s was largely
shaped by pressing macroeconomic issues of the period, such as European
high rates of unemployment or the challenges presented by the European
integration (fiscal convergence, monetary union, etc.). Beyond examining
these specific objects, many European macroeconomists also adopted a
specific perspective on these problems by utilising a political economy
framework.

Political economy, which encompasses various research lines research
lines that emerged in the 1970s, can be understood as the application of
modern economic analysis techniques (such optimisation or game theory)
to examine the impact of politics on economics.\footnote{For a history
  of the emergence of the ``new political economy'' or ``new political
  macroeconomics'' label, see Galvão de~Almeida
  (\protect\hyperlink{ref-galvaodealmeida2021}{2021}).} As Allan Drazen
defined it in his 2000 handbook, \emph{Political Economy in
Macroeconomics}, the ``new political economy'' focused on understanding
``how political constraints may explain the choice of policies (and thus
economic outcomes) that differ from optimal policies''
(\protect\hyperlink{ref-drazen2002}{Drazen, 2002, p. 7}). In Europe,
Torsten Persson and Guido Tabellini
(\protect\hyperlink{ref-persson2002}{2002}) provided a detailed
introduction to political economy.\footnote{Torsten Persson obtained his
  PhD in 1982 at the Institute for International Economic Studies in
  Stockholm under the supervision of Lars Svensson and became professor
  in Stockholm in 1987. Tabellini graduated from Torino before moving to
  UCLA for his PhD. After an initial position at Stanford, he returned
  to Italy in 1990. In their book, they used the term ``political
  economics'' rather than ``political economy'' to avoid a supposed
  association with ``an alternative analytical approach, as if the
  traditional tools of analysis in economics were not appropriate to
  study political phenomena''
  (\protect\hyperlink{ref-persson2002}{Persson and Tabellini, 2002, p.
  2}).} They traced ``political economics'' back to three traditions
(2): \emph{(i)} ``the theory of macroeconomic policy'' inspired by
Lucas, which focused on the impact of policy decisions on macroeconomic
variables; \emph{(ii)} the public choice tradition of James Buchanan,
Gordon Tullock and Mancur Olson, which applied economic theories to
analyse political decision-making processes; and \emph{(iii)} the formal
analysis in political analysis, inspired by Riker, that employs
mathematical models especially to study voting and decision-making
processes.

That is this first tradition that lies at the core of European
macroeconomics in the 1990s. The integration of rational expectations in
the 1970s had highlight specific policy problems. The most emblematic
example is the time-consistency problem popularized by Kydland and
Prescott (\protect\hyperlink{ref-kydland1977}{1977}). The basic idea is
that the optimal policy at time \emph{t} differs from the optimal policy
at time \emph{t + s}, because policymakers have an interest to deceive
economic agents for the very benefit of these same agents. If agents are
rational, they will anticipate policymakers incentive, rendering the
optimal policy unattainable. The article thus prompted the question of
whether it is necessary to ``tie the hands'' of policymakers, leading to
numerous extensions, especially regarding central banks and the concepts
of credibility and reputation (\protect\hyperlink{ref-barro1983}{Barro
and Gordon, 1983a}, \protect\hyperlink{ref-barro1983c}{1983b}), or the
selection of central bankers as well as the formalisation of delegation
(\protect\hyperlink{ref-rogoff1985b}{Rogoff, 1985}). This literature
traces its origins to the US academic debates surrounding rational
expectations and the efficiency of macroeconomic policies in the 1970s
(\protect\hyperlink{ref-hoover1988}{Hoover, 1988, pp. 80--86}). However,
the articles cited above experiences an unusual citation trajectory:
after an initial surge of popularity and subsequent decline (as is
common for many influential articles), they saw a resurgence of
popularity in the 1990s (Figure \ref{fig:plot-political-economy}). This
renewed interest can be attributed to European macroeconomists who
increasingly cited these references more than their US counterparts
(Figure \ref{fig:plot-political-economy-europe}). Prominent European
macroeconomists, such as Persson, Svensson, Horn in Sweden, Daniel Cohen
in France, Dolado in Spain, or Francesco Giavazzi and Tabellini in
Italy, played a significant role in revitalising these ideas in the
context of European macroeconomic issues.

\begin{figure}[h]

{\centering \includegraphics[width=1\linewidth]{../../../../../../../../Mon Drive/data/EER/pictures/Graphs/g1_citations_bw} 

}

\caption{Share of articles citing political economy literature (5-year moving average)}\label{fig:plot-political-economy}
\end{figure}

\begin{figure}[h]

{\centering \includegraphics[width=1\linewidth]{../../../../../../../../Mon Drive/data/EER/pictures/Graphs/g1_over_citations_bw} 

}

\caption{Citation of political economy articles by European economists relatively to US-based economists (log of ratios on 7-year moving average)}\label{fig:plot-political-economy-europe}
\end{figure}

This interest of European economists for the new political economy
literature of the late 1970s and early 1980s is confirmed by both
bibliometric and topic modelling analyses. We find these references as
most cited references in different clusters and topics and they were
cited by many influential European contributions.\footnote{Again, we
  mean here articles written by European authors which were highly cited
  in one or several topics and clusters.} The cluster ``Political
Economics of Central Banks'' is one of the most European clusters while
being comparable in size to the cluster on ``Disequilibrium \& Keynesian
Macro''. Similarly, the topic 8 on credibility, optimal policy and
policy rule clearly represents a highly European topic. But the topic
modelling allows us to observe how the new political economy literature
infused many subjects in the 1990s.\footnote{We are able to observe that
  notably by looking for different topics what are the most cited
  references if the articles are written by European and if they are
  not. In many topics, the difference about the references cited is
  explained by the fact that Europeans refer more to political economy
  contributions.}

We can distinguish three areas where a political economy framework is
recurrently used by European macroeconomists in the 1990s and
constitutes a particularity of European macroeconomics in comparison to
the US. First, many discussions about the appropriate framework for
monetary policy involved political economy contributions. An important
contribution for European macroeconomics here is Giavazzi and Pagano's
(\protect\hyperlink{ref-giavazzi1988}{1988}) article published in the
EER. The authors questioned the advantages of adhering to the European
monetary system (EMS) for countries with higher rate of inflation. They
deal with the idea that the EMS would constitute a solution to the
time-consistency problem: it would ``tie the hands'' of high-inflation
countries which would have to keep their exchange rate stable, thus
reducing their incentive to generate surprise inflation and increasing
the credibility of monetary authorities in these countries. The adhesion
to the EMS thus ``parallels that in Rogoff
(\protect\hyperlink{ref-rogoff1985b}{1985}), who shows that the
non-cooperative rate of inflation can be reduced `through a system of
rewards and punishments which alters the incentives of the central
bank'\,'' (\protect\hyperlink{ref-giavazzi1988}{Giavazzi and Pagano,
1988, p. 1057}). The question was whether the adhesion to the EMS would
be ``welfare-improving'' for high-inflation countries
(\emph{ibid.}).\footnote{Giavazzi and Pagano's article constitutes a
  major reference for topic 3 and topic 8.} Still in the EER, Daniel
Laskar (\protect\hyperlink{ref-laskar1989}{1989}) also started from
Rogoff's (\protect\hyperlink{ref-rogoff1985b}{1985}) argument that
appointing a conservative central banker could be beneficial for
society, but extended the issue to a two-country model to discuss in
which cases appointing conservative central bankers in both countries
could be detrimental or beneficial to both countries.\footnote{We also
  observe another important way to approach the issue of the EMS, less
  empirical and a bit less framed in political economy terms, but still
  dealing with ``credibility'': the expectation from going out of the
  EMS and thus the credibility associated to some exchange rates in a
  target zone regime (\protect\hyperlink{ref-rose1994}{Rose and
  Svensson, 1994}; \protect\hyperlink{ref-svensson1993a}{Svensson,
  1993}).} Still regarding monetary policy framework, the monetary union
issue also stimulated contributions in the terms of political economy.
In his EER survey about the theoretical justification for the
convergence requirements of the Maastricht treaty, Paul distinguished
between two types of justification: \emph{(i)} ``the traditional theory
of optimum currency areas (OCA)'' and \emph{(ii)} ``the more recent `new
view' based on credibility issues''
(\protect\hyperlink{ref-degrauwe1996}{De Grauwe, 1996, pp. 1091--1092}).
Contrarily to the OCA theory, the second approach relied on the
intuition of the Barro-Gordon model. It analyses ``how countries can
gain (or loose) credibility by joining a monetary union'' and thus how
inflation rates would converge.\footnote{Whereas the OCA theory rather
  focused on the divergence in output and employment trends.} When
dealing with the monetary union issue, European macroeconomists favoured
the credibility approach and the OCA theory appeared less influential
(see notably topic 3).\footnote{Our goal is not to be fully
  comprehensive here. We can find another important discussion about
  monetary policy framework with a political economy taste on inflation
  targeting (see notably \protect\hyperlink{ref-svensson1997}{Svensson,
  1997}).}

A second area where a political economy framework was influential is the
issue of wage-setting. In the EER, Horn and Persson
(\protect\hyperlink{ref-horn1988}{1988}) studied the interaction between
exchange rate policy and the role of unions in wage-setting. If
devaluations used to maintain or increase competitiveness are followed
by compensatory wage increases, the effects on competitiveness are
cancelled and the economy is in a situation of a ``devaluation-wage
spiral'' (\protect\hyperlink{ref-horn1988}{Horn and Persson, 1988, p.
1621}). The point of departure of the authors' analysis is that ``if
wage setters are rational and forward-looking and understand the
objectives behind the government's exchange rate policy (\ldots) they
will anticipate exchange rate changes and take them into account in
their wage decisions'' (p.~1622). Their goal was thus to endogenise both
wage decisions and policy formation in a game-theoretic
framework.\footnote{Horn and Persson's
  (\protect\hyperlink{ref-horn1988}{1988}) article was an important
  reference for topic 3 (on monetary union), topic 6 (on exchange rate
  dynamics), topic 25 (on real wages, employment and contracts) and
  topic 8 (on strategic policy making issues).} Thorvadur Gylfasson and
Lindbeck's work on the links between wage-setting and monetary policy is
also enlightening here. In a 1984 article in the EER, they tried to
integrate together cost push and demand pull inflation in a Keynesian
framework taking into account the behaviour of aggregate supply and the
Phillips curve for wage formation
(\protect\hyperlink{ref-gylfason1984}{Gylfason and Lindbeck, 1984}). As
they acknowledged themselves, the issues raised by their model echoed
Malinvaud's (\protect\hyperlink{ref-malinvaud1977}{1977}) opposition
between Classical and Keynesian unemployment
(\protect\hyperlink{ref-gylfason1984}{Gylfason and Lindbeck, 1984, pp.
6--7}). Their article had a political economy flavour as they dealt with
``competing wage claims'' and framed their model as a duopoly problem
\emph{à la} Cournot. In their following article in the EER, they relied
explicitly on game theory to deal with the interaction of wages
determination and government spending
(\protect\hyperlink{ref-gylfason1986}{Gylfason and Lindbeck, 1986}).
Some years later, going back to the issue of wage setting and monetary
policy, they refer to the ``wage gaps'' debate of the 1970s and the
``cases where government efforts to reduce unemployment by bringing real
wages through price inflation were frustrated by subsequent nominal wage
increases'' (\protect\hyperlink{ref-gylfason1994}{Gylfason and Lindbeck,
1994, p. 34}), but no reference was made to classical and Keynesian
unemployment. They proposed a model in which wages are determined
``through collective bargaining among strong and well coordinated labor
unions'' (34) and explored its consequences for monetary policy in a
game-theoretic model similar to Barro and Gordon
(\protect\hyperlink{ref-barro1983}{1983a},
\protect\hyperlink{ref-barro1983c}{1983b}). To some extent, the
trajectory of Gylfason and Lindbeck's work is representative of the
transformation of European macroeconomics between the 1970s and the
1990s.

A third area concerns fiscal policy and European integration. Alberto
Alesina, Tabellini and Persson defended in the late 1980s and early
1990s the development of a ``positive theory'' of fiscal policy. The two
first explained in the AER that the goal was to ``{[}abandon{]} the
assumption that fiscal policy is set by a benevolent social planner who
maximizes the welfare of a representative consumer \ldots{} {[}for{]} an
economy with two policymakers with different objectives alternating in
office as a result of elections''
(\protect\hyperlink{ref-alesina1990}{Alesina and Tabellini, 1990}).
Persson and Tabellini (\protect\hyperlink{ref-persson1992}{1992})
defended a similar ``positive public finance'' research agenda. Their
goal was to understand how the rising European integration and the
removal of barriers to the mobility of capital, goods and labour could
affect the ``politico-economic equilibrium that determines fiscal
policy'' (\protect\hyperlink{ref-persson1992}{Persson and Tabellini,
1992, p. 689}).\footnote{There references are important for European
  economists in Topic 36, in comparison to US-based economists. We find
  a distinction similar in topic 22, where US economists mainly cited
  endogenous growth references, when Europeans were sticking to the
  political economy literature.}

As the three examples testify, after the late 1980s, (new) political
economy and its pioneering works
(\protect\hyperlink{ref-barro1983}{Barro and Gordon, 1983a},
\protect\hyperlink{ref-barro1983c}{1983b};
\protect\hyperlink{ref-kydland1977}{Kydland and Prescott, 1977};
\protect\hyperlink{ref-rogoff1985b}{Rogoff, 1985}) represented a
unifying framework for many European macroeconomists to deal with
different macroeconomic issues. It constituted a resource for tackling
the issues raised by the European integration and the building of a
European monetary system.

\hypertarget{conclusion}{%
\section*{Conclusion}\label{conclusion}}
\addcontentsline{toc}{section}{Conclusion}

Despite the widespread internationalisation and standardisation of
economics after the 1970s, European macroeconomics maintained
characteristic features distinct from US macroeconomics between the
1970s and 1990s. These features were of different nature. In the late
1970s and early 1980s, disequilibrium theory constituted a significant
part of the research undertaken by European macroeconomists, and did not
limit to the general equilibrium theory, but also represented a unifying
framework to deal with different macroeconomic issues (unemployment,
stagflation, stabilization policies, international trade, etc.).
Disequilibrium theory represented an alternative research program to the
challenges raised by new classical economists in the US and
macroeconomics in Europe took for some years another path. Nonetheless,
disequilibrium theory did not succeed in seducing US macroeconomists
and, from a unifying research framework in Europe, it progressively
became a minor and declining research program after the mid-1980s. In
the same time, new classical economists' contributions eventually
encountered some success in Europe, notably with the import of Real
Business Cycle modelling, but above all with the influence of the new
political economy literature of Kydland and Prescott
(\protect\hyperlink{ref-kydland1977}{1977}) and Barro and Gordon
(\protect\hyperlink{ref-barro1983}{1983a},
\protect\hyperlink{ref-barro1983c}{1983b}).

In the 1990s, European macroeconomics appeared closer to its US
counterpart on the theoretical and methodological levels. However, it
does not mean that no difference existed and European macroeconomists
specialised on subjects that were only occasionally tackled (if at all)
by their US colleagues. The different macroeconomic situation of
European economies (particularly regarding unemployment), the issue of
economic interdependence between them, and the construction of the
European union probably pushed European economists towards different
avenues. And the importance of political economy in European
macroeconomics in the 1990s probably reflects these different incentives
for European researchers.

\newpage

\hypertarget{references}{%
\section*{References}\label{references}}
\addcontentsline{toc}{section}{References}

\hypertarget{refs}{}
\begin{CSLReferences}{1}{0}
\leavevmode\vadjust pre{\hypertarget{ref-alesina1990}{}}%
Alesina, A., Tabellini, G., 1990. A {Positive Theory} of {Fiscal
Deficits} and {Government Debt}. The Review of Economic Studies 57,
403--414. doi:\href{https://doi.org/10.2307/2298021}{10.2307/2298021}

\leavevmode\vadjust pre{\hypertarget{ref-azariadis1975}{}}%
Azariadis, C., 1975. Implicit contracts and underemployment equilibria.
Journal of Political Economy 83, 1183--1202.

\leavevmode\vadjust pre{\hypertarget{ref-backhouse1997a}{}}%
Backhouse, R.E., 1997. The changing character of {British} economics.
History of Political Economy 28, 33--60.

\leavevmode\vadjust pre{\hypertarget{ref-backhouseboianovski2013}{}}%
Backhouse, R.E., Boianovski, M., 2013. Tranforming modern
macroeconomics. {Exploring} disequilibrium microfoundations (1956-2003).
{Cambridge University Press}, {Cambdrige}.

\leavevmode\vadjust pre{\hypertarget{ref-barnett1984}{}}%
Barnett, W.A., 1984. On the flexibility of the {Rotterdam} model: {A}
first empirical look. European Economic Review 24, 285--289.
doi:\href{https://doi.org/10.1016/0014-2921(84)90057-6}{10.1016/0014-2921(84)90057-6}

\leavevmode\vadjust pre{\hypertarget{ref-barro1976}{}}%
Barro, R.J., 1976. Rational expectations and the role of monetary
policy. Journal of Monetary Economics 2, 1--32.
doi:\href{https://doi.org/10.1016/0304-3932(76)90002-7}{10.1016/0304-3932(76)90002-7}

\leavevmode\vadjust pre{\hypertarget{ref-barro1983}{}}%
Barro, R.J., Gordon, D.B., 1983a. Rules, discretion and reputation in a
model of monetary policy. Journal of Monetary Economics 12, 101--121.
doi:\href{https://doi.org/10.1016/0304-3932(83)90051-X}{10.1016/0304-3932(83)90051-X}

\leavevmode\vadjust pre{\hypertarget{ref-barro1983c}{}}%
Barro, R.J., Gordon, D.B., 1983b. A {Positive Theory} of {Monetary
Policy} in a {Natural Rate Model}. Journal of Political Economy 91,
589--610.

\leavevmode\vadjust pre{\hypertarget{ref-barro1971}{}}%
Barro, R.J., Grossman, H.I., 1971. A {General Disequilibrium Model} of
{Income} and {Employment}. The American Economic Review 61, 82--93.

\leavevmode\vadjust pre{\hypertarget{ref-barten1964}{}}%
Barten, A.P., 1964. Consumer demand functions under conditions of almost
additive preferences. Econometrica: Journal of the Econometric Society
1--38.

\leavevmode\vadjust pre{\hypertarget{ref-barten1969}{}}%
Barten, A.P., 1969. Maximum likelihood estimation of a complete system
of demand equations. European Economic Review 1, 7--73.
doi:\href{https://doi.org/10.1016/0014-2921(69)90017-8}{10.1016/0014-2921(69)90017-8}

\leavevmode\vadjust pre{\hypertarget{ref-barten1976}{}}%
Barten, A.P., d'Alcantara, G., Carrin, G.J., 1976. {COMET}: {A}
medium-term macroeconomic model for the {European} economic community.
European Economic Review 7, 63--115.
doi:\href{https://doi.org/10.1016/0014-2921(76)90019-2}{10.1016/0014-2921(76)90019-2}

\leavevmode\vadjust pre{\hypertarget{ref-baumol1952}{}}%
Baumol, W.J., 1952. The transactions demand for cash: {An} inventory
theoretic approach. The Quarterly journal of economics 545--556.

\leavevmode\vadjust pre{\hypertarget{ref-bean1994}{}}%
Bean, C.R., 1994a. European unemployment: {A} retrospective. European
Economic Review 38, 523--534.
doi:\href{https://doi.org/10.1016/0014-2921(94)90088-4}{10.1016/0014-2921(94)90088-4}

\leavevmode\vadjust pre{\hypertarget{ref-bean1994a}{}}%
Bean, C.R., 1994b. European unemployment: A survey. Journal of economic
literature 32, 573--619.

\leavevmode\vadjust pre{\hypertarget{ref-bentolila1990}{}}%
Bentolila, S., Bertola, G., 1990. Firing costs and labour demand: How
bad is eurosclerosis? The Review of Economic Studies 57, 381--402.

\leavevmode\vadjust pre{\hypertarget{ref-bentolila1992a}{}}%
Bentolila, S., Saint-Paul, G., 1992. The macroeconomic impact of
flexible labor contracts, with an application to {Spain}. European
Economic Review 36, 1013--1047.
doi:\href{https://doi.org/10.1016/0014-2921(92)90043-V}{10.1016/0014-2921(92)90043-V}

\leavevmode\vadjust pre{\hypertarget{ref-bertola1990a}{}}%
Bertola, G., 1990. Job security, employment and wages. European Economic
Review 34, 851--879.
doi:\href{https://doi.org/10.1016/0014-2921(90)90066-8}{10.1016/0014-2921(90)90066-8}

\leavevmode\vadjust pre{\hypertarget{ref-bischof2012}{}}%
Bischof, J., Airoldi, E.M., 2012. Summarizing topical content with word
frequency and exclusivity. p. 201208.

\leavevmode\vadjust pre{\hypertarget{ref-blei2007}{}}%
Blei, D.M., Lafferty, J.D., 2007.
\href{https://www.jstor.org/stable/4537420}{A correlated topic model of
science}. The Annals of Applied Statistics 1, 17--35.

\leavevmode\vadjust pre{\hypertarget{ref-boumans2019}{}}%
Boumans, M., Duarte, P.G., 2019. The {History} of {Macroeconometric
Modeling} : {An Introduction}. History of Political Economy 51,
391--400.

\leavevmode\vadjust pre{\hypertarget{ref-brunosachs1985}{}}%
Bruno, M., Sachs, J.D., 1985. Economics of worldwide stagflation.
{National Bureau of Economic Research}, {Cambridge, Mass.}

\leavevmode\vadjust pre{\hypertarget{ref-bryant1979}{}}%
Bryant, J., Wallace, N., 1979. The inefficiency of interest-bearing
national debt. Journal of Political Economy 87, 365--381.

\leavevmode\vadjust pre{\hypertarget{ref-byron1984}{}}%
Byron, R.P., 1984. On the flexibility of the {Rotterdam} model. European
Economic Review 24, 273--283.
doi:\href{https://doi.org/10.1016/0014-2921(84)90056-4}{10.1016/0014-2921(84)90056-4}

\leavevmode\vadjust pre{\hypertarget{ref-calmfors1988}{}}%
Calmfors, L., Driffill, J., 1988. Bargaining {Structure}, {Corporatism}
and {Macroeconomic Performance}. Economic Policy 3, 13.
doi:\href{https://doi.org/10.2307/1344503}{10.2307/1344503}

\leavevmode\vadjust pre{\hypertarget{ref-cherrier2018c}{}}%
Cherrier, B., Saïdi, A., 2018. The indeterminate fate of sunspots in
economics. History of Political Economy 50, 425--481.

\leavevmode\vadjust pre{\hypertarget{ref-cherrier2021}{}}%
Cherrier, B., Saïdi, A., 2021. Back to {Front}: {The Role} of
{Seminars}, {Conferences} and {Workshops} in the {History} of
{Economics}. {Preface} to the {Second Issue}. Revue d'economie politique
131, 723--727.

\leavevmode\vadjust pre{\hypertarget{ref-claveau2016}{}}%
Claveau, F., Gingras, Y., 2016.
\href{http://hope.dukejournals.org/cgi/content/short/48/4/551?rss=1}{Macrodynamics
of economics: A bibliometric history}. History of Political Economy.

\leavevmode\vadjust pre{\hypertarget{ref-clements2015a}{}}%
Clements, K.W., Gao, G., 2015. The {Rotterdam} demand model half a
century on. Economic Modelling 49, 91--103.
doi:\href{https://doi.org/10.1016/j.econmod.2015.03.019}{10.1016/j.econmod.2015.03.019}

\leavevmode\vadjust pre{\hypertarget{ref-clower1965}{}}%
Clower, R., 1965. The {Keynesian Counter-Revolution}: {A Theoretical
Appraisal}, in: The {Theory} of {Interest Rates}.

\leavevmode\vadjust pre{\hypertarget{ref-coats1996}{}}%
Coats, A.W., 1996. The Post-1945 Internationalization of Economics. Duke
University Press.

\leavevmode\vadjust pre{\hypertarget{ref-davidson1981}{}}%
Davidson, J.E.H., Hendry, D.F., 1981. Interpreting econometric evidence.
European Economic Review 16, 177--192.
doi:\href{https://doi.org/10.1016/0014-2921(81)90058-1}{10.1016/0014-2921(81)90058-1}

\leavevmode\vadjust pre{\hypertarget{ref-davidson1978}{}}%
Davidson, J.E.H., Hendry, D.F., Srba, F., Yeo, S., 1978. Econometric
{Modelling} of the {Aggregate Time-Series Relationship Between
Consumers}' {Expenditure} and {Income} in the {United Kingdom}. The
Economic Journal 88, 661--692.
doi:\href{https://doi.org/10.2307/2231972}{10.2307/2231972}

\leavevmode\vadjust pre{\hypertarget{ref-degrauwe1996}{}}%
De Grauwe, P., 1996. Monetary union and convergence economics. European
Economic Review, Papers and {Proceedings} of the {Tenth Annual Congress}
of the {European Economic Association} 40, 1091--1101.
doi:\href{https://doi.org/10.1016/0014-2921(95)00117-4}{10.1016/0014-2921(95)00117-4}

\leavevmode\vadjust pre{\hypertarget{ref-devroey2016}{}}%
De Vroey, M., 2016. A history of modern macroeconomics from keynes to
lucas and beyond. Cambridge University Press, Cambridge.

\leavevmode\vadjust pre{\hypertarget{ref-deaton1980}{}}%
Deaton, A., Muellbauer, J., 1980. An almost ideal demand system. The
American economic review 70, 312--326.

\leavevmode\vadjust pre{\hypertarget{ref-diamond1982}{}}%
Diamond, P.A., 1982. Aggregate demand management in search equilibrium.
Journal of political Economy 90, 881--894.

\leavevmode\vadjust pre{\hypertarget{ref-diebold2003}{}}%
Diebold, F.X., 2003. The {ET} interview: {Professor} robert f. Engle,
january 2003. Econometric Theory 19, 1159--1193.

\leavevmode\vadjust pre{\hypertarget{ref-dixit1978}{}}%
Dixit, A., 1978. The balance of trade in a model of temporary
equilibrium with rationing. The Review of Economic Studies 45, 393--404.

\leavevmode\vadjust pre{\hypertarget{ref-dixit1980}{}}%
Dixit, A., Norman, V., 1980. Theory of {International Trade}: {A Dual},
{General Equilibrium Approach}. {Cambridge University Press}.

\leavevmode\vadjust pre{\hypertarget{ref-drazen2002}{}}%
Drazen, A., 2002. Political {Economy} in {Macroeconomics}, Nouvelle. ed.
{Princeton University Press}, {New Haven}.

\leavevmode\vadjust pre{\hypertarget{ref-dreze1981}{}}%
Drèze, J.H., Modigliani, F., 1981. The trade-off between real wages and
employment in an open economy ({Belgium}). European Economic Review 15,
1--40.
doi:\href{https://doi.org/10.1016/0014-2921(81)90065-9}{10.1016/0014-2921(81)90065-9}

\leavevmode\vadjust pre{\hypertarget{ref-duartelima2012a}{}}%
Duarte, P.G., Lima, G.T., 2012. Microfoundations {Reconsidered}. {The
Relationship} of {Micro} and {Macroeconomics} in {Historical
Perspective}. {Edward Elgar Publishing}, {Cheltenham}.

\leavevmode\vadjust pre{\hypertarget{ref-duppe2017}{}}%
Düppe, T., 2017. How modern economics learned french: Jacques drèze and
the foundation of CORE. The European Journal of the History of Economic
Thought 24, 238--273.

\leavevmode\vadjust pre{\hypertarget{ref-engle1987}{}}%
Engle, R.F., Granger, C.W., 1987. Co-integration and error correction:
Representation, estimation, and testing. Econometrica: journal of the
Econometric Society 251--276.

\leavevmode\vadjust pre{\hypertarget{ref-engle1983}{}}%
Engle, R.F., Hendry, D.F., Richard, J.-F., 1983. Exogeneity.
Econometrica 51, 277--304.
doi:\href{https://doi.org/10.2307/1911990}{10.2307/1911990}

\leavevmode\vadjust pre{\hypertarget{ref-ericsson2004}{}}%
Ericsson, N.R., 2004. The {ET} interview: {Professor David F}. {Hendry}:
{Interviewed} by neil {R}. {Ericsson}. Econometric Theory 20, 743--804.

\leavevmode\vadjust pre{\hypertarget{ref-fama1975}{}}%
Fama, E.F., 1975. Short-term interest rates as predictors of inflation.
American Economic Review 65.

\leavevmode\vadjust pre{\hypertarget{ref-fourcade2006}{}}%
Fourcade, M., 2006. The construction of a global profession: The
transnationalization of economics. American Journal of Sociology 112,
145194.

\leavevmode\vadjust pre{\hypertarget{ref-fourcade2009}{}}%
Fourcade, M., 2009. Economists and societies: Discipline and profession
in the united states, britain, and france, 1890s to 1990s. Princeton
University Press, Princeton.

\leavevmode\vadjust pre{\hypertarget{ref-frasser2020}{}}%
Frasser, C., 2020. Essays on liquidity-based asset classification and
illegal means of payment (PhD thesis). Université Paris 1
Panthéon-Sorbonne, Paris.

\leavevmode\vadjust pre{\hypertarget{ref-frenkel1976}{}}%
Frenkel, J.A., Johnson, H.G. (Eds.), 1976. The monetary approach to the
balance of payments. {G. Allen \& Unwin}, {London}.

\leavevmode\vadjust pre{\hypertarget{ref-friedman1957}{}}%
Friedman, M., 1957. A theory of the consumption function. Princeton
University Press, Princeton.

\leavevmode\vadjust pre{\hypertarget{ref-friedman1963}{}}%
Friedman, M., Schwartz, A.J., 1963. A {Monetary} history of the {United}
{States} 1867-1960, Studies in business cycles. Princeton university
press, Princeton.

\leavevmode\vadjust pre{\hypertarget{ref-galvaodealmeida2021}{}}%
Galvão de~Almeida, R., 2021. A {Macroeconomic View} of {Public Choice}:
{New Political Macroeconomics} as a {Separate Tradition} of {Public
Choice}. Œconomia. History, Methodology, Philosophy 77--105.
doi:\href{https://doi.org/10.4000/oeconomia.10402}{10.4000/oeconomia.10402}

\leavevmode\vadjust pre{\hypertarget{ref-giavazzi1988}{}}%
Giavazzi, F., Pagano, M., 1988. The advantage of tying one's hands:
{EMS} discipline and central bank credibility. European economic review
32, 1055--1075.

\leavevmode\vadjust pre{\hypertarget{ref-gottfries1992}{}}%
Gottfries, N., 1992. Insiders, outsiders, and nominal wage contracts.
Journal of Political Economy 100, 252--270.

\leavevmode\vadjust pre{\hypertarget{ref-gottfries1987}{}}%
Gottfries, N., Horn, H., 1987. Wage formation and the persistence of
unemployment. The Economic Journal 97, 877--884.

\leavevmode\vadjust pre{\hypertarget{ref-goutsmedt2021b}{}}%
Goutsmedt, A., 2021. From the {Stagflation} to the {Great Inflation}:
{Explaining} the {US} economy of the 1970s. Revue d'Economie Politique
131, 557--582.

\leavevmode\vadjust pre{\hypertarget{ref-goutsmedtetal2019}{}}%
Goutsmedt, A., Pinzon-Fuchs, E., Renault, M., Sergi, F., 2019. Reacting
to the lucas critique: {The} keynesians' replies. History of Political
Economy 51, 533--556.
doi:\href{https://doi.org/10.1215/00182702-7551912}{10.1215/00182702-7551912}

\leavevmode\vadjust pre{\hypertarget{ref-goutsmedt2021}{}}%
Goutsmedt, A., Renault, M., Sergi, F., 2021. European {Economics} and
the {Early Years} of the {``{International Seminar} on
{Macroeconomics}.''} Revue d'Economie Politique 131, 693--722.

\leavevmode\vadjust pre{\hypertarget{ref-grubb1982}{}}%
Grubb, D., Jackman, R., Layard, R., 1982. Causes of the {Current
Stagflation}. The Review of Economic Studies 49, 707--730.
doi:\href{https://doi.org/10.2307/2297186}{10.2307/2297186}

\leavevmode\vadjust pre{\hypertarget{ref-grubb1983a}{}}%
Grubb, D., Jackman, R., Layard, R., 1983. Wage rigidity and unemployment
in {OECD} countries. European Economic Review 21, 11--39.

\leavevmode\vadjust pre{\hypertarget{ref-gylfason1984}{}}%
Gylfason, T., Lindbeck, A., 1984. Competing wage claims, cost inflation,
and capacity utilization. European Economic Review 24, 1--21.
doi:\href{https://doi.org/10.1016/0014-2921(84)90010-2}{10.1016/0014-2921(84)90010-2}

\leavevmode\vadjust pre{\hypertarget{ref-gylfason1986}{}}%
Gylfason, T., Lindbeck, A., 1986. Endogenous unions and governments: {A}
game-theoretic approach. European Economic Review 30, 5--26.
doi:\href{https://doi.org/10.1016/0014-2921(86)90029-2}{10.1016/0014-2921(86)90029-2}

\leavevmode\vadjust pre{\hypertarget{ref-gylfason1994}{}}%
Gylfason, T., Lindbeck, A., 1994. The interaction of monetary policy and
wages. Public Choice 79, 33--46.

\leavevmode\vadjust pre{\hypertarget{ref-hagemann2011a}{}}%
Hagemann, H., 2011. European émigrés and the {``{Americanization}''} of
economics. The European Journal of the History of Economic Thought 18,
643--671.
doi:\href{https://doi.org/10.1080/09672567.2011.629056}{10.1080/09672567.2011.629056}

\leavevmode\vadjust pre{\hypertarget{ref-hall1978b}{}}%
Hall, R.E., 1978. Stochastic implications of the life cycle-permanent
income hypothesis: Theory and evidence. Journal of political economy 86,
971--987.

\leavevmode\vadjust pre{\hypertarget{ref-hart1982}{}}%
Hart, O., 1982. A model of imperfect competition with {Keynesian}
features. The Quarterly Journal of Economics 97, 109--138.

\leavevmode\vadjust pre{\hypertarget{ref-hesse2012}{}}%
Hesse, J.-O., 2012. The {``{Americanisation}''} of {West German}
economics after the {Second World War}: {Success}, failure, or something
completely different? The European Journal of the History of Economic
Thought 19, 67--98.
doi:\href{https://doi.org/10.1080/09672567.2010.487283}{10.1080/09672567.2010.487283}

\leavevmode\vadjust pre{\hypertarget{ref-hoover1988}{}}%
Hoover, K.D. (Ed.), 1988. The new classical macroeconomics, The
{International} library of critical writings in economics. E. Elgar,
Aldershot (GB) Brookfield (Vt.).

\leavevmode\vadjust pre{\hypertarget{ref-hoover2012}{}}%
Hoover, K.D., 2012. Microfoundational programs, in: Duarte, P.G., Lima,
G.T. (Eds.), Microfoundations {Reconsidered}. Edward Elgar Publishing,
Cheltenham, pp. 19--61.

\leavevmode\vadjust pre{\hypertarget{ref-horn1988}{}}%
Horn, H., Persson, T., 1988. Exchange rate policy, wage formation and
credibility. European Economic Review 32, 1621--1636.
doi:\href{https://doi.org/10.1016/0014-2921(88)90021-9}{10.1016/0014-2921(88)90021-9}

\leavevmode\vadjust pre{\hypertarget{ref-jel1991}{}}%
JEL, 1991. \href{https://www.jstor.org/stable/2727351}{Classification
system: Old and new categories}. Journal of Economic Literature 29,
xviii--xxviii.

\leavevmode\vadjust pre{\hypertarget{ref-johansen1988}{}}%
Johansen, S., 1988. Statistical analysis of cointegration vectors.
Journal of Economic Dynamics and Control 12, 231--254.
doi:\href{https://doi.org/10.1016/0165-1889(88)90041-3}{10.1016/0165-1889(88)90041-3}

\leavevmode\vadjust pre{\hypertarget{ref-kiyotaki1989}{}}%
Kiyotaki, N., Wright, R., 1989. On money as a medium of exchange.
Journal of political Economy 97, 927--954.

\leavevmode\vadjust pre{\hypertarget{ref-kiyotaki1993}{}}%
Kiyotaki, N., Wright, R., 1993. A search-theoretic approach to monetary
economics. The American Economic Review 63--77.

\leavevmode\vadjust pre{\hypertarget{ref-kloek2001}{}}%
Kloek, T., 2001. Obituary: {Henri Theil}, 1924\textendash 2000.
Statistica Neerlandica 55, 263--269.
doi:\href{https://doi.org/10.1111/1467-9574.00169}{10.1111/1467-9574.00169}

\leavevmode\vadjust pre{\hypertarget{ref-kydland1977}{}}%
Kydland, F.E., Prescott, E.C., 1977. Rules {Rather} than {Discretion}:
{The Inconsistency} of {Optimal Plans}. Journal of Political Economy 85,
473--491.

\leavevmode\vadjust pre{\hypertarget{ref-laskar1989}{}}%
Laskar, D., 1989. Conservative central bankers in a two-country world.
European Economic Review 33, 1575--1595.
doi:\href{https://doi.org/10.1016/0014-2921(89)90079-2}{10.1016/0014-2921(89)90079-2}

\leavevmode\vadjust pre{\hypertarget{ref-layard1986}{}}%
Layard, R., Nickell, S., 1986. Unemployment in {Britain}. Economica 53,
S121--S169. doi:\href{https://doi.org/10.2307/2554377}{10.2307/2554377}

\leavevmode\vadjust pre{\hypertarget{ref-layard1991a}{}}%
Layard, R., Nickell, S., Jackman, R., 1991. Unemployment: {Macroeconomic
Performance} and the {Labour Market}. {Oxford University Press},
{Oxford}.
doi:\href{https://doi.org/10.1093/acprof:oso/9780199279166.001.0001}{10.1093/acprof:oso/9780199279166.001.0001}

\leavevmode\vadjust pre{\hypertarget{ref-kuleuven2016}{}}%
Leuven, K., 2016. Anton {Barten} (1930-2016). Obituary - KU Leuven.

\leavevmode\vadjust pre{\hypertarget{ref-lindbeck1986}{}}%
Lindbeck, A., Snower, D.J., 1986. Wage setting, unemployment, and
insider-outsider relations. The American Economic Review 76, 235--239.

\leavevmode\vadjust pre{\hypertarget{ref-lindbeck1987a}{}}%
Lindbeck, A., Snower, D.J., 1987. Efficiency wages versus insiders and
outsiders. European Economic Review 31, 407--416.
doi:\href{https://doi.org/10.1016/0014-2921(87)90058-4}{10.1016/0014-2921(87)90058-4}

\leavevmode\vadjust pre{\hypertarget{ref-lucas1973}{}}%
Lucas, R.E., 1973. \href{http://www.jstor.org/stable/1914364}{Some
international evidence on output-inflation tradeoffs}. The American
Economic Review 326--334.

\leavevmode\vadjust pre{\hypertarget{ref-lucas1972}{}}%
Lucas, R.E., Jr., 1972. Expectations and the neutrality of money.
Journal of Economic Theory 4, 103--124.
doi:\href{https://doi.org/10.1016/0022-0531(72)90142-1}{10.1016/0022-0531(72)90142-1}

\leavevmode\vadjust pre{\hypertarget{ref-maes2005}{}}%
Maes, I., Buyst, E., 2005. Migration and americanization: The special
case of belgian economics. The European Journal of the History of
Economic Thought 12, 7388.

\leavevmode\vadjust pre{\hypertarget{ref-malinvaud1977}{}}%
Malinvaud, E., 1977. The theory of unemployment reconsidered.
{Basil-Blackwell}, {Oxford}.

\leavevmode\vadjust pre{\hypertarget{ref-malinvaud1986}{}}%
Malinvaud, E., 1986. The rise of unemployment in {France}. Economica 53,
S197--S217.

\leavevmode\vadjust pre{\hypertarget{ref-morgan1998}{}}%
Morgan, M.S., Rutherford, M., 1998.
\href{http://search.ebscohost.com/login.aspx?direct=true\&db=bth\&AN=7752144\&lang=fr\&site=ehost-live}{American
economics: The character of the transformation}. History of Political
Economy 30, 1--26.

\leavevmode\vadjust pre{\hypertarget{ref-mortensen1994}{}}%
Mortensen, D.T., Pissarides, C.A., 1994. Job creation and job
destruction in the theory of unemployment. The review of economic
studies 61, 397415.

\leavevmode\vadjust pre{\hypertarget{ref-persson1992}{}}%
Persson, T., Tabellini, G., 1992. The {Politics} of 1992: {Fiscal
Policy} and {European Integration}. The Review of Economic Studies 59,
689--701. doi:\href{https://doi.org/10.2307/2297993}{10.2307/2297993}

\leavevmode\vadjust pre{\hypertarget{ref-persson2002}{}}%
Persson, T., Tabellini, G.E., 2002. Political economics: Explaining
economic policy. {MIT press}.

\leavevmode\vadjust pre{\hypertarget{ref-plassard2021}{}}%
Plassard, R., Renault, M., Rubin, G., 2021. Modelling market dynamics :
{Jean-Pascal Bénassy}, {Edmond Malinvaud}, and the development of
disequilibrium macroeconomics. Modelling market dynamics : Jean-Pascal
Bénassy, Edmond Malinvaud, and the development of disequilibrium
macroeconomics 83--114.
doi:\href{https://doi.org/10.19272/202106101005}{10.19272/202106101005}

\leavevmode\vadjust pre{\hypertarget{ref-portes1987}{}}%
Portes, R., 1987. Economics in europe. European Economic Review 31,
1329--1340.
doi:\href{https://doi.org/10.1016/S0014-2921(87)80021-1}{10.1016/S0014-2921(87)80021-1}

\leavevmode\vadjust pre{\hypertarget{ref-qin2013a}{}}%
Qin, D., 2013. {A History of Econometrics: The Reformation from the
1970s}. {Oxford University Press}, {Oxford}.

\leavevmode\vadjust pre{\hypertarget{ref-renault2020a}{}}%
Renault, M., 2020. Edmond {Malinvaud}'s {Criticisims} of the {New
Classical Economics}: {Restoring} the {Nature} and the {Rationale} of
the {Old Keynesians}' {Opposition}. Journal of the History of Economic
Thought 42, 563--585.
doi:\href{https://doi.org/10.1017/S1053837219000610}{10.1017/S1053837219000610}

\leavevmode\vadjust pre{\hypertarget{ref-renault2019}{}}%
Renault, M., 2022b. Theory to the {Rescue} of the {Large-Scale Models}:
{Edmond Malinvaud}'s {Alternative View} on the {Search} for
{Microfoundations}. History of Political Economy 54.

\leavevmode\vadjust pre{\hypertarget{ref-renault2022}{}}%
Renault, M., 2022a. Theory to the {Rescue} of the {Large-Scale Models}:
{Edmond Malinvaud}'s {Alternative View} on the {Search} for
{Microfoundations}. History of Political Economy 54.

\leavevmode\vadjust pre{\hypertarget{ref-rogoff1985b}{}}%
Rogoff, K., 1985. The optimal degree of commitment to a monetary target.
Quarterly Journal of Economics 100, 1169--1190.

\leavevmode\vadjust pre{\hypertarget{ref-rose1994}{}}%
Rose, A.K., Svensson, L.E.O., 1994. European exchange rate credibility
before the fall. European Economic Review 38, 1185--1216.
doi:\href{https://doi.org/10.1016/0014-2921(94)90067-1}{10.1016/0014-2921(94)90067-1}

\leavevmode\vadjust pre{\hypertarget{ref-sandelin1997}{}}%
Sandelin, B., Ranki, S., 1997. Internationalization or americanization
of swedish economics? The European Journal of the History of Economic
Thought 4, 284--298.
doi:\href{https://doi.org/10.1080/10427719700000040}{10.1080/10427719700000040}

\leavevmode\vadjust pre{\hypertarget{ref-sargent1975}{}}%
Sargent, T.J., Wallace, N., 1975.
\href{http://econpapers.repec.org/article/ucpjpolec/v_3A83_3Ay_3A1975_3Ai_3A2_3Ap_3A241-54.htm}{"{Rational}"
{Expectations}, the {Optimal} {Monetary} {Instrument}, and the {Optimal}
{Money} {Supply} {Rule}}. Journal of Political Economy 83, 241--54.

\leavevmode\vadjust pre{\hypertarget{ref-sargent1982}{}}%
Sargent, T.J., Wallace, N., 1982.
\href{http://www.jstor.org/stable/1830945}{The real-bills doctrine
versus the quantity theory: {A} reconsideration}. The Journal of
Political Economy 90, 1212--1236.

\leavevmode\vadjust pre{\hypertarget{ref-shen2019}{}}%
Shen, S., Zhu, D., Rousseau, R., Su, X., Wang, D., 2019. A refined
method for computing bibliographic coupling strengths. Journal of
Informetrics 13, 605--615.
doi:\href{https://doi.org/10.1016/j.joi.2019.01.012}{10.1016/j.joi.2019.01.012}

\leavevmode\vadjust pre{\hypertarget{ref-sneessens1986}{}}%
Sneessens, H.R., Drèze, J.H., 1986. A {Discussion} of {Belgian
Unemployment}, {Combining Traditional Concepts} and {Disequilibrium
Econometrics}. Economica 53, S89--S119.
doi:\href{https://doi.org/10.2307/2554376}{10.2307/2554376}

\leavevmode\vadjust pre{\hypertarget{ref-snowdon2005}{}}%
Snowdon, B., Vane, H.R., 2005. Modern macroeconomics: Its origins,
development and current state. {E. Elgar}, {Cheltenham (GB)}.

\leavevmode\vadjust pre{\hypertarget{ref-svensson1993a}{}}%
Svensson, L.E.O., 1993. Assessing target zone credibility: {Mean}
reversion and devaluation expectations in the {ERM},
1979\textendash 1992. European Economic Review 37, 763--793.
doi:\href{https://doi.org/10.1016/0014-2921(93)90087-Q}{10.1016/0014-2921(93)90087-Q}

\leavevmode\vadjust pre{\hypertarget{ref-svensson1997}{}}%
Svensson, L.E.O., 1997. Inflation forecast targeting: {Implementing} and
monitoring inflation targets. European Economic Review 41, 1111--1146.
doi:\href{https://doi.org/10.1016/S0014-2921(96)00055-4}{10.1016/S0014-2921(96)00055-4}

\leavevmode\vadjust pre{\hypertarget{ref-theil1965}{}}%
Theil, H., 1965. The information approach to demand analysis,
Econometrica.

\leavevmode\vadjust pre{\hypertarget{ref-traag2019}{}}%
Traag, V.A., Waltman, L., van Eck, N.J., 2019. From louvain to leiden:
Guaranteeing well-connected communities. Scientific reports 9, 112.

\leavevmode\vadjust pre{\hypertarget{ref-trejos1995}{}}%
Trejos, A., Wright, R., 1995. Search, bargaining, money, and prices.
Journal of political Economy 103, 118--141.

\leavevmode\vadjust pre{\hypertarget{ref-truc2021}{}}%
Truc, A., Claveau, F., Santerre, O., 2021. Economic methodology: a
bibliometric perspective. Journal of Economic Methodology 1--12.
doi:\href{https://doi.org/10.1080/1350178X.2020.1868774}{10.1080/1350178X.2020.1868774}

\leavevmode\vadjust pre{\hypertarget{ref-waelbroeck1969}{}}%
Waelbroeck, J.L., Glejser, H., 1969. Editor's introduction. European
Economic Review 1, 3--6.
doi:\href{https://doi.org/10.1016/0014-2921(69)90016-6}{10.1016/0014-2921(69)90016-6}

\end{CSLReferences}

\newpage

\hypertarget{appendices}{%
\section*{Appendices}\label{appendices}}
\addcontentsline{toc}{section}{Appendices}

\hypertarget{a---summary-tables}{%
\subsection*{A - Summary Tables}\label{a---summary-tables}}
\addcontentsline{toc}{subsection}{A - Summary Tables}

\hypertarget{r-summary-communities-communities_table---communities-selectlabel_com-sum_diff_odds_windows-arrangedescsum_diff_odds_windows-communities_table-kblcaptionsummary-of-bibliographic-clusters-booktabs-true-longtable-false-col.names-ccommunities-differences-format-latex-alignlr-kable_stylinglatex_options-cstriped-repeat_header-hold_position-font_size-9-alt-index-communities_table_log_by_cluster---communities-selectlabel_com-sum_diff_odds-arrangedescsum_diff_odds-communities_table_chi2---communities_chi-selectlabel_com.x-sum_diff-arrangedescsum_diff}{%
\section{\texorpdfstring{\texttt{\{r\ summary-communities\}\ \#\ communities\_table\ \textless{}-\ communities\ \%\textgreater{}\%\ select(Label\_com,\ sum\_diff\_odds\_windows)\ \%\textgreater{}\%\ arrange(desc(sum\_diff\_odds\_windows))\ \#\ \ \#\ communities\_table\ \%\textgreater{}\%\ \#\ \ \ kbl(caption="Summary\ of\ Bibliographic\ Clusters",\ \#\ \ \ \ \ \ \ booktabs\ =\ TRUE,\ \#\ \ \ \ \ \ \ longtable\ =\ FALSE,\ \#\ \ \ \ \ \ \ col.names\ =c("Communities",\ "Differences"),\ \#\ \ \ \ \ \ \ format=\ "latex",\ \#\ \ \ \ \ \ \ align="lr")\ \%\textgreater{}\%\ \#\ \ \ kable\_styling(latex\_options\ =\ c("striped",\ "repeat\_header",\ "hold\_position"),\ \#\ \ \ \ \ \ \ \ \ \ \ \ \ \ \ \ \ font\_size\ =\ 9)\ \#\ \ \#\ \#\ alt\ index\ \#\ communities\_table\_log\_by\_cluster\ \textless{}-\ communities\ \%\textgreater{}\%\ select(Label\_com,\ sum\_diff\_odds)\ \%\textgreater{}\%\ arrange(desc(sum\_diff\_odds))\ \#\ communities\_table\_chi2\ \textless{}-\ communities\_chi\ \%\textgreater{}\%\ select(Label\_com.x,\ sum\_diff)\ \%\textgreater{}\%\ arrange(desc(sum\_diff))\ \#\ \ \#}}{\{r summary-communities\} \# communities\_table \textless- communities \%\textgreater\% select(Label\_com, sum\_diff\_odds\_windows) \%\textgreater\% arrange(desc(sum\_diff\_odds\_windows)) \#  \# communities\_table \%\textgreater\% \#   kbl(caption="Summary of Bibliographic Clusters", \#       booktabs = TRUE, \#       longtable = FALSE, \#       col.names =c("Communities", "Differences"), \#       format= "latex", \#       align="lr") \%\textgreater\% \#   kable\_styling(latex\_options = c("striped", "repeat\_header", "hold\_position"), \#                 font\_size = 9) \#  \# \# alt index \# communities\_table\_log\_by\_cluster \textless- communities \%\textgreater\% select(Label\_com, sum\_diff\_odds) \%\textgreater\% arrange(desc(sum\_diff\_odds)) \# communities\_table\_chi2 \textless- communities\_chi \%\textgreater\% select(Label\_com.x, sum\_diff) \%\textgreater\% arrange(desc(sum\_diff)) \#  \#}}\label{r-summary-communities-communities_table---communities-selectlabel_com-sum_diff_odds_windows-arrangedescsum_diff_odds_windows-communities_table-kblcaptionsummary-of-bibliographic-clusters-booktabs-true-longtable-false-col.names-ccommunities-differences-format-latex-alignlr-kable_stylinglatex_options-cstriped-repeat_header-hold_position-font_size-9-alt-index-communities_table_log_by_cluster---communities-selectlabel_com-sum_diff_odds-arrangedescsum_diff_odds-communities_table_chi2---communities_chi-selectlabel_com.x-sum_diff-arrangedescsum_diff}}

\begingroup\fontsize{9}{11}\selectfont

\begin{longtable}[t]{>{}l>{}r>{\raggedright\arraybackslash}m{25em}}
\caption{\label{tab:summary-topics}Summary of Topics}\\
\toprule
Topics & Differences & Terms with the highest frex value\\
\midrule
\endfirsthead
\caption[]{Summary of Topics \textit{(continued)}}\\
\toprule
Topics & Differences & Terms with the highest frex value\\
\midrule
\endhead

\endfoot
\bottomrule
\endlastfoot
Topic 3 & 2.037 & system;
monetary
expansion;
monetary
system;
union;
expansion;
\cellcolor{gray!6}{stability}\\
Topic 4 & 1.695 & macroeconomics;
rich;
history;
robert;
divide;
lead\\
Topic 46 & 1.244 & german;
money
demand;
germany;
unite
kingdom;
unite;
\cellcolor{gray!6}{kingdom}\\
Topic 22 & 1.033 & political;
oecd
country;
oecd;
world;
country;
index\\
Topic 37 & 0.994 & unemployment;
job;
unemployment
rate;
creation;
flow;
phillips
\cellcolor{gray!6}{curve}\\
\addlinespace
Topic 6 & 0.767 & real
exchange;
real
exchange
rate;
exchange
rate;
flexible
exchange;
flexible
exchange
rate;
target
zone\\
Topic 25 & 0.756 & real
wage;
contract;
employment;
capacity;
wage;
\cellcolor{gray!6}{stickiness}\\
Topic 39 & 0.745 & trade
balance;
trade;
wealth;
relative
price;
balance;
external\\
Topic 8 & 0.708 & credibility;
strategic;
policy;
maker;
economic
policy;
policy
\cellcolor{gray!6}{rule}\\
Topic 28 & 0.699 & exchange
market;
foreign
exchange
market;
intervention;
foreign
exchange;
transaction;
transaction
cost\\
\addlinespace
Topic 13 & 0.615 & unanticipated;
activity;
economic
activity;
national;
economy;
\cellcolor{gray!6}{gap}\\
Topic 44 & 0.454 & level;
national
income;
inflationary;
equation;
price
level;
money
balance\\
Topic 31 & 0.374 & welfare
cost;
cost;
welfare;
bear;
survey;
\cellcolor{gray!6}{household}\\
Topic 43 & 0.369 & federal;
signal;
monetary
policy;
federal
reserve;
revision;
feed\\
Topic 23 & 0.349 & short
run;
run;
short;
burden;
indirect;
\cellcolor{gray!6}{externality}\\
\addlinespace
Topic 20 & 0.340 & inflation
target;
target;
stabilize;
central
bank;
central;
length\\
Topic 26 & 0.333 & inflation;
inflation
rate;
relative;
evidence;
dispersion;
nominal
\cellcolor{gray!6}{price}\\
Topic 45 & 0.258 & power
parity;
purchase
power
parity;
purchase
power;
power;
purchase;
parity\\
Topic 40 & 0.256 & economic
growth;
growth
rate;
productivity
growth;
growth;
fast;
\cellcolor{gray!6}{region}\\
Topic 36 & 0.208 & spend;
government
spend;
deficit;
fiscal;
government;
government
debt\\
\addlinespace
Topic 17 & 0.139 & indexation;
distortion;
labor
market;
labor;
product;
\cellcolor{gray!6}{corporate}\\
Topic 11 & 0.134 & walrasian;
competitive;
temporary;
existence;
search;
equilibrium\\
Topic 35 & 0.033 & likelihood;
variable;
estimation;
autoregressive;
variance;
endogenous
\cellcolor{gray!6}{variable}\\
Topic 24 & 0.002 & term;
spread;
short
term;
term
structure;
premium;
structure\\
Topic 15 & -0.031 & production;
class;
factor;
identical;
preference;
\cellcolor{gray!6}{input}\\
\addlinespace
Topic 50 & -0.097 & budget
constraint;
constraint;
project;
budget;
bad;
loan\\
Topic 30 & -0.146 & business
cycle;
business;
cycle;
real
business
cycle;
real
business;
\cellcolor{gray!6}{volatility}\\
Topic 47 & -0.160 & investment;
monopolistic;
dynamic;
competition;
macroeconomic;
replace\\
Topic 38 & -0.253 & asset
price;
financial
market;
stock
market;
return;
asset
market;
\cellcolor{gray!6}{stock}\\
Topic 21 & -0.258 & process;
procedure;
property;
incentive;
build;
endogenous\\
\addlinespace
Topic 42 & -0.296 & stationary;
rational
expectation
equilibrium;
expectation
equilibrium;
expectation;
unique;
rational
\cellcolor{gray!6}{expectation}\\
Topic 5 & -0.310 & price
adjustment;
oil;
price;
commodity
price;
sticky;
import\\
Topic 9 & -0.333 & skill;
asymmetric
information;
program;
change;
research;
\cellcolor{gray!6}{complementarity}\\
Topic 29 & -0.425 & capital
market;
mobility;
capital
mobility;
capital;
imperfect;
intensity\\
Topic 18 & -0.603 & generation;
overlap;
overlap
generation;
social
security;
live;
generation
\cellcolor{gray!6}{model}\\
\addlinespace
Topic 34 & -0.605 & plan;
stage;
multiple
equilibrium;
option;
crisis;
currency\\
Topic 33 & -0.664 & depression;
theory;
subsequent;
classical;
pure;
\cellcolor{gray!6}{principle}\\
Topic 48 & -0.715 & liquidity;
credit;
debt;
insurance;
access;
investor\\
Topic 10 & -0.799 & tax;
capital
income;
income
tax;
tax
system;
redistribution;
income
\cellcolor{gray!6}{taxation}\\
Topic 27 & -0.808 & perfect
foresight;
foresight;
time
vary;
time;
perfect;
continuous
time\\
\addlinespace
Topic 7 & -0.812 & control;
stochastic;
game;
equivalence;
equivalent;
\cellcolor{gray!6}{solution}\\
Topic 16 & -0.845 & public;
strategy;
finance;
local;
provision;
desirable\\
Topic 41 & -1.025 & optimal;
optimal
tax;
growth
model;
function;
optimal
policy;
optimal
\cellcolor{gray!6}{taxation}\\
Topic 12 & -1.339 & lm;
risk
aversion;
utility
function;
aversion;
intertemporal;
risk\\
Topic 49 & -1.427 & report;
composition;
regime;
critique;
puzzle;
\cellcolor{gray!6}{profit}\\
\addlinespace
Topic 1 & -1.608 & inventory;
hold;
association;
century;
create;
rationally\\
Topic 14 & -1.645 & income
distribution;
labor
income;
permanent;
sensitivity;
permanent
income;
\cellcolor{gray!6}{income}\\
Topic 32 & -1.746 & standard;
gold;
dollar;
reserve;
price
level;
size\\
Topic 2 & -1.916 & money
supply;
money
stock;
money;
fix
exchange;
supply;
fix
exchange
\cellcolor{gray!6}{rate}\\
Topic 19 & -1.927 & expect
rate;
cash;
sargent;
nominal;
expect
inflation;
expect\\*
\end{longtable}
\endgroup{}

\newpage

\hypertarget{appendix}{%
\subsection*{B - Information on the Methods}\label{appendix}}
\addcontentsline{toc}{subsection}{B - Information on the Methods}

\hypertarget{corpus}{%
\subsubsection*{B.1. Corpus Creation}\label{corpus}}
\addcontentsline{toc}{subsubsection}{B.1. Corpus Creation}

For the present study we used two different corpora. The first corpus is
composed of all EER articles and allows us to track how publications,
citations, references and authors affiliations evolved since the
creation of the journal in 1969 up to 2002. The second corpus is
composed of all macroeconomic articles published in the top five
economics journals (\emph{American Economic Review}, \emph{Journal of
Political Economy}, \emph{Econometrica}, \emph{Quarterly Journal of
Economics}, \emph{Review of Economic Studies}) and the EER.
Macroeconomic articles are identified thanks to the former and new
classification of the JEL codes (\protect\hyperlink{ref-jel1991}{JEL,
1991}).\footnote{See \ref{eer-top5-macro} for the list of JEL codes
  used.} This corpus is used as the basis for topic modelling and
bibliographic coupling analysis to contrast macroeconomics publications
authored by Europe-based and US-based authors, and/or published in top 5
journals and in the EER.

\hypertarget{eer-publications}{%
\paragraph*{EER Publications}\label{eer-publications}}
\addcontentsline{toc}{paragraph}{EER Publications}

For the creation of the first corpus composed of all EER articles, we
used a mix of \emph{Web of Science} (WoS) and \emph{Scopus}. While WoS
has all articles of the EER between 1969-1970 and 1974-2002, it is
missing most articles published between 1971 and 1973. To make up for
the missing data, we use Scopus to complete the dataset. This operation
required normalization of the Scopus dataset, and manual cleaning of
variables that were missing from Scopus compared to WoS. This mostly
includes cleaning the references to match \emph{Scopus} references with
WoS ones, and identification of author's affiliation.

\hypertarget{eer-top5-macro}{%
\paragraph*{EER and Top 5 Macroeconomics
Articles}\label{eer-top5-macro}}
\addcontentsline{toc}{paragraph}{EER and Top 5 Macroeconomics Articles}

The construction of this corpus is made in multiple steps:

\begin{enumerate}
\def\labelenumi{\arabic{enumi}.}
\tightlist
\item
  Identifying macroeconomics articles
\end{enumerate}

\begin{itemize}
\item
  We identified all articles published in macroeconomics using JEL codes
  related to macroeconomics (we get JEL codes of Top 5 and EER articles
  thanks to the Econlit database). We consider that an article is a
  macroeconomics article if it has one of the following codes:
\item
  For old JEL codes (pre-1991): 023, 131, 132, 133, 134, 223, 311, 313,
  321, 431, 813, 824.
\item
  For new JEL codes (1991 onward): all E, F3 and F4.\footnote{The new
    classification has a clear categorisation of Macroeconomics (the
    letter `E'), but we had F3 and F4 as they deal with international
    macroeconomics. For the older JEL codes, we use the table of
    correspondence produce by the \emph{Journal of Economic Literature}
    itself (\protect\hyperlink{ref-jel1991}{JEL, 1991}).}.
\end{itemize}

\begin{enumerate}
\def\labelenumi{\arabic{enumi}.}
\setcounter{enumi}{1}
\tightlist
\item
  Using these JEL codes, we match econlit articles with WoS articles
  using the following matching variables:
\end{enumerate}

\begin{itemize}
\tightlist
\item
  Journal, Volume, First Page
\item
  Year, Journal, First Page, Last Page
\item
  Year, Volume, First Page, Last Page
\item
  First Author, Year, Volume, First Page
\item
  First Author, Title, Year
\item
  Title, Year, First Page
\end{itemize}

\begin{enumerate}
\def\labelenumi{\arabic{enumi}.}
\setcounter{enumi}{2}
\item
  We then kept articles published in the EER (Corpus 1 improved with
  Scopus), and in the top five journals between 1973 and 2002. Out of
  the 3592 articles in econlit, we matched 3428. \footnote{Most of the
    unmatched articles are not `articles' properly speaking: they often
    are reply and comments on other published articles.}
\item
  Finally, we were able to collect abstracts:
\end{enumerate}

\begin{itemize}
\tightlist
\item
  using \emph{Scopus} for the EER. All abstracts have been matched with
  the EER corpus.
\item
  using \emph{Microsoft Academics} to collect the highest number of
  available abstracts for the Top 5 as too many abstracts were missing
  in WoS or \emph{Scopus}. The abstracts extracted from this database
  are matched with our WoS Top 5 corpus using
\end{itemize}

Moreover, given that the size of our corpus is modest, we made an
extensive semi-automatic cleaning of references to improve references
identification by adding the most commonly cited books, book chapter,
and articles that are not otherwise identified in WoS when possible.

\hypertarget{b.2.-variable-creation}{%
\subsubsection*{B.2. Variable creation}\label{b.2.-variable-creation}}
\addcontentsline{toc}{subsubsection}{B.2. Variable creation}

\hypertarget{author-affiliation}{%
\paragraph*{Authors' affiliation}\label{author-affiliation}}
\addcontentsline{toc}{paragraph}{Authors' affiliation}

Authors' affiliations information were extracted from WoS. However, the
affiliations are not per author, but instead per institutional
departments per paper. For example, in the case of an article with two
authors from the same department, the department (and institution or
country associated with it) is only counted once. Similarly, a
single-authored article where the author has three affiliations can
result in one article having three affiliations. While in some cases we
can inferred the institutional affiliation for each author (e.g., one
institution, multiple authors), in others we cannot (e.g., two
institutions, three authors). For example, in an article with two
authors from Princeton and one author from Stanford, we only know that
the article was written by at least one author from Princeton and at
least one from Stanford, but not that the individual ratio was two
third.

For the descriptive analysis we simply use the count of unique
combinations of institution and country per article, and use occurrences
as an approximation of affiliation. However, for the more detailed
network and topic analysis, we restructured the information. given that
we are mostly interested in the relationship between Europe and US
economics, we simply looked at the share of papers authored by
Europe-based and US-based economists. While we do not have individual
affiliation, we know with certainty when a paper has only European
authors, only American authors, or a mix of the two. For this reason,
while the share of institutions within the corpus is only an estimation
based on the occurrences of affiliation, the information generated to
identify US authored papers and European authored paper is certain.

\hypertarget{network}{%
\subsubsection*{B.3. Bibliographic Coupling and Cluster
Detection}\label{network}}
\addcontentsline{toc}{subsubsection}{B.3. Bibliographic Coupling and
Cluster Detection}

\hypertarget{main-index}{%
\paragraph*{Main Index}\label{main-index}}
\addcontentsline{toc}{paragraph}{Main Index}

A first way to identify potential differences between European and
American macroeconomics is to find articles written by Europeans and
published in a European journal, the EER, resembling each others but
dissimilar to American articles. To do that, we used bibliographic
coupling techniques. In a bibliographic coupling network, a link is
created between two articles when they have one or more references in
common. The more references that two articles have in common, the
stronger the link. Bibliographic coupling is one way to measure how
similar two articles are in a corpus. To normalize and weight the link
between two articles, we used the refined bibliographic coupling
strength of Shen et al.
(\protect\hyperlink{ref-shen2019}{2019}).\footnote{We have implemented
  this method in the \emph{biblionetwork} R package: Aurélien Goutsmedt,
  François Claveau and Alexandre Truc (2021). biblionetwork: A Package
  For Creating Different Types of Bibliometric Networks. R package
  version 0.0.0.9000.} This method normalized and weight the strength
between articles by taking into account two important elements

\begin{enumerate}
\def\labelenumi{\arabic{enumi}.}
\tightlist
\item
  The size of the bibliography of the two linked articles. It means that
  common references between two articles with long bibliography are
  weighted as less significant since the likeliness of potential common
  references is higher. Conversely, common references between two
  articles with a short bibliography is weighted as more significant.
\item
  The number of occurrences of each reference in the overall corpus.
  When a reference is shared between two articles, it is weighted as
  less significant if it is a very common reference across the entire
  corpus and very significant if it is scarcely cited. The assumption is
  that a very rare common reference points to a higher content
  similarity between two articles than a highly cited reference.
\end{enumerate}

For all macroeconomics articles published in the EER and in the Top 5,
we build the networks with 8-year overlapping windows. This results in
23.

We use Leiden detection algorithm
(\protect\hyperlink{ref-traag2019}{Traag et al., 2019}) that optimize
the modularity on each network to identify groups of articles that are
similar to each other and dissimilar to the rest of the network. We use
a resolution of 1 with 1000 iterations. This results in 466 clusters
across all networks. Because networks have a lot of overlaps, many
clusters between two periods are composed of the same articles. To
identify these clusters that are very similar between two time windows,
we considered that \emph{(i)} if at least 55\% of the articles in a
cluster of the first time window where in the same cluster in the second
time window, and that \emph{(ii)} if the cluster was also composed by at
least 55\% of articles of the first time window, \emph{then} it is the
same cluster. Simply put, if two clusters share a high number of
articles, and are both mostly composed by these shared articles, they
are considered the same cluster.

This gives us 154 clusters, with 33 that represent at least 4\% of a
network and are stable enough to exists for at least 2 time windows. We
are thus able to project the composition of each network and how nodes
circulated between clusters from one time window to the following one.

\begin{figure}[h]

{\centering \includegraphics[width=1\linewidth]{../../../../../../../../Mon Drive/data/EER/pictures/Graphs/Intertemporal_communities} 

}

\caption{The distribution of bibliographic clusters over time}\label{fig:plot-cluster-flow}
\end{figure}

For each cluster, we identify the US or European oriented nature of its
publications and authors. A first measure we used is the over/under
representation of European/US authors in the cluster. For each time
window, we computed the \emph{(i)} share of articles written by European
authors and US authors in each cluster, and \emph{(ii)} the overall
share of European authors and US authors. Then, for each cluster, we
computed the \emph{(i)} mean share of European authors and US authors,
and \emph{(ii)} the overall mean share of European authors and US
authors, across all time windows in which the cluster exists. When then
compared these mean share using the log of ratios

\bigskip

\({\scriptstyle \text{Author EU/US Orientation}=\log(\frac{(\frac{\text{Mean Share Of European Authored Articles In The Cluster}}{\text{Mean Share Of US Authored Articles In The Cluster}})} {(\frac{\text{Overall Mean Share Of European Authored Articles in Cluster's Time Window}}{\text{Overall Mean Share Of US Authored Articles in Cluster's Time Window}})})}\)
\bigskip

We then use a second similar index for the publication venue of the
articles in the cluster:

\bigskip

\({\scriptstyle \text{Journal EU/US Orientation}=\log(\frac{(\frac{\text{Mean Share Of EER Articles In The Cluster}}{\text{1-Mean Share Of EER Articles In The Cluster}})} {(\frac{\text{Overall Mean Share of EER Articles in Cluster's Time Window}}{\text{1-Overall Mean Share of EER Articles in Cluster's Time Window}})})}\)
\bigskip

Finally, clusters are placed on a scatterplot with the X-axis for the
\emph{Author EU/US Orientation} score, and the X-Axis for the
\emph{Journal EU/US Orientation} score. The size of the points captures
the number of articles in the cluster (see Figure
\ref{fig:plot-community-diff}).

Supplementary information about each cluster can be found in the online
appendix ``Bibliographic information about the EER and details on the
bibliographic coupling clusters''.

\hypertarget{alt-index}{%
\paragraph*{Alternative Indexes}\label{alt-index}}
\addcontentsline{toc}{paragraph}{Alternative Indexes}

In addition to the main index, we propose a few alternative indexes:

\begin{itemize}
\tightlist
\item
  \textbf{Mean of Log of Ratios Index} (see Figure
  \ref{fig:plot-mean-log-ratios}): this index computes a very similar
  log ratio as the main index, but the \emph{Author EU/US Orientation}
  and \emph{Journal EU/US Orientation} scores are computed on each
  individual window for each clusters, and the overall score of each
  cluster is the mean of the score across all time windows in which the
  cluster exists. However, one issue is that the score is not computed
  for time windows in which the clusters have 0 observations for some
  modalities.
\item
  \textbf{All Nodes Ratios Index} (see Figure
  \ref{fig:plot-all-nodes-ratios}): this index computes a very similar
  log ratio as the main index, but the \emph{Author EU/US Orientation}
  and \emph{Journal EU/US Orientation} scores are computed for each
  cluster using all nodes across all time windows in which the cluster
  exists. In other words, for each cluster, we take all the nodes that
  are the clusters, and all the nodes that are in the time windows of
  each cluster, and compute the log of ratios for the \emph{Author EU/US
  Orientation} and \emph{Journal EU/US Orientation} scores. One
  difference with the main index is that more weights is given to time
  windows in which the cluster is bigger while the main index is simply
  the mean across all time windows no matter the size of the cluster in
  each time window.
\item
  \textbf{Chi2 Index} (see Figure \ref{fig:plot-chi-two-mean}): this
  index is very different from all other indexes. For each time window,
  we made a chi-square test of independence. We then used the adjusted
  standardized residuals computed for each time window to investigate
  the difference between expected and observed frequency for each
  modality of our variables. A positive residual indicates that the
  observed frequency is greater than what was expected, while a negative
  residual indicates that the observed frequency is less than what was
  expected. For each cluster, we then used the average residual across
  all time windows in which the cluster exists. For the \emph{Author
  EU/US Orientation} score with four modalities, the frequency of
  \emph{European Authored Articles} was used for the residual.
\end{itemize}

\begin{figure}[h]

{\centering \includegraphics[width=1\linewidth]{../../../../../../../../Mon Drive/data/EER/pictures/Graphs/Communities_europeanisation_mean_of_log_ratio} 

}

\caption{Citation of political economy articles by European economists relatively to US-based economists (log of ratios on 7-year moving average)}\label{fig:plot-mean-log-ratios}
\end{figure}

\begin{figure}[h]

{\centering \includegraphics[width=1\linewidth]{../../../../../../../../Mon Drive/data/EER/pictures/Graphs/Communities_europeanisation_odds_bw} 

}

\caption{Citation of political economy articles by European economists relatively to US-based economists (log of ratios on 7-year moving average)}\label{fig:plot-all-nodes-ratios}
\end{figure}

\begin{figure}[h]

{\centering \includegraphics[width=1\linewidth]{../../../../../../../../Mon Drive/data/EER/pictures/Graphs/Communities_europeanisation__chi_mean} 

}

\caption{Citation of political economy articles by European economists relatively to US-based economists (log of ratios on 7-year moving average)}\label{fig:plot-chi-two-mean}
\end{figure}

\hypertarget{topic}{%
\subsubsection*{B.4. Topic Modelling}\label{topic}}
\addcontentsline{toc}{subsubsection}{B.4. Topic Modelling}

\hypertarget{preprocessing}{%
\paragraph*{Preprocessing}\label{preprocessing}}
\addcontentsline{toc}{paragraph}{Preprocessing}

Our text corpus is composed of the titles and abstracts (when available)
of macroeconomics articles published in the Top 5 and EER. We have
several steps to clean our corpus before running our topic models:

\begin{enumerate}
\def\labelenumi{\arabic{enumi}.}
\tightlist
\item
  Titles and abstracts are merged together for all EER and Top 5
  articles.
\item
  We use the \emph{tidytext} and \emph{tokenizers} R packages to
  `tokenise' the resulting texts (when there is no abstract, only the
  title is thus tokenise)?\footnote{See Silge J, Robinson D (2016).
    ``tidytext: Text Mining and Analysis Using Tidy Data Principles in
    R.'' \emph{JOSS}, \emph{1}(3) and Lincoln A. Mullen et al., ``Fast,
    Consistent Tokenization of Natural Language Text,'' Journal of Open
    Source Software 3, no.23 (2018): 655.} Tokenisation is the process
  of transforming human-readable text into machine readable objects.
  Here, the text is split in unique words (unigrams), bigrams (pair of
  words) and trigrams. In other words, to each article is now associated
  a list of unigrams, bigrams and trigrams, some appearing several times
  in the same title \emph{plus} abstract.
\item
  Stop words are removed using the \emph{Snowball}
  dictionary.\footnote{See
    \url{http://snowball.tartarus.org/algorithms/english/stop.txt}.} We
  add to this dictionary some common verbs in abstract like
  ``demonstrate'', ``show'', ``explain''. Such verbs are likely to be
  randomly distributed in abstracts, and we want to limit the noise as
  much as possible.
\item
  We lemmatise the words using the \emph{textstem} package.\footnote{Rinker,
    T. W. (2018). textstem: Tools for stemming and lemmatizing text
    version 0.1.4. Buffalo, New York.} The lemmatisation is the process
  of grouping words together according to their ``lemma'' which depends
  on the context. For instance, different form of a verb are reduced to
  its infinitive form. The plural of nouns are reduced to the singular.
\end{enumerate}

\hypertarget{choosing-the-number-of-topics}{%
\paragraph*{Choosing the number of
topics}\label{choosing-the-number-of-topics}}
\addcontentsline{toc}{paragraph}{Choosing the number of topics}

We use the Correlated Topic Model (\protect\hyperlink{ref-blei2007}{Blei
and Lafferty, 2007}) method implemented in the \emph{STM} R
package.\footnote{Roberts ME, Stewart BM, Tingley D (2019). ``stm: An R
  Package for Structural Topic Models.'' \emph{Journal of Statistical
  Software}, \emph{91}(2), 1-40.}

From the list of words we have tokenised, cleaned and lemmatised, we
test different thresholds and choices by running different models:

\begin{itemize}
\tightlist
\item
  by exluding trigrams or not;
\item
  by removing the terms that are present in less than 0.6\% of the
  Corpus (20 articles), 0.8\% (27) and 1\% (34);
\item
  by removing articles with less than 8 words or with less than 12
  words.\footnote{Here, only articles with no abstract are impacted.}
\end{itemize}

Crossing all these criteria, we thus have 12 different possible
combinations. For each of these 12 different combinations, we have run
topic models for different number of topics from 20 to 110 with a gap of
5. The chosen model integrates trigrams, removes only terms that appear
in less than 0,6\% of the documents and keep all articles if they have
more than 8 words in their title \emph{plus} abstract. We choose to keep
the model with 50 topics.

We have chosen the criteria and the number of topics by comparing the
performance of the different models in terms of the FREX value
(\protect\hyperlink{ref-bischof2012}{Bischof and Airoldi, 2012}). We
have tested alternative specification for preprocessing steps and
different number of topics when the performance regarding FREX values
was similar. It seems to us that 50 topics allows us to have a model
with easily understandable topics and an interesting level of ``zoom''.
Indeed, increasing the number of topics just splits some topics in two,
but did not lead to fundamentally different results.

For each cluster, we are able to plot the distribution of the years of
publications of article, depending on their \emph{gamma} value for the
corresponding topic.

\begin{figure}[h]

{\centering \includegraphics[width=1\linewidth]{../../../../../../../../Mon Drive/data/EER/pictures/Graphs/topic_per_year} 

}

\caption{The distribution of topics over time}\label{fig:plot-topic-year}
\end{figure}

\hypertarget{studying-the-european-character-of-topics}{%
\paragraph*{Studying the European character of
topics}\label{studying-the-european-character-of-topics}}
\addcontentsline{toc}{paragraph}{Studying the European character of
topics}

To look at the features of the topics regarding our two variables of
interest (EER \emph{vs.} Top 5 publications and US authors \emph{vs.}
European authors), we use two methods. The first one keeps all articles
and, for each topic, calculate the average gamma value for articles
published in the EER and in the Top 5. We subtract the two means. We do
the same for articles written by European authors only and by US authors
only. The two resulting differences are plot in the following Figure
\ref{fig:plot-topic-diff-alternative}.

\begin{figure}[h]

{\centering \includegraphics[width=1\linewidth]{../../../../../../../../Mon Drive/data/EER/pictures/Graphs/mean_diff_plot_new_bw} 

}

\caption{The most European topics (Differences of mean method)}\label{fig:plot-topic-diff-alternative}
\end{figure}

In Figure \ref{fig:plot-topic-diff} in the text above, we are only
keeping, for each topic, articles with a \emph{gamma} value above 0.1.
We then calculate the log ratio of EER and Top 5 articles for each
topic:

\bigskip

\({\scriptstyle \text{Journal EU/US Orientation}=\log(\frac{\frac{\text{Share Of EER Articles In The Topic}}{\text{Total Share Of EER Articles}}} {\frac{\text{Share Of Top 5 Articles In The Topic}}{\text{Total Share Of Top 5 Articles}}})}\)

\bigskip

We do the same for articles written by Europe-based and US-based
authors:

\bigskip

\({\scriptstyle \text{Author EU/US Orientation}=\log(\frac{\frac{\text{Share Of European Authored Articles In The Topic}}{\text{Total Share Of European Authored Articles}}} {\frac{\text{Share Of US Authored Articles In The Topic}}{\text{Total Share Of US Authored Articles}}})}\)

\end{document}
